\documentclass{amsart}
\usepackage{preamble}

\begin{document}
\section*{\textbf{Appendix: Elliptic regularity for first order
    operators}}
\subsection*{First order operators}
\begin{defn}
  Let $E_{1},E_{2}$ be complex vector bundles over $Y$.
  
  A first order differential operator is a $\C$-linear map
  \begin{equation*}
    L:\Gamma(E_{1})\to \Gamma(E_{2})
  \end{equation*}
  with the property that for all $\C$-valued functions $f$
  \begin{equation*}
    [L,f]=L\circ f-f\circ L:\Gamma(E_{1})\to \Gamma(E_{2})
  \end{equation*}
  is given by a tensor, i.e.\ is induced by a section of $\Hom_{\C}(E_{1},E_{2})$. In this definition we use the fact that $\Gamma(E_{i})$ are modules over the ring $\Omega^{0}(\C)$.  
\end{defn}
\begin{xca}
  Prove that first order differential operators are \textbf{local} in the sense that if $s\in \Gamma(E_{1})$ is supported in a closed set $K$, then $L(s)$ is also supported in $K$.

  Use this to show that $L$ is induced by a map of sheaves $\Gamma(E_{1};-)\to \Gamma(E_{2};-)$.
\end{xca} 
\begin{example}[The structure of differential operators]
  Let $s_{1},\cdots,s_{n}$ and $r_{1},\cdots,r_{m}$ be local frames for $\Gamma(E_{1})$ and $\Gamma(E_{2})$, respectively. Let $x_{1},\cdots,x_{d}$ be local coordinates on the base. If $L$ is a first order differential operator then
  \begin{equation*}
    L(\sum_{i=1}^{n}a_{i}s_{i})=\sum_{i=1}^{n}[L,a_{i}]s_{i}+a_{i}Ls_{i}.
  \end{equation*}
  We can compute the tensor $[L,a_{i}]_{x}\in \Hom(E_{1,x},E_{2,x})$ using the following trick. Near a given point $p_{0}$, say $x(p_{0})=c$ we can write
  \begin{equation*}
    a_{i}(p)=a_{i}(p_{0})+\sum_{j=1}^{n}b_{ij}(p)(x_{j}(p)-c_{j}),
  \end{equation*}
  for some smooth functions $b_{ij}$. Moreover it is clear that
  \begin{equation*}
    \d a_{i}(p_{0})=\sum_{j}b_{ij}(p_{0})\d x_{j}(p_{0})\iff \pd{a_{i}}{x_{j}}=b_{ij}(p_{0}).
  \end{equation*}
  Now near $p_{0}$ write
  \begin{equation*}
    \begin{aligned}
      [L,a_{i}]=[L,a_{i}(p_{0})]+\sum_{j=1}^{n}[L,b_{ij}(x_{j}-c_{j})]&=0+\sum_{j=1}^{n}b_{ij}[L,x_{j}-c_{j}]+[L,b_{ij}](x_{j}-c_{j})\\
      &=\sum_{j=1}^{n}b_{ij}[L,x_{j}]+[L,b_{ij}](x_{j}-c_{j})
    \end{aligned}
      \end{equation*}
  Evaluating this tensor at $p_{0}$ yields (recall $x_{j}(p_{0})=c_{j}$) 
  \begin{equation*}
    [L,a_{i}](p_{0})=\sum_{j=1}^{n}b_{ij}(p_{0})[L,x_{j}]=\sum_{j=1}^{n}\pd{a_{i}}{x_{j}}(p_{0})[L,x_{j}].
  \end{equation*}
  But $p_{0}$ was arbitrary. It follows that
  \begin{equation*}
    L(s)=L(\sum_{i=1}^{n}a_{i}s_{i})=\sum_{i,j}\pd{a_{i}}{x_{j}}[L,x_{j}]s_{i}+a_{i}L(s_{i}).
  \end{equation*}
  In the presence of a connection, we can use the above to say that locally
  \begin{equation*}
    L(s)=A(\nabla s)+B(s),
  \end{equation*}
  where $A,B$ are tensors:
  \begin{equation*}
    A\in \Gamma \Hom(\Hom(TY,E_{1}),E_{2}) \text{ and }B\in \Gamma\Hom(E_{1},E_{2}).
  \end{equation*}
  We claim this actually holds globally. To see this, cover $Y$ by a locally finite open cover $\set{U_{\alpha}}$, with associated partition of unity $\set{\rho_{\alpha}}$. Assume that for sections $s_{\alpha}$ supported in $U_{\alpha}$ we have
  \begin{equation*}
    L(s_{\alpha})=A_{\alpha}(\nabla s_{\alpha})+B_{\alpha}(s_{\alpha}).
  \end{equation*}
  Then for any $s$
  \begin{equation*}
    L(s)=L(\sum_{\alpha}\rho_{\alpha}s)=\sum_{\alpha}A_{\alpha}(\rho_{\alpha}\nabla s+\d \rho_{\alpha}\ocross s)+B_{\alpha}(\rho_{\alpha}s)=A(\nabla s)+B(s),
  \end{equation*}
  provided we define
  \begin{equation*}
    A=\sum_{\alpha}\rho_{\alpha}A_{\alpha}\text{ and }B=\sum_{\alpha}A_{\alpha}(\d\rho_{\alpha}\ocross -)+\rho_{\alpha}B_{\alpha}(-).
  \end{equation*}
  By local finiteness, $A,B$ are well-defined sections of the appropriate bundles. We have shown
\end{example}
\begin{thm}
  Let $L$ be a first order differential operator $E_{1}\to E_{2}$. Let $\nabla$ be a connection on $E_{1}$. Then there are sections $A,B$ of $\Hom(\Hom(TY,E_{1}),E_{2})$ and $\Hom(E_{1},E_{2})$, respectively, so that
  \begin{equation*}
    L(s)=A(\nabla s)+B(s).
  \end{equation*}
\end{thm}
\begin{xca}
  Prove that $A,B$ are uniquely determined by $L$. Hints: pick $s_{0}\in E_{1,p_{0}}$ and a section $s$ so $s(p_{0})=s_{0}$, and so $\nabla s$ vanishes at $p_{0}$ (this can be achieved by parallel-transporting along radial lines). Conclude that $B$ is completely determined by $L$.

  Now for any element $\theta\in \Hom(TY,E_{1})$, extend $\theta$ to a section of $\Omega^{1}(E_{1})$. Then we can construct a section $s$ (in a neighborhood of $p_{0}$) by parallel transporting along radial lines so that
  \begin{equation*}
    \bd_{r}\intprod \nabla s=\bd_{r}\intprod \theta,
  \end{equation*}
  where $\bd_{r}$ is the radial vector field. A-priori $s$ is smooth
  when pulled back to the blow up $\S^{d-1}\cross [0,\infty)$. It can
  be shown that $s$ is smooth when considered as a section on
  $\R^{d}$. The key is that its (covariant) partial derivatives exist
  and are continuous functions on $\R^{d}$, and agree with $\theta$ at
  $p_{0}$. Then, conclude that $A_{p_{0}}((\nabla s)_{p_{0}})=A_{p_{0}}(\theta)$. Conclude that $A$ is determined by $L$.
\end{xca}
\clearpage


\subsection*{Sobolev embedding/compactness theorems}
\begin{defn}
  Let $g$ be a Riemannian metric on $Y$, an oriented Riemannian manifold (potentially with boundary), let $\ip{-,-}$ be an inner product on $E$ (anti-linear in the first factor). Let $\nabla$ denote the Levi-Civita connection on $TY$, and also a connection on $E$ compatible with its metric (where there is hopefully no chance of confusion).

  For codimension $0$ submanifolds with boundary $U$, we define the space $L^{p}(U,E_{1})$ by duality: $L^{p}(U,E_{1})$ consists of distributions $C^{\infty}_{c}(\text{int}(U),E_{1})\to \C$ satisfying the estimate
  \begin{equation*}
    \varphi \in L^{p}(U,E_{1})\implies \abs{(\varphi,f)}\le C_{\varphi}\left[\int_{U}\abs{f}^{q}\,\dVol_{g}\right]^{1/q},
  \end{equation*}
  where $\abs{f}=\ip{f,f}^{1/2}$, and $q$ is the H\"older conjugate to $p$. We define the $L^{p}$ norm on $L^{p}(U,E_{1})$ by duality:
  \begin{equation*}
    \norm{\varphi}_{L^{p}}=\sup\set{\abs{(\varphi,f)}: f\in C^{\infty}_{c}(\text{int}(U),E_{1})\text{ and }\left[\int_{U}\abs{f}^{q}\,\dVol_{g}\right]^{1/q}=1}.
  \end{equation*}
  \begin{xca}
    Let $\varphi$ be a continuous section of $E_{1}$ compactly supported in $\closure{U}$. It is clear that
    \begin{equation*}
      \left[\int_{U}\abs{\varphi}^{p}\,\dVol_{g}\right]^{1/p}<\infty.
    \end{equation*}
    Then $\varphi$ induces an element of $L^{p}(U,E_{1})$ by
    \begin{equation*}
      (\varphi,f)=\int_{U}\ip{\varphi(x),f(x)}\dVol_{g}(x),
    \end{equation*}
    and
    \begin{equation*}
      \norm{\varphi}_{L^{p}(U,E_{1})}= \left[\int_{U}\abs{\varphi}^{p}\,\dVol_{g}\right]^{1/p}.
    \end{equation*}
    Hint: without loss, suppose that the integrals of $\abs{\varphi}^{p}$ and $\abs{f}^{q}$ are $1$. Estimate
    \begin{equation*}
      \abs{\ip{\varphi(x),f(x)}}\le \abs{\varphi(x)}\abs{f(x)}\le \frac{1}{p}\abs{\varphi(x)}^{p}+\frac{1}{q}\abs{f(x)}^{q},
    \end{equation*}
    to conclude that $\norm{\varphi}_{L^{p}}\le 1$. For the converse, consider smooth approximations (i.e.\ use convolution) to 
    \begin{equation*}
      \frac{\varphi}{\epsilon+\abs{\varphi}^{2-p}}
    \end{equation*}
    and show that
    \begin{equation*}
      \norm{\varphi}_{L^{p}}\ge \int\abs{\varphi}^{p}\frac{\abs{\varphi}^{2}}{\epsilon\abs{\varphi}^{p}+\abs{\varphi}^{2}}\dVol_{g}.
    \end{equation*}
    By continuity of the integral $C^{0}\to \R$ (on the support of $\varphi$) conclude that
    \begin{equation*}
      \norm{\varphi}_{p}\ge 1.
    \end{equation*}
    (where we are still assuming we normalized $\varphi$ so $\abs{\varphi}^{p}$ had integral $1$).
  \end{xca}
  \begin{thm}
    The continuous sections compactly supported in $\text{int}(U)$ are dense in $L^{p}(U,E_{1})$. 
  \end{thm}
  \begin{clear}{Remark}
    This is a fairly deep theorem.
  \end{clear}
  \begin{cor}
    Using convolution, prove that $C^{\infty}_{c}(\text{int}(U),E_{1})$ is dense in $L^{p}(U,E_{1})$. Conclude that the dual of $L^{p}(U,E_{1})$ is naturally identified with $L^{q}(U,E_{1})$. More precisely, there is a unique pairing
    \begin{equation*}
      (\varphi,f)\in L^{p}\cross L^{q}\mapsto \C
    \end{equation*}
    extending the pairing of $L^{p}$ on $C^{\infty}$ and vice-versa, so that
    \begin{equation*}
      \abs{(\varphi,f)}\le \norm{\varphi}_{L^{p}}\norm{f}_{L^{q}},
    \end{equation*}
    and that this pairing establishes the bijection $(L^{p})^{*}\simeq L^{q}$.
  \end{cor}  
\end{defn}
\begin{defn}[The adjoint of a tensor]
  Let $E,F$ be bundles with metrics denoted $\ip{-,-}$. We obtain a metric on $\Hom(E,F)$ by identifying $\Hom(E,F)\simeq \Hom(E,\C)\ocross F$. The induced norm is equivalent to the operator norm
  \begin{equation*}
    \abs{\Phi}_{\text{oper}}=\sup\set{\abs{\Phi(x)}_{F}:\abs{x}_{E}=1}.
  \end{equation*}
  Given a smooth section $\Phi\in \Hom(E,F)$, we observe that
  \begin{equation*}
    \int \ip{\Phi(e),f}\d\Vol=\int \ip{e,\Phi^{*}(f)}\d\Vol,
  \end{equation*}
  where $\Phi^{*}$ is the pointwise adjoint. Therefore we can define the evaluation of $\Phi$ on an $E$-valued distribution via duality: for distributions valued in $E$, let $\Phi(e)$ be the distribution satisfying
  \begin{equation*}
    (\Phi(e),f)=(e,\Phi^{*}(f)).
  \end{equation*}
\end{defn}
\begin{xca}
  If $f$ is in $L^{p}(E)$, and $\Phi$ is a smooth section of $\Hom(E,F)$ so that $\abs{\Phi}<C$ is uniformly bounded on $Y$, then $\Phi(f)$ is in $L^{p}(E)$, and moreover
  \begin{equation*}
    \norm{\Phi(f)}_{L^{p}}\le C\norm{f}_{L^{p}}.
  \end{equation*}
  Hint: use the fact that $L^{p}$ norm is defined by duality.
\end{xca}
\begin{xca}
  If $f$ is a continuous section of $E$, show that the weak definition of $\Phi(f)$ agrees with the definition of $\Phi(f)$ as a continuous section.
\end{xca}
\begin{defn}[connections and metrics on induced bundles]
  Let $\nabla$ be a connection on $E$ and also denote the Levi-Civita connection on $(Y,g)$. Suppose that $\nabla$ is compatible with $\ip{-,-}$. We define a metric on $\Hom(TM^{\ocross k},E)\simeq T^{*}M^{\ocross k}\ocross E$ by
  \begin{equation*}
    \ip{\theta_{1}\ocross\cdots\ocross \theta_{k}\ocross e, \theta_{1}'\ocross\cdots\ocross \theta_{k}'\ocross e'}=g(\theta_{1},\theta_{1}')\cdots g(\theta_{k},\theta_{k}')\ip{e,e'}.
  \end{equation*}
  We define a connection by
  \begin{equation*}
    \nabla(\theta_{1}\ocross \cdots \ocross \theta_{k}\ocross e)=\theta_{1}\ocross \cdots\ocross\theta_{k}\ocross \nabla e +\sum_{i} \theta_{1}\ocross \cdots \ocross \nabla \theta_{i}\ocross \cdots \theta_{k}\ocross e.
  \end{equation*}
  \begin{xca}
    Check that the induced connection on $(T^{*}M)^{\ocross k}\ocross E$ is compatible with the metric.
  \end{xca}
\end{defn}
\begin{defn}[the adjoint of the covariant derivative]
  Suppose that we have a local orthonormal frame $y_{1},\cdots,y_{n}$ of $TY$. Let $\theta_{i}=g(y_{i},-)$.
  
  For smooth compactly supported sections $s\in \Gamma(E)$, $\lambda\ocross r\in \Gamma(\Hom(TM,E))$ we compute $\nabla s=\sum \theta_{i}\ocross\nabla_{y_{i}}s$ and hence
  \begin{equation*}
    \ip{\nabla s,\lambda\ocross r}=\sum_{i}g(\theta_{i},\lambda)\ip{\nabla_{y_{i}}s,r}=\sum_{i}g(\theta_{i},\lambda)(y_{i}\intprod \d\ip{s,r})-g(\theta_{i},\lambda)\ip{s,\nabla_{y_{i}}r},
  \end{equation*}
  so
  \begin{equation*}
    (\nabla s,\lambda\ocross r)=\int X_{\lambda}\intprod \d\ip{s,r}\dVol_{g}-\ip{s,\nabla_{X_{\lambda}}r}\dVol_{g},
  \end{equation*}
  where $X_{\lambda}=\sum_{i}g(\theta_{i},\lambda)y_{i}$ ($X_{\lambda}$ is the $g$-``dual'' of $\lambda$).

  A bit of differential form gymnastics produces
  \begin{equation*}
    \begin{aligned}
      0&=X_{\lambda}\intprod (\d\ip{s,r}\wedge \dVol)=(X_{\lambda}\intprod \d \ip{s,r})\dVol-\d\ip{s,r}(X_{\lambda}\intprod \dVol)\\
      0&=(X_{\lambda}\intprod \d\ip{s,r})\dVol-\d(\ip{s,r}(X_{\lambda}\intprod \d\Vol))+\ip{s,r}\text{div}(X_{\lambda})\d\Vol,
    \end{aligned}
  \end{equation*}
  whre $\text{div}(X_{\lambda})\dVol=\d(X_{\lambda}\intprod\dVol)$. Since the integral of a compactly supported exact form is $0$, we conclude
  \begin{equation*}
    (\nabla s,\lambda\ocross r)=\int \ip{s,\text{div}(X_{\lambda})r -\nabla_{X_{\lambda}}r}\dVol_{g}.
  \end{equation*}
  Conclude that $\nabla_{X}$ has formal adjoint
  \begin{equation*}\tag{$\ast$}
    \nabla^{*}:\lambda\ocross r\mapsto \text{div}(X_{\lambda})r-\nabla_{X_{\lambda}}r.
  \end{equation*}
  One can easily check that this is a first order differential operator. While we proved this under the assumption that it has an orthonormal coordinate frame, it is clear that $X_{\lambda}$ doesn't actually depend on the particular orthonormal coordinate frame, and hence $\nabla^{*}$ defined in $(\ast)$ is the adjoint of $\nabla$ on any bundle $E$.
\end{defn}
\begin{defn}
  Let $\varphi$ be a distribution $C^{\infty}(\text{int}(U), E)\to \C$. The weak derivative $\nabla \varphi$ is the distribution valued in the bundle $\Hom(TM,E)$ defined by
  \begin{equation*}
    (\nabla \varphi,\Phi)=(\varphi,\nabla^{*}\Phi),
  \end{equation*}
  where $\Phi\in \Gamma\Hom(TM,E)$ is a test function.

  Given a smooth vector field $X$, we can consider the tensor
  \begin{equation*}
    X\intprod(-):\Gamma\Hom (TM,E)\to \Gamma E.
  \end{equation*}
  This defines a distributional evaluation $\Hom(TM,E)\to E$. In this fashion, we can define the evaluations $\nabla_{X}\varphi$ for any distribution $\varphi$.
  \begin{xca}
    Show that
    \begin{equation*}
      (\nabla_{X}\varphi,\rho)=(\varphi,\nabla_{X}^{*}\rho),
    \end{equation*}
    where
    \begin{equation*}
      \nabla_{X}^{*}\rho=\text{div}(X)\rho-\nabla_{X}\rho.
    \end{equation*}    
  \end{xca}  
\end{defn}


\begin{defn}[Sobolev Spaces]
  Let $U$ be a codimension 0 submanifold of $Y$. We define the space $W^{k,p}(U,E)$ by recursion $    W^{0,p}(U,E)=L^{p}(U,E)$, and $W^{k,p}(U,E)$ consists of all distributions $\varphi$ whose weak derivative $\nabla \varphi$ lies in $W^{k-1,p}(U,\Hom(TY,E))$.

  If the tangent bundle of $U$ has global orthonormal frame $y_{1},\cdots,y_{n}$, it is equivalent to suppose that
  \begin{equation*}
    \nabla_{y_{1}}\varphi,\cdots,\nabla_{y_{n}}\varphi\in W^{k-1,p}(U,E),
  \end{equation*}
  Indeed the decomposition
  \begin{equation*}
    \nabla\varphi=\sum_{i}\theta_{i}\ocross\nabla_{y_{i}}\varphi
  \end{equation*}
  identifies $\Hom(TY,E)$ with $E^{\oplus n}$ isometrically, and so the induced ``weak'' identification induces an isometry on $L^{p}$. We put a metric on $W^{k,p}(U,E)$ via it's inclusion into
  \begin{equation*}\tag{$\ast$}
    W^{k,p}(E)\to L^{p}(E)\oplus L^{p}(\Hom(TM,E))\oplus \cdots \oplus L^{p}(\Hom(TM^{\ocross k},E)).
  \end{equation*}
  \begin{xca}
    Prove that the inclusion in ($\ast$) is a closed inclusion. Hint: arguing by induction, it suffices to show that if
    \begin{equation*}
      u_{n}\to u_{\infty}\text{ in $W^{k-1,p}$}\text{ and }\nabla u_{n}\to w_{\infty}\text{ in $W^{k-1,p}$},
    \end{equation*}
    then $\nabla u_{\infty}=w_{\infty}$. It will follow that $u_{n}\to u_{\infty}$ in $W^{k,p}$. The following estimate
    \begin{equation*}
      (\nabla u_{\infty}-w_{\infty},\varphi)=(u_{\infty}-u_{n},\nabla^{*}\varphi)+(\nabla u_{n}-w_{\infty},\varphi)
    \end{equation*}
    and taking the limit $n\to\infty$ proves that $\nabla u_{\infty}=w_{\infty}$. 
  \end{xca}
  \begin{xca}
    For $p\in (1,\infty)$, we have shown that $L^{p}$ is reflexive. It follows that $L^{p}\oplus \cdots \oplus L^{p}$ is reflexive. Conclude that $W^{k,p}$ is reflexive. Hint: this is an exercise in applying the Hahn-Banach theorem. 
  \end{xca}
  \begin{defn}
    It is clear that smooth compactly supported functions induce elements of $W^{k,p}(U,E)$. We define
    \begin{equation*}
      W^{k,p}_{0}(U,E)=\text{closure of smooth compactly supported functions in $W^{k,p}(U,E)$}.
    \end{equation*}
    Warning: $W^{k,p}_{0}$ is generally not equal to $W^{k,p}$. It turns out that $W^{k,p}_{0}$ is much better behaved (in certain regards) compared to $W^{k,p}$.
  \end{defn}
\end{defn}
\begin{thm}
  Let $E$ be a vector bundle on $Y$, and let $g,\ip{-,-},\nabla$ and $g',\ip{-,-}',\nabla'$ be two sets of data. Suppose that we can uniformly bound the $C^{\infty}$ distance between the two sets of data, i.e.\
  \begin{equation*}
 \norm{g-g'}_{C^{\infty},g}+\norm{\ip{-,-}-\ip{-,-}'}_{C^{\infty},\ip{-,-}}+\norm{\nabla-\nabla'}_{C^{\infty},g,\ip{-,-}}<\infty,
  \end{equation*}
  (recall that the difference of two connections is a tensor, and hence it makes sense to talk about its $C^{\infty}$ norm). Then we have an equality of sets
  \begin{equation*}
    W^{k,p}(Y,g,\ip{-,-},\nabla)=W^{k,p}(Y,g',\ip{-,-},\nabla')\text{ and similarly for $W^{k,p}_{0}$}.  \end{equation*}
  and moreover the two norms on $W^{k,p}$ are comparable.\hfill$\square$
\end{thm}
\begin{thm}
  Let $U\subset Y$ be a codimension $0$ manifold with boundary. Then there is an extension (isometric) embedding
  \begin{equation*}
    \text{ext}:W^{k,p}_{0}(U,E)\to W^{k,p}_{0}(Y,E),
  \end{equation*}
  defined in the obvious way on smooth functions and extended by density. There is also a restriction
  \begin{equation*}
    \text{res}:W^{k,p}(Y,E)\to W^{k,p}(U,E),
  \end{equation*}
  and $\text{res}\circ \text{ext}=\id$.
\end{thm}
\begin{cor}
  Let $Y$ be compact with boundary. Let $f:Y\to Z$ be a smooth
  embedding of a codimension $0$ submanifold with boundary, and let
  $F:E_{1}\to f^{*}E_{2}$ be a bundle isomorphism. Then there is a
  bounded extension
  \begin{equation*}
    W^{k,p}_{0}(Y,E_{1},g_{Y},\ip{-,-}_{Y},\nabla_{Y})\to W^{k,p}_{0}(Y,E_{2},g_{Z},\ip{-,-}_{Z},\nabla_{Z}),
  \end{equation*}
  and a bounded restriction
  \begin{equation*}
    W^{k,p}(Y,E_{2},g_{Z},\ip{-,-}_{Z},\nabla_{Z})\to W^{k,p}(Y,E_{1},g_{Y},\ip{-,-}_{Y},\nabla_{Y}),
  \end{equation*}
  so that the restriction composed with the extension is the identity. 
\end{cor}
\begin{clear}{Remark}
  These results allow us freedom to change the data defining the Sobolev norms, without actually changing the topology of the space $W^{k}_{0}(Y,E)$. We remark that some form of compactness on $Y$ is crucial. 
\end{clear}
\begin{lemma}
  Let $\rho$ be a smooth section of $\Hom(E,F)$. If $u$ is in $W^{k,p}(Y,E)$ (resp.\ $W^{k,p}_{0}$), then 
  \begin{equation*}
    \rho(u)\in W^{k,p}(Y,F)\text{ resp.\ $W^{k,p}_{0}$}
  \end{equation*}
  and
  \begin{equation*}
    \norm{\rho(u)}_{W^{k,p}(Y,F)}\le C(\rho)\norm{u}_{W^{k,p}(Y,E)},
  \end{equation*}
  where $C(\rho)$ depends on the sizes of the first $k$ derivatives of $\rho$.
\end{lemma}
\begin{proof}
  Argue by induction.
  \begin{equation*}
    \nabla \rho(u)=\nabla \text{ev}(\rho\ocross u)=\text{ev}(\nabla \rho\ocross u)+\text{ev}(\rho\ocross \nabla u),
  \end{equation*}
  to conclude that
  \begin{equation*}
    \norm{\nabla(\rho(u))}_{W^{k-1,p}}\le C(\nabla \rho)\norm{u}_{W^{k-1,p}}+C(\rho)\norm{\nabla u}_{W^{k-1,p}}.
  \end{equation*}
\end{proof}

\begin{clear}{Sobolev Embedding}
  Let $Y^{d}$ be compact with boundary. Let $E$ be a vector bundle on $Y$. Suppose that $k>\ell$ and
  \begin{equation*}
    k-\frac{p}{d}\ge \ell-\frac{r}{d}.
  \end{equation*}
  Then the inclusion of test functions into $W^{\ell,r}_{0}$ extends to a bounded inclusion
  \begin{equation*}
    W^{k,p}_{0}(Y,E)\subset W^{\ell,r}_{0}(Y,E).
  \end{equation*}
\end{clear}
\begin{proof}
  First we claim that is suffices to prove it locally. To see this, cover $Y$ by finitely many open sets $U_{\alpha}$ where it is known that inclusion extends to a bounded inclusion $$W^{k,p}_{0}(U_{\alpha},E)\to W^{\ell,r}_{0}(U_{\alpha},E).$$

  Fix a partition of unity $\rho_{\alpha}$. Then for any $u$ smooth and compactly supported in $Y$, we have
  \begin{equation*}
    \norm{u}_{W^{k,p}}\le \sum_{\alpha} C_{\alpha}\norm{\rho_{\alpha}u}_{W^{\ell,r}}\le \sum_{\alpha}C_{\alpha}'\norm{u}_{W^{\ell},r}=C(\set{U_{\alpha},\rho_{\alpha}})\norm{u}_{W^{\ell,r}},
  \end{equation*}
  as desired.

  To prove it locally, note that it does not depend on the data used to define the norms. Thus we may suppose that $U=B(r)\subset \R^{d}$, $E=\C^{n}$ is a trivial bundle, and the metrics/connections are all the standard ones. The following tricky lemma is the key result
  \begin{lemma}[Gagliardo-Nirenberg Inequality]
    Let $u\in W^{1,1}_{0}(\R^{d},\C^{n})$. Then $u\in L^{d/(d-1)}$ and
    \begin{equation*}
      \norm{u}_{L^{d/(d-1)}}\le \left[\prod_{1}^{d}\norm{\bd_{i}u}_{L^{1}}\right]^{1/d}.
    \end{equation*}
  \end{lemma}
  Assuming this technical lemma, we can complete the proof. First we prove the following

  \begin{claim}
    Let $p<d$ and define $p^{*}>p$ by
    \begin{equation*}
      1-\frac{d}{p}=\frac{d}{p^{*}}.
    \end{equation*}
    Then there is constant $C_{p}$ so that for all $u\in W^{1,p}_{0}$, $u\in L^{p^{*}}$ and
    \begin{equation*}
      \norm{u}_{L^{p^{*}}}\le C_{p}\norm{\nabla u}_{L^{p}}.
    \end{equation*}
    \begin{proof}     
  Suppose that $u$ is a test function, and consider $v_{\epsilon}=(\epsilon+\abs{u}^{2})^{(s-1)/2}u$, where $s>1$ will be specified later. Then
  \begin{equation*}
    \bd_{i}v_{\epsilon}=\frac{(s-1)}{2} (\epsilon+\abs{u}^{2})^{(s-3)/2}\sum_{j}u_{j}\bd_{i}u_{j} u+(\epsilon+\abs{u}^{2})^{(s-1)/2}\bd_{i}u,
  \end{equation*}
  hence
  \begin{equation*}
    \abs{\bd_{i}v_{\epsilon}}\le (\frac{(s-1)}{2}(\epsilon+\abs{u}^{2})^{(s-3)/2}\abs{u}^{2}+(\epsilon+\abs{u}^{2})^{(s-1)/2})\abs{\nabla u}.
  \end{equation*}
  Applying the lemma yields
  \begin{equation*}
    \left[\int (\epsilon+\abs{u}^{2})^{(s-1)d/2(d-1)}\abs{u}^{d/(d-1)}\d\Vol\right]^{(d-1)/d}\le \left[\prod_{1}^{d}\int\abs{\bd_{i}v_{\epsilon}}\d\Vol\right]^{1/d}.
  \end{equation*}
  Applying monotone convergence theorem proves
  \begin{equation*}    \left[\int\abs{u}^{sd/(d-1)}\d\Vol\right]^{(d-1)/d}\le C_{s}\int \abs{u}^{s-1}\abs{\nabla u}\d\Vol.
  \end{equation*}
  Now suppose that $p<d$ and $p^{*}$ is defined by
  \begin{equation*}
    1-\frac{d}{p}=-\frac{d}{p^{*}}\implies p^{*}p-dp^{*}=-dp\implies p^{*}=\frac{dp}{d-p}.
  \end{equation*}
  Pick $s$ so that
  \begin{equation*}
    \frac{sd}{d-1}=p^{*}.
  \end{equation*}
  Then we conclude that
  \begin{equation*}
    \norm{u}_{L^{p^{*}}}^{s}\le C_{s}\int \abs{u}^{s-1}\norm{\nabla u}\dVol.
  \end{equation*}
  Let $q$ be Hold\"er conjugate to $p$. Then
  \begin{equation*}
    \norm{u}_{L^{p^{*}}}^{s}\le C_{s}\left[\int \abs{u}^{(s-1)q}\dVol\right]^{1/q}\norm{\nabla u}_{L^{p}}.
  \end{equation*}
  Now since
  \begin{equation*}
    \frac{s}{d-1}=p^{*}\implies (s-1)q=(d-1)p^{*}q-q\implies (s-1)q=\frac{d-1}{d-p}pq-q=\frac{d}{d-p}(pq-q)=p^{*}.
  \end{equation*}
  Therefore
  \begin{equation*}
    \norm{u}_{L^{p^{*}}}^{s}\le C_{s}\norm{u}_{L^{p^{*}}}^{p^{*}/q}\norm{\nabla u}_{L^{p}}\implies \norm{u}_{L^{p^{*}}}^{s-p^{*}/q}\le C_{s}\norm{\nabla u}_{L^{p}}.
  \end{equation*}
  However, the reader can easily check that $s-p^{*}/q=1$. Therefore
  \begin{equation*}
    \norm{u}_{L^{p^{*}}}\le C_{p}\norm{\nabla u}_{L^{p}},
  \end{equation*}
  as desired. Since this holds for test functions, it extends to $W^{1,p}_{0}$ by density. \end{proof}
\end{claim}

Now we can complete the proof. Suppose that
\begin{equation*}
  1-\frac{d}{p}\ge -\frac{d}{r}.
\end{equation*}
Define $r^{*}$ by
\begin{equation*}
  1-\frac{d}{r^{*}}=-\frac{d}{r}.
\end{equation*}
Then $r^{*}\le p$ and $r^{*}<d$. Therefore we can apply our claim to conclude
\begin{equation*}
  \norm{u}_{L^{r}}\le C_{r}\norm{\nabla u}_{L^{r^{*}}}.
\end{equation*}
Since we are on a bounded domain,
\begin{equation*}
  \norm{\nabla u}_{L^{r^{*}}}\le C\norm{\nabla u}_{L^{p}},
\end{equation*}
as desired. The constant $C$ depends on $r,p$ and the domain. We conclude a continuous embedding:
\begin{equation*}
  W^{1,p}(B)\subset L^{r}(B).
\end{equation*}
Now suppose that $k-d/p\ge \ell-d/r$. Pick $r_{0},\cdots,r_{k-\ell}=r$ so that
\begin{equation*}
  k-d/r_{0}=(k-1)-d/r_{1}=\cdots=\ell-\frac{d}{r}.
\end{equation*}
Then by using the result for $k=1$ conclude
\begin{equation*}
  W^{k,r_{0}}_{0}(B)\subset W^{k-1,r_{1}}_{0}\cdots \subset W^{\ell,r}_{0}(B).
\end{equation*}
Since $p\ge r_{0}$, conclude the bounded inclusion
\begin{equation*}
  W^{k,p}_{0}(B)\subset W^{\ell,r_{0}}_{0}(B).
\end{equation*}
This completes the proof of the Sobolev embedding theorem. 
\end{proof}
\begin{xca}
  Using the Gagliardo-Nirenberg inequality for $u\in W^{1,1}_{0}(B)$
  \begin{equation*}
    C^{-1}\norm{u}_{L^{1}}\le \norm{u}_{L^{d/(d-1)}}\le \norm{\nabla u}_{L^{1}},
  \end{equation*}
  prove that $1\not\in W^{1,1}_{0}(B)$.
\end{xca}
\begin{thm}[Rellich-Kondrachov compactness]
  Suppose that $k>\ell$ and $k-d/p>\ell-d/r$, and $Y^{d}$ is a compact manifold with boundary, then the Sobolev embedding
  \begin{equation*}
    W^{k,p}_{0}(Y,E)\subset W^{\ell,r}_{0}(Y,E)
  \end{equation*}
  is a compact inclusion.  
\end{thm}
\begin{proof}
  As in the proof of the Sobolev embedding theorem, we will reduce the theorem to a local computation in $\R^{d}$.

  Suppose that the theorem holds locally, i.e.\ we can cover $Y$ by finitely many open sets $U_{\alpha}$ where it is know that the inclusion
  \begin{equation*}
    W^{k,p}_{0}(U_{\alpha},E)\subset W^{\ell,r}_{0}(U_{\alpha},E)
  \end{equation*}
  is compact. Let $u_{n}$ be a bounded sequence in $W^{k,p}_{0}(Y,E)$. If $\set{\rho_{\alpha}}$ is a partition of unity, then $\rho_{\alpha}u_{n}$ is a bounded sequence in $W^{k,p}_{0}(U_{\alpha},E)$, and hence, after a subsequence, converges to something $u_{\infty,\alpha}\in W^{\ell,r}_{0}(U_{\alpha},E)$. Then
  \begin{equation*}
    \norm{u_{n}-\sum_{\alpha}u_{\infty,\alpha}}_{W^{\ell,r}(Y)}\le \sum_{\alpha}\norm{\rho_{\alpha}u_{n}-u_{\infty,\alpha}}_{W^{\ell,r}(Y)}=\sum_{\alpha}\norm{\rho_{\alpha}u_{n}-u_{\infty,\alpha}}_{W^{\ell,r}(U_{\alpha})}\to 0.
  \end{equation*}
  Therefore it suffices to prove the theorem locally. We may therefore suppose that $E$ is trivial and $Y=B(1)$ is a ball in $\R^{n}$ with the standard metric, connection, etc.

  We can also reduce to the case $k=1$, because if we know that $W^{1,p}_{0}\subset W^{0,r}_{0}$ is compact when $1-d/p>-d/r$, then we also know $W^{k,p}_{0}\subset W^{k-1,r}_{0}$ when $k-d/p>(k-1)-d/r$, and by finding $r_{1},\cdots,r_{n}=r$ so that
  \begin{equation*}
    k-d/p>(k-1)-d/r_{1}>\cdots>\ell-d/r,
  \end{equation*}
  we conclude the result for $W^{k,p}_{0}\subset W^{\ell,r}_{0}$.

  The strategy of proof is the following: if $\mathscr{E}$ is a uniformly bounded family of functions in $W^{1,p}_{0}(B(1))$, and $\rho_{\delta}$ is a mollifying function, we will show that there is a universal bound
  \begin{equation*}
    \norm{u-\rho_{\delta}\ast u}_{L^{r}}<c(\delta),\text{ $c(\delta)\to 0$ as $\delta\to 0$, for all $u\in \mathscr{E}$}.
  \end{equation*}
  We will then show that the family of smooth functions $\rho_{\delta}\ast \mathscr{E}$ is bounded and equicontinuous (and are all supported in $B(1+\delta)$). By Arz\'ela-Ascoli, we conclude that $\rho_{\delta}\ast\mathscr{E}$ is pre-compact in $C^{0}(B(1+\delta))$ topology, and hence also in $L^{q}(B(1))$.

  Now pick $\epsilon>0$. Choose $\delta$ so $c(\delta)<\epsilon/3$. Since $\rho_{\delta}\ast \mathscr{E}$ is compact, we can find elements $v_{1},\cdots,v_{n}\in \mathscr{E}$ so that the $\epsilon/2$ balls around $\rho_{\delta}\ast v_{i}$ cover $\rho_{\delta}\ast\mathscr{E}$. Since
  \begin{equation*}
    \norm{\mathscr{E}-\rho_{\delta}\ast\mathscr{E}}
    _{L^{q}}<\epsilon/3,
  \end{equation*}
  we conclude that the $\epsilon$ balls around $v_{i}$ cover $\mathscr{E}$, and since $\epsilon$ was arbitrary, we conclude $\mathscr{E}$ is pre-compact. Any sequence in a pre-compact space inside of a Banach space converges after taking a subsequence, and so we conclude the compactness statement.

  It suffices to prove the two claims:
  \begin{claim}\label{claim:20}
    If $1-d/p>-d/q$, and $\mathscr{E}$ is bounded in $W^{1,p}_{0}(B(1))$, then there is a uniform bound
    \begin{equation*}      \norm{v-\rho_{\delta}\ast v}_{L^{q}}<c(\delta)\text{ for all $v\in \mathscr{E}$},
    \end{equation*}
    and $c(\delta)\to 0$ as $\delta\to 0$.
  \end{claim}
  \begin{claim}\label{claim:21}
    The family $\rho_{\delta}\ast \mathscr{E}$ is equicontinuous. 
  \end{claim}
\end{proof}
In order to prove Claim \ref{claim:20}, we introduce some results about convolution. We define the \textbf{translation} operator on smooth functions by
\begin{equation*}
  \tau_{h}(f)(x)=f(x+h).
\end{equation*}
It is clear that $\tau_{h}^{*}=\tau_{-h}$. It is readily checked that
\begin{equation*}\tag{$\ast$}
  \norm{\tau_{h}f}_{W^{k,p}(B)}\le \norm{f}_{W^{k,p}},
\end{equation*}
and hence $\tau_{h}$ extends by density to $W^{k,p}_{0}(B)$. Moreover, for all $u\in W^{k,p}_{0}(B)$ we have $\tau_{h}u\to u$ in $W^{k,p}$ norm, as $h\to 0$.

We define the \textbf{convolution} of $u$ with a bump function $\rho_{\delta}=\delta^{-d}\rho(\delta^{-1} x)$ by
\begin{equation*}
  u\ast\rho_{\delta}(x)=\int_{\R^{d}} \rho_{\delta}(y)\tau_{-y}u(x)\,\dVol(y).
\end{equation*}
This is well-defined for smooth compactly functions. By pairing with a test-function $v$, we conclude that
\begin{equation*}
  \begin{aligned}
    (u\ast\rho_{\delta},v)&=\int_{\R^{d}} \int_{\R^{d}} \rho_{\delta}(y)u(x-y)v(x)\d\Vol(y)\d\Vol(x)\\
    &=\int_{\R^{d}}\rho_{\delta}(y)\int_{\R^{d}}u(x-y)v(x)\d\Vol(x)\d\Vol(y)\\
    &\le \norm{u}_{L^{p}}\norm{v}_{L^{q}},
  \end{aligned}
\end{equation*}
hence $\norm{u\ast\rho_{\delta}}_{L^{p}}\le \norm{u}_{L^{p}}$. Therefore convolution extends to a bounded map $L^{p}\to L^{p}$.

It is clear that $\nabla(u\ast\rho_{\delta})=(\nabla u)\ast\rho_{\delta}$ (for smooth $u$) and hence 
\begin{equation*}
  \norm{u\ast\rho_{\delta}}_{W^{k,p}}\le \norm{u}_{W^{k,p}},
\end{equation*}
and this estimate shows that $\rho_{\delta}\ast(-)$ extends to a bounded map $W^{k,p}_{0}\to W^{k,p}_{0}$.

We can compute
\begin{equation*}
  \rho_{\delta}\ast u(x)=\int \rho_{\delta}(y)u(x-y)\d\Vol(y)=\int \rho_{\delta}(x-y)u(y)\dVol(y),
\end{equation*}
and so
\begin{equation*}
  \nabla (\rho_{\delta}\ast u)=(\nabla \rho_{\delta})\ast u.
\end{equation*}
Consequently,
\begin{equation*}
  \norm{\nabla (\rho_{\delta}\ast u)}_{C^{0}}\le \norm{\nabla \rho_{\delta}}_{C^{0}}\norm{u}_{L^{1}}\le \norm{\nabla \rho_{\delta}}_{C^{0}}\norm{u}_{L^{p}}.
\end{equation*}
It follows that $\rho_{\delta}\ast u$ is differentiable if $u\in L^{p}$ (by density, and completeness of $C^{1}$ topology) and that
\begin{equation*}
  \nabla(\rho_{\delta}\ast u)=(\nabla \rho_{\delta})\ast u. 
\end{equation*}
Consequently $\rho_{\delta}\ast u$ is $C^{\infty}$ if $u\in L^{p}$.

\begin{lemma}
  Suppose that $1-d/p>-d/r$. Define $p^{*}$ by
  \begin{equation*}
    1-d/p\ge -d/p^{*}>-d/r. 
  \end{equation*}
  This can always be achieved. Since $1\ge 1/r>1/p^{*}$, there is $\alpha>0$ so that
  \begin{equation*}
    \frac{1}{r}=\alpha+\frac{1-\alpha}{p^{*}}.
  \end{equation*}
  We claim that there is a constant $C$ so that for all $u\in W^{1,p}_{0}(B)$,
  \begin{equation*}
    \norm{\tau_{h}u-u}_{L^{r}(\R^{d})}\le C\abs{h}^{\alpha}\norm{u}_{W^{1,p}(B)}.
  \end{equation*}
\end{lemma}
\begin{proof}
  It suffices to prove this when $u\in C^{\infty}_{0}(B)$. Then we compute
  \begin{equation*}
    \abs{\tau_{h}u(x)-u(x)}\le \abs{h}\int_{0}^{1}\abs{\nabla u(x+th)}\d t,
  \end{equation*}
  hence
  \begin{equation*}
    \norm{\tau_{h}u-u}_{L^{1}}\le \abs{h}\int_{0}^{1}\int_{\R^{d}}\abs{\nabla u(x+th)}\dVol\,\d t\le \abs{h}\norm{\nabla u}_{W^{1,p}}\text{Vol}(B)^{1/q},
  \end{equation*}
  where we use the fact that $\nabla u(x+th)$ is supported in $B(1)-th$.
  
  Now we compute
  \begin{equation*}
    \abs{\tau_{h}u-u}^{r}=\abs{\tau_{h}u-u}^{\alpha r}\abs{\tau_{h}u-u}^{(1-\alpha)r},
  \end{equation*}
  and then apply H\"older's inequality, with $1/\beta$, $1/(1-\beta)$, to obtain
  \begin{equation*}
   \int \abs{\tau_{h}u-u}^{r}\le \left[\int\abs{\tau_{h}u-u}^{\frac{\alpha r}{\beta}}\dVol\right]^{\beta}\left[\int\abs{\tau_{h}u-u}^{\frac{(1-\alpha) r}{1-\beta}}\dVol\right]^{1-\beta}.
  \end{equation*}
  Pick $\beta=\alpha r$, so that $1-\beta=\frac{(1-\alpha)r}{p^{*}}$, to obtain
  \begin{equation*}
    \norm{\tau_{h}u-u}_{L^{r}}\le \norm{\tau_{h}u-u}_{L^{1}}^{\alpha }\norm{\tau_{h}u-u}_{L^{p^{*}}}^{1-\alpha}.
  \end{equation*}
  Using Sobolev embedding for $1-d/p\ge -d/p^{*}$, we conclude
  \begin{equation*}
    \norm{\tau_{h}u-u}_{L^{p^{*}}}^{1-\alpha}\le C\norm{u}_{W^{1,p}}^{1-\alpha}.
  \end{equation*}
  We also have our previous estimate
  \begin{equation*}
    \norm{\tau_{h}u-u}_{L^{1}}^{\alpha}\le C\abs{h}^{\alpha}\norm{u}_{W^{1,p}}^{\alpha}.
  \end{equation*}
  Therefore
  \begin{equation*}
    \norm{\tau_{h}u-u}_{L^{r}}\le C\abs{h}^{\alpha}\norm{u}_{W^{1,p}},
  \end{equation*}
  as desired.  
\end{proof}
\begin{lemma}[Claim \ref{claim:20}]
  For $u\in W^{1,p}_{0}(B)$, we have
  \begin{equation*}
    \norm{u-\rho_{\delta}\ast u}_{L^{q}}\le C\abs{\delta}^{\alpha}\norm{u}_{W^{1,p}},    
  \end{equation*}
  for some $C$ and $\alpha>0$, dependent only $p,r,B,d$.
\end{lemma}

\begin{proof}
  We compute
  \begin{equation*}
    u-\rho_{\delta}\ast u=\int_{\R^{d}}\rho_{\delta}(y) (u-\tau_{-y}u)\dVol(y).
  \end{equation*}
  Pair this with a test-function $\varphi$ to conclude
  \begin{equation*}
    (u-\rho_{\delta}\ast u,\varphi)=\int_{\R^{d}}\rho_{\delta}(y)\int_{\R^{d}}\varphi(x)(u(x)-u(x-y))\d\Vol(x)\d\Vol(y).
  \end{equation*}
  This produces the estimate
  \begin{equation*}
    \abs{(u-\rho_{\delta}\ast u,\varphi)}\le \int_{\R^{d}}\rho_{\delta}(y)\norm{u-\tau_{-y}u}_{L^{r}}\dVol(y)\norm{\varphi}_{L^{s}},
  \end{equation*}
  where $s$ is conjugate to $r$. Since $\rho_{\delta}(y)$ vanishes for $\abs{y}>\delta$, we conclude
  \begin{equation*}
    \int_{\R^{d}}\rho_{\delta}(y)\norm{u-\tau_{-y}u}_{L^{r}}\dVol(y)\le C\abs{\delta}^{\alpha}\norm{u}_{W^{1,p}},    
  \end{equation*}
  hence
  \begin{equation*}
    \norm{u-\rho_{\delta}\ast u}_{L^{q}}\le C\abs{\delta}^{\alpha}\norm{u}_{W^{1,p}}.
  \end{equation*}
  As a corollary, if $\mathscr{E}$ is a bounded set in $W^{1,p}_{0}$, then we conclude
  \begin{equation*}
    \text{sup}_{\mathscr{E}}\norm{u-\rho_{\delta}\ast u}\le c(\delta)\ \ \text{where $c(\delta)\to 0$ as $\delta\to 0$.}
  \end{equation*}
\end{proof}
\begin{lemma}[Claim \ref{claim:21}]
  Let $u\in W^{1,p}_{0}(B(1))$, then the smooth section $u\ast \rho_{\delta}$ is supported in $B(1+\delta)$, and 
  \begin{equation*}\tag{$1$}
    \norm{u\ast \rho_{\delta}}_{C^{0}}\le C(\delta)\norm{u}_{W^{1,p}(B)},
  \end{equation*}
  and 
  \begin{equation*}\tag{$2$}
    \norm{\tau_{h}(u\ast \rho_{\delta})-(u\ast \rho_{\delta})}_{C^{0}}\le C(\delta) \abs{h}^{\alpha}\norm{u}_{W^{1,p}(B)},
  \end{equation*}
  for some $\alpha>0$. As a corollary, if $\mathscr{E}$ is a bounded set in $W^{1,p}_{0}(B(1))$, then $\rho_{\delta}\ast \mathscr{E}$ is a bounded equicontinuous family of functions supported on $B(1+\delta)$.
\end{lemma}
\begin{proof}
  It suffices to prove the two estimates for test $u$. We compute
  \begin{equation*}
    \abs{u\ast \rho_{\delta}(x)}\le \int \abs{u(x-y)}\abs{\rho_{\delta}(y)}\,\d\Vol(y)\le \norm{u}_{L^{p}}\norm{\rho_{\delta}}_{L^{q}}=C(\delta)\norm{u}_{W^{1,p}}.
  \end{equation*}
  For (2), we compute
  \begin{equation*}
    \tau_{h}(u\ast \rho_{\delta})(x)-u\ast\rho_{\delta}(x)=\int (u(x+h-y)-u(x-y))\rho_{\delta}(y)\d\Vol(y)\le \norm{\tau_{h}u-u}_{L^{r}}\norm{\rho_{\delta}}_{L^{s}}.
  \end{equation*}
  Using our previous deduction,
  \begin{equation*}
    \norm{\tau_{h}u-u}_{L^{r}}\le C\abs{h}^{\alpha}\norm{u}_{W^{1,p}},
  \end{equation*}
  for some $\alpha>0$. Hence
  \begin{equation*}
    \norm{\tau_{h}(u\ast \rho_{\delta})-u\ast\rho_{\delta}}_{C^{0}}\le C(\delta) \abs{h}^{\alpha}\norm{u}_{W^{1,p}}.
  \end{equation*}
  This completes the proofs of Claims \ref{claim:20} and \ref{claim:21}, and hence completes the proof of the Rellich compactness theorem.
\end{proof}

\begin{clear}{Morrey embedding}
  The goal of this section is to prove that $W^{k,p}_{0}$ functions automatically have a certain degree of H\"older continuity.

  % \begin{defn}
  %   A distribution $u$ is said to be in $W^{k,p}_{\text{loc}}(U)$ if $u$ is in $W^{k,p}(V)$ for each $V$ so that $\closure{V}\subset \text{int}(U)$. Convergence in $W^{k,p}_{\text{loc}}$ is convergence on $W^{k,p}(V)$ for each $\closure{V}\subset \text{int}(U)$.
  % \end{defn}
  \begin{defn}
    The H\"older space $C^{0,\alpha}(\R^{d},\C^{n})$ is defined by
    \begin{equation*}
      u\in C^{0,\alpha}\iff \norm{\tau_{h}u-\tau u}_{C^{0}}\le C\abs{h}^{\alpha}\text{ for $\abs{h}<1$}.
    \end{equation*}
    Defining
    \begin{equation*}
      [u]_{\alpha}=\sup_{0<\abs{h}<1}\frac{\norm{\tau_{h}u-\tau u}_{C^{0}(U)}}{\abs{h}^{\alpha}},
    \end{equation*}
    then $\norm{u}_{C^{0,\alpha}}=\norm{u}_{C^{0}}+[u]_{\alpha}$ is a complete metric on $C^{0,\alpha}$.

    We define $C^{k,\alpha}$ by recursion: $C^{k,\alpha}$ consists of all $C^{k}$ functions whose first derivative is $C^{k-1,\alpha}$.
  \end{defn}
\end{clear}
\begin{thm}[Morrey Embedding]
  If $k>\ell$, $\alpha\in (0,1)$, and $k-d/p\ge \ell+\alpha$, then the inclusion of test functions into $C^{\ell,\alpha}$ extends to a continuous extension $W^{k,p}_{0}(B)\subset C^{\ell,\alpha}$. Consequently, $W^{k,p}_{0}(B)$ functions are represented by $\ell$ times differentiable functions.  
\end{thm}
\begin{proof}
  First we claim that it suffices to prove the case $k=1$, $\ell=0$. To see why, observe that if we know it for $1-d/p\ge \alpha$, then if $u\in W^{k,p}$, all the first $k-1$ derivatives of $u$ are in $W^{1,p}$ and hence in $C^{0,\alpha}$. It follows that $u$ is $C^{k-1,\alpha}$. Therefore
  \begin{equation*}
    W^{k,p}\subset C^{k-1,\alpha}.
  \end{equation*}
  Now if $p<d$, then we can find $p<p^{*}<\infty$ so that
  \begin{equation*}
    k-d/p=(k-1)-d/p^{*}.
  \end{equation*}
  By continuing, we conclude that we can find $\ell <j\le k$ so that
  \begin{equation*}
    k-d/p=j-d/p^{*}\ge \ell+\alpha,
  \end{equation*}
  where $p^{*}\ge d$. To see why $\ell<j$, observe that if not, then we would conclude $\alpha<0$. Since $p^{*}\ge d$, we can find $\beta\in [0,1)$ so that
  \begin{equation*}
    j-d/p^{*}=(j-1)+\beta\ge \ell+\alpha.
  \end{equation*}
  To see why $\beta\ne 1$, use $p^{*}<\infty$. If $\beta\ne 0$, then by applying the composite $W^{k,p}\subset W^{j,p^{*}}\subset C^{j-1,\beta}\subset C^{\ell,\alpha}$, we conclude the desired result.

  If $\beta=0$, then we need to a bit more careful. We essentially are in the case when $p=d$ and
  \begin{equation*}
   k-d/p\ge \ell+\alpha, 
 \end{equation*}
 so $k-2\ge \ell$, since $\alpha>0$. Since $\alpha<1$, the inequality is strict:
 \begin{equation*}
   k-d/p>\ell+\alpha.
 \end{equation*}
 Then we can find $d<p^{*}<\infty$ so
 \begin{equation*}
   k-d/p>k-1-d/p^{*}=k-2+\beta>\ell+\alpha,
 \end{equation*}
 where $\beta>0$, and then use the inclusion $W^{k,p}\subset W^{k-1,p^{*}}\subset W^{k-2,\beta}\subset W^{\ell,\alpha}$.
 
  Now we prove the case when $k=1$, and $1-d/p\ge \alpha$.
  \begin{prop}
    Let $u$ be a smooth function (not necessarily a test function). There is a constant $C$ depending only on $d$ (and not the radius $r$, etc) so that
    \begin{equation*}
      \norm{u(z)-\bar{u}(x)}_{C^{0}(B(x;r))}\le C(d)\int_{B(x;r)}\frac{\abs{\nabla u(y)}}{\abs{z-y}^{d-1}}\dVol(y)
    \end{equation*}
    where $\bar{u}(x,r)$ denotes the average of $u$ over $B(x;r)$.
  \end{prop}
  \begin{proof}
    This is a fairly straightforward computation using the fundamental theorem of calculus. We compute
    \begin{equation*}
      u(z_{2})-u(z_{1})=\int_{0}^{\abs{z_{2}-z_{1}}}\nabla u(z_{1}+t\frac{z_{2}-z_{1}}{\abs{z_{2}-z_{1}}})\frac{z_{2}-z_{1}}{\abs{z_{2}-z_{1}}}\d t.
    \end{equation*}
    Integrating over $z_{2}$, and dividing by the volume of the ball, yields
    \begin{equation*}
      \bar{u}(x)-u(z_{1})=\frac{1}{\text{Vol}}\int_{B(x;r)}\int_{0}^{\abs{z_{2}-z_{1}}}\nabla u(z_{1}+t\frac{z_{2}-z_{1}}{\abs{z_{2}-z_{1}}})\frac{z_{2}-z_{1}}{\abs{z_{2}-z_{1}}}\d \Vol(z_{2})\d t.
    \end{equation*}
    Introduce polar coordinates $z_{2}=z_{1}+\rho \theta$, where $\rho$ ranges from $0$ to $\ell(\theta)<2r$. 
    \begin{equation*}
      \bar{u}(x)-u(z_{1})=\frac{1}{\text{Vol}}\int_{\S^{d-1}}\int_{0}^{\ell(\theta)}\int_{0}^{\rho}\nabla u(z_{1}+t\theta)\theta \rho^{d-1}\,\d t\,d\rho\,\d\theta.
    \end{equation*}
    The region $\set{(\rho,t):t<\rho, \rho<\ell(\theta)}$ can be reparametrized to give
    \begin{equation*}
      \bar{u}(x)-u(z_{1})=\frac{1}{\text{Vol}}\int_{\S^{d-1}}\int_{0}^{\ell(\theta)}\frac{u(z_{1}+t\theta)}{t^{d-1}}t^{d-1}\int_{t}^{\ell(\theta)}\rho^{d-1}\d\rho\,\d t\,d\theta.
    \end{equation*}
    Changing back to regular coordinates $y=z_{1}+t\theta$, we conclude,
    \begin{equation*}
      \abs{\bar{u}(x)-u(z_{1})}\le \frac{D^{d}}{\Vol\,d}\int\frac{\abs{\nabla u(y)}}{\abs{y-z_{1}}^{d-1}}\d\Vol(y),
    \end{equation*}
    Where $D$ is the diameter of the ball. Note that $D^{d}/\Vol$ is a constant independent of $r$. This completes the proof of the proposition.
  \end{proof}
  
  Now suppose that $u$ is a test function on $B(1)$. For each $z\in B$, we use the previous proposition to conclude
  \begin{equation*}
    \abs{u(z)}\le \abs{\bar{u}}+C \int_{B}\frac{\abs{\nabla u(y)}}{\abs{y-z}^{d-1}}\d\Vol(y).
  \end{equation*}
  We can estimate $\bar{u}\le C\norm{u}_{L^{p}}$. Now apply H\"older's inequality with $p,q$ to conclude
  \begin{equation*}
    \abs{u(z)}\le C\norm{u}_{L^{p}}+\norm{\nabla u}_{L^{p}}\left[\int \frac{1}{\abs{y-z}^{(d-1)q}}\d\Vol\right]^{1/q}.
  \end{equation*}
  We conclude
  \begin{equation*}
    \int \frac{1}{\abs{y-z}^{(d-1)q}}\d\Vol=C\int_{0}^{1}\rho^{(d-1)(1-q)}d\rho.
  \end{equation*}
  We compute
  \begin{equation*}
    1-\frac{d}{p}>\alpha\implies 1-d+\frac{d}{q}>\alpha\implies (q-1)(1-d)+1>q\alpha,
  \end{equation*}
  which implies $\rho^{(d-1)(1-q)}$ is integrable over $[0,1]$, and hence:
  \begin{equation*}\tag{1}
    \abs{u(z)}\le C(d,p)\norm{u}_{W^{1,p}}.
  \end{equation*}
  This proves that $W^{1,p}(B)$ functions satisfy a uniform $C^{0}$ bound. We now to establish oscillation estimates to conclude that $W^{1,p}(B)$ functions lie in $C^{0,\alpha}$. As before, let $u$ be a test function supported in $B$. Pick $x\in B$. For $z\in B_{r}(x)$, we apply the proposition to conclude
  \begin{equation*}
    \abs{u(z)-\bar{u}}\le C\int_{B_{r}(x)}\frac{\abs{\nabla u(y)}}{\abs{z-y}^{d-1}}\d\Vol(y).
  \end{equation*}
  We conclude, for $r<1$,
  \begin{equation*}
    \abs{u(z)-\bar{u}}\le C\norm{\nabla u}_{L^{p}}\int_{0}^{r}\rho^{(d-1)(1-q)}\,d\rho\le C\norm{\nabla u}_{L^{p}}r^{\alpha},
  \end{equation*}
  This was independent of $x$, and hence:
  \begin{equation*}
    \abs{u(z_{1})-u(z_{2})}\le 2C\norm{\nabla u}_{L^{p}}\abs{z_{1}-z_{2}}^{\alpha}.
  \end{equation*}
  Thus, combined with (1), we conclude
  \begin{equation*}\tag{2}
    \norm{u}_{C^{0,\alpha}}\le C\norm{u}_{W^{1,p}}.
  \end{equation*}
  This proves that $W^{1,p}$ includes into $C^{0,\alpha}$, as desired.  
\end{proof}
\begin{xca}
  If $k+\beta>\ell+\alpha$, use Arz\'ela Ascoli to conclude $C^{k,\beta}(B)\subset C^{\ell,\alpha}(B)$ is a compact inclusion, where $C^{k,\beta}(B)$ are the $C^{k,\beta}$ functions which vanish outside of $B$.
\end{xca}
\begin{xca}[Morrey compactness]
  If $k-d/p>\ell+\alpha$, $\alpha\in [0,1)$, then the inclusion
  \begin{equation*}
    W^{k,p}_{0}(B)\subset C^{\ell,\alpha}
  \end{equation*}
  is compact. 
\end{xca}
\begin{cor}
  Let $Y$ be a compact manifold with boundary, and suppose $k-d/p>\ell$. Then the inclusion of test sections of $E$ into $C^{\ell}(E)$ extends to a compact inclusion of $$W^{k,p}_{0}(Y,E)\subset C^{\ell}(E).$$
\end{cor}
\begin{proof}
  As usual, it suffices to prove this locally. However, locally, it is a consequence of the Morrey embedding theorem, and the previous exercise.
\end{proof}
\begin{example}
  Let $Y$ be a Riemann surface. Then Sobolev class functions $W^{1,p}(Y,\C^{n})$ are automatically continuous provided $p>2$. This is a crucial fact when trying to define $W^{1,p}$ Sobolev spaces of maps $Y\to M$, where $M$ is another manifold.
\end{example}
\begin{clear}{Remark}
  The two results: (i) Sobolev embedding/Rellich compactness and (ii) Morrey embedding/compactness are fundamental results in the study of partial differential operators on manifolds.
\end{clear}
\begin{xca}
  Let $f\in W^{k,p}(Y,E)$, where we do not assume that $f\in W^{k,p}_{0}(Y,E)$. Suppose that $k-p/d>\ell$. Prove that $f$ is of class $C^{\ell}$.

  Hint: First prove that $\rho f\in W^{k,p}_{0}(Y,E)$, if $\rho$ is compactly supported in the interior of $Y$. Conclude that $f$ is a sum of smooth sections, and hence is smooth. 
\end{xca}
\clearpage

\subsection*{Elliptic operators}

\begin{defn}
  Let $E_{i}\to Y$, $i=1,2$, be vector bundles over $Y$. Given a differential operator $L:E_{1}\to E_{2}$, we can consider the assignment
  \begin{equation*}
    f\in \Gamma(\C)\mapsto [L,f]\in \Gamma\Hom(E_{1},E_{2}).
  \end{equation*}
  This assignment is actually a first-order linear differential operator $\C\to \Hom(E_{1},E_{2})$. In fact, there exists a unique tensor $\sigma:T^{*}M\to \Hom(E_{1},E_{2})$ so that
  \begin{equation*}
    [L,f]=\sigma(\d f).
  \end{equation*}
  This tensor $\sigma$ is called the \textbf{symbol} of $L$. To see why there exists such a tensor $\sigma$, we review some results proved in the first section.

  \begin{xca}
    Prove that $[L,-]$ is a first order linear differential operator $f\mapsto [L,f]$. Hint: $[L,\varphi f]-\varphi[L,f]=[L,\varphi]f$.    
  \end{xca}
  \begin{xca}
    Since $\d$ is a connection on the trivial bundle $\C$, we already proved that we can express
    \begin{equation*}
      [L,f]=A(\d f)+B(f)
    \end{equation*}
    for some tensors $A$ and $B$. Use this to prove that $B=0$.

    We proved that $A:T^{*}Y\to \Hom(E_{1},E_{2})$ and $B:\C\to \Hom(E_{1},E_{2})$ are uniquely determined by $L$. Conclude that there is unique tensor $\sigma$ so that $[L,f]=\sigma(\d f)$.
  \end{xca}
  \end{defn}
  \begin{defn}
    A first order operator $L:E_{1}\to E_{2}$ is called elliptic if its symbol $\sigma$ maps $(T^{*}Y)^{\cross}$ into $\GL(E_{1},E_{2})$. Note that this forces $\dim E_{1}=\dim E_{2}$.    
  \end{defn}
  \begin{example}
    Let $Y$ be a manifold with $\dim Y\ge 2$. There are no elliptic first order operators on the trivial bundle $\R$. To see why, note that $\Hom(\R,\R)$ is one-dimensional, while $T^{*}Y$ is greater than $2$ dimensional, so the tensor $\sigma:T^{*}Y\to \Hom(\R,\R)$ must have some kernel. This is not true for higher order operators. This partially explains the prevalence of \emph{second} order operators when working with real-valued functions.
  \end{example}
  \begin{example}
    Let $\dim M>n$. Then there are no first order differential operators from $E_{1}\to E_{2}$, if $\dim E_{1}=\dim E_{2}=n$.

    To see why, consider the space $\Hom(E_{1},E_{2})\simeq \R^{n\cross n}$, and consider a $>n$-dimensional subspace $\Pi$ of $\R^{n\cross n}$. Suppose that $\Pi$ doesn't touch the singular set. Then we can find matrices $A_{0},\cdots,A_{n}$ so that
    \begin{equation*}
      \sum_{i}x_{i}A_{i}\text{ is non-singular }
    \end{equation*}
    for every $x_{0},\cdots,x_{n}$, not all $0$. In particular, for any vector $v$,
    \begin{equation*}
      \sum_{i}x_{i}A_{i}v\ne 0\text{ for all $x_{0},\cdots,x_{n}$}.
    \end{equation*}
    But this implies that we have found a linearly independent set with $>n$ elements in $E_{2}$, contrary to the requirement that $\dim E_{1}=\dim E_{2}=n$.
  \end{example}
  \begin{example}
    As a corollary to the previous example, consider a $>2$ dimensional manifold $X$ with almost complex structure $J$. The equation for holomorphic functions $X\to \C$ is not elliptic, because $\dim X>\dim \C=2$.
  \end{example}
    \begin{clear}{Constant coefficient first-order elliptic operators on $\T^{d}$}
    Let $\C^{n}\to \T^{d}$ be a trivial line bundle endowed the the standard metric $\ip{-,-}$, and suppose that $\T^{d}$ has the Euclidean metric $g$ and connection $\nabla$ inherited from $\R^{d}$. The trivial bundle has connection $\nabla=\d$ (let's agree to use the symbols $\nabla$ and $\d$ interchangeably for sections of a trivial bundle).

    We have shown that any first order operator is of the form
    \begin{equation*}
      Lu=A(\nabla u)+B(u),
    \end{equation*}
    Observe that
    \begin{equation*}
      L(fu)-fLu=A(\d f\ocross u).
    \end{equation*}
    Therefore
    \begin{equation*}
      \sigma(\d f)=A(\d f\ocross -).
    \end{equation*}
    Now recall that
    \begin{equation*}
      \nabla u=\d x_{i}\ocross \pd{u}{x_{i}},
    \end{equation*}
    and hence
    \begin{equation*}
      L(u)=\sum_{i}\sigma(\d x_{i})\pd{u}{x_{i}}+B(u)=\sum_{i}A_{i}\pd{u}{x_{i}}+B u.
    \end{equation*}
  \end{clear}
  \begin{example}
    Let $\sigma:T^{*}\T^{2}\to \Hom(\R^{2n},\R^{2n})$ by the symbol
    \begin{equation*}
      \sigma(a\d x+b\d y)=a+bJ,
    \end{equation*}
    where $J$ is a complex structure. This is clearly elliptic, because
    \begin{equation*}
      (a+bJ)^{-1}=\frac{a-bJ}{a^{2}+b^{2}}.
    \end{equation*}
    This induces a partial differential operator on $\R^{2n}\to \R^{2n}$ by
    \begin{equation*}
      \conj{\bd}_{U}(u)=\ud{u}{x}+J\ud{u}{y}.
    \end{equation*}
  \end{example}
  \begin{defn}
    We say that a differential operator $L:\C^{n}\to \C^{n}$ on the torus $\T^{d}$ has \textbf{constant coefficients}, provided
    \begin{equation*}
      Lu=A(\d u)+B(u)
    \end{equation*}
    where $A,B$ are constant tensors. The goal of the next section is to prove the interior elliptic estimates for constant coefficient linear elliptic operators on $\T^{d}$.
  \end{defn}
  \clearpage

  \subsection*{Interior elliptic estimates}
  Here is the general statement we will ultimately prove.
  \begin{thm}[Interior elliptic estimates]
    Let $L$ be a first-order elliptic operator $E_{1}\to E_{2}$, and let $Y$ be a compact manifold with boundary. For all $p\in (1,\infty)$, and all $k=1,2,\cdots$, there is constant $C=C(p,k)$ so that
    \begin{equation*}
      \norm{u}_{W^{k,p}(Y,E_{1})}\le C\left[\,\norm{L u}_{W^{k-1,p}(Y,E_{2})}+\norm{u}_{L^{p}(Y,E_{1})}\,\right].
    \end{equation*}
    for all $u\in W^{k,p}_{0}(Y,E_{1})$. 
  \end{thm}
  \begin{clear}{Remark}
    These estimates are called \textbf{interior} because they deal with the space $W^{k,p}_{0}$, which is the closure of $C^{\infty}_{c}(\text{int}(Y),E)$ in $W^{k,p}$. We still require $Y$ to be compact.
  \end{clear}
  \begin{xca}
    Let $L$ be a first order differential operator on $Y$. The very first thing to observe (which we have not said yet) is that linear differential operators extend to all distributions via duality -- this is possible since every operator has an adjoint (which is immediate since $L=A\circ \nabla+B$, where $A,\nabla,B$ all have adjoints). Moreover, $A,\nabla,B$ all extend to distributions, and the extensions satisfy
    \begin{equation*}
      L=A\circ \nabla+B.
    \end{equation*}

    By the $W^{\bullet,p}$ boundedness properties of $\nabla$ and tensors, we conclude a constant $C=C(L,k)$ so
    \begin{equation*}
      \norm{Lu}_{W^{k-1,p}}\le C\norm{u}_{W^{k,p}}.
    \end{equation*}
  \end{xca}

  Suppose that $L$ is constant coefficients on $\T^{d}$. We will prove that the elliptic estimates hold for $L$ using Fourier analysis on $\T^{d}$. We recall the necessary definitions.

  Define $e_{\ell}=\exp(i\ip{\ell,-}):\T^{d}\simeq \R^{d}/\Z^{d}\to \C$, for $\ell\in \Z^{d}$. Then $e_{\ell}$ are test functions which are orthonormal with respect to the $L^{2}$ inner product.
  \begin{thm}
    If $f\in L^{2}(\T^{d},\C^{n})$, then
    \begin{equation*}\tag{$\ast$}
      f=\sum_{\ell\in \Z}(e_{\ell},f) e_{\ell}.
    \end{equation*}
    Here the pairing $(e_{\ell},f)$ is vector valued.
  \end{thm}
  \begin{proof}
    This is standard, by now. By continuity of both sides, it suffices to prove it for smooth~$f$. It is clear that the set of linear combinations of $e_{k}$ is an  unital algebra closed under conjugation, and moreover the algebra seperates points in the sense that for any two points $x,y$ we can find a function $\varphi$ in the algebra generated by $e_{k}$ so that $\varphi(x)\ne \varphi(y)$. Then by the Stone-Weierstrass theorem, the span of $\set{e_{k}}$ is dense in $C^{0}$. Since the $C^{0}$ norm controls the $L^{2}$ norm, and $C^{0}$ is dense in $L^{2}$, we conclude $e_{\ell}$ are dense in $L^{2}$, and hence they form an orthonormal basis for the Hilbert space $L^{2}$.

    To prove that ($\ast$) holds, we apply the fact that $e_{\ell}$ are a basis to each of the component functions of $f$. 
  \end{proof}
  \begin{prop}
    Suppose that a section $f$ has a weak derivative $\nabla f\in \Hom(\C^{n},\C^{n})$. Then
    \begin{equation*}
      (\d x_{j}\ocross s_{i}e_{\ell},\nabla f)=(\nabla^{*}(\d x_{j}\ocross s_{i}e_{\ell}),f)=-i\ell_{j}(e_{\ell},f_{i}),
    \end{equation*}
    where $s_{1},\cdots,s_{n}$ be an orthonormal frame $\C^{n}$, and $f_{i}$ denotes $\ip{s_{i},f}$.
    
    We remark that the sections $\d x_{j}\ocross s_{i}e_{\ell}$ form an orthonormal basis for $L^{2}(T^{*}\T^{d})$, since they are certainly orthonormal and contain a generating set of smooth sections 
    \begin{equation*}
      \d x_{j}\ocross u,
    \end{equation*}
    where $u$ is a smooth $\C^{n}$ valued function in the closure of their span.
    
    As a consequence, if $f\in W^{1,2}$, then
    \begin{equation*}
      \norm{\nabla f}_{L^{2}}^{2}=\sum_{\ell}\abs{\ell}^{2}\abs{(e_{\ell},f)}^{2}<\infty.
    \end{equation*}
    Conversely, if $f$ is a distribution and the Fourier coefficients of $\nabla f$ are in $\ell^{2}$, then $f$ is $W^{1,2}$. Note that $(e_{\ell},f)$ is a vector, and so $\abs{(e_{\ell},f)}$ is measured using the inner product on $\C^{n}$.
  \end{prop}
  \begin{proof}
    Since $f$ is a distribution, $(e_{0},f)<\infty$, and hence the sum
    \begin{equation*}
      \sum_{\ell}\abs{\ell}^{2}\abs{(e_{\ell},f)}^{2}<\infty
    \end{equation*}
    implies that $\norm{f}_{L^{2}}<\infty$, and so $f$ is represented by a $L^{2}$ function. By assumption that the fourier series for the distribution $\nabla f$ converges, we conclude that $\nabla f$ is an $L^{2}$ section of $\Hom(T\T,\C^{n})$, and hence $f$ is in $W^{1,2}$. We also observe that
    \begin{equation*}
      \norm{f}_{W^{1,2}}^{2}=\sum_{\ell}(1+\abs{\ell}^{2})\abs{(e_{\ell},f)}^{2}.
    \end{equation*}
  \end{proof}
  More generally, the sections
  \begin{equation*}
    \d x_{j_{1}}\ocross\cdots\ocross \d x_{j_{k}}\ocross e_{\ell}\, s_{i}
  \end{equation*}
  form an orthonormal basis for $L^{2}(\Hom(TM^{\ocross k},\C^{n}))$. It is easy to check that
  \begin{equation*}
    \nabla^{*}(\d x_{j_{1}}\ocross \d x_{j_{2}}\ocross \cdots\ocross \d x_{j_{k}}\ocross e_{\ell}\, s_{i})=\d x_{j_{2}}\ocross \cdots\ocross\d x_{j_{k}} \ell_{j_{1}}\,e_{\ell}\,s_{i},
  \end{equation*}
  (i.e.\ this can be figured out fairly easily remembering our general formula for the adjoint of a connection). Therefore, if $f$ is a section of $\C^{n}$, then the coefficients of $\nabla^{k}f$ are
  \begin{equation*}
    (\d x_{j_{1}}\ocross \d x_{j_{2}}\ocross \cdots\ocross \d x_{j_{k}}\ocross e_{\ell}\, s_{i},\nabla^{k}f)=\ell_{j_{1}}\cdots\ell_{j_{k}}(e_{\ell},\ip{s_{i},f}).
  \end{equation*}
  Therefore
  \begin{equation*}
    \norm{\nabla^{k}f}_{L^{2}}^{2}=\sum\abs{\ell}^{2k}\abs{(e_{\ell},f)}^{2}.
  \end{equation*}
  Hence for $f\in W^{k,2}$, we have
  \begin{equation*}    \norm{f}_{W^{k,2}}^{2}=\sum_{\ell}(1+\abs{\ell}^{2}+\cdots+\abs{\ell}^{2k})\abs{(e_{\ell},f)}^{2}.
  \end{equation*}

  Now suppose that $L$ is an first-order elliptic operator with constant coefficients,
  \begin{equation*}
    L(u)=A(\nabla u)+B(u).
  \end{equation*}
  We compute
  \begin{equation*}
    (e_{\ell} s_{i},L(u))=(e_{\ell} s_{i},A(\nabla u)+B(u))=(\nabla^{*}A^{*}(e_{\ell} s_{i})+B^{*}(e_{\ell} s_{i}),u),
  \end{equation*}
  where, recall, $s_{1},\cdots,s_{n}$ are the constant orthonormal frame for $\C^{n}$. Since $A$ is constant, so is $A^{*}$, and hence
  \begin{equation*}
    \nabla^{*}e_{\ell} A^{*}(s_{i})=-e_{\ell}\,i\ell\intprod A^{*}(s_{i})+e_{\ell} \nabla^{*}A^{*}(s_{i})
  \end{equation*}
  \begin{xca}
    Prove this relation by exploring the commutator $[\nabla^{*},e_{\ell}]$.
  \end{xca}
  Then, we conclude
  \begin{equation*}
    (e_{\ell} s_{i},L(u))=(-e_{\ell}\,i\ell\intprod A^{*}(s_{i})+e_{\ell} \nabla^{*}A^{*}(s_{i})+e_{\ell}B(s_{i}),u).
  \end{equation*}
  Since $(e_{\ell},L(u))=(e_{\ell}s_{i},L(u))s_{i}$, conclude
  \begin{equation*}
    (e_{\ell},L(u))=\sum_{i}(-e_{\ell},\ip{i\ell\intprod A^{*}(s_{i}),u})s_{i}+(e_{\ell},\ip{\nabla^{*}A^{*}(s_{i})+B(s_{i}),u})s_{i}.
  \end{equation*}
  \begin{xca}
    If $s$ is a constant section, then $(e_{\ell},\ip{s,u})=\ip{s,(e_{\ell},u)}$.
  \end{xca}
  We therefore obtain the nice formula
  \begin{equation*}
    (e_{\ell},L(u))=\sum_{i}\ip{i\ell\intprod A^{*}(s_{i}),\hat{u}(\ell)}+\ip{\nabla^{*}A^{*}(s_{i})+B(s_{i}),\hat{u}(\ell)}s_{i}.
  \end{equation*}
  We conclude
  \begin{equation*}
    \sum_{i,\ell}\abs{\ip{\ell\intprod A^{*}(s_{i}),\hat{u}(\ell)}}^{2}\le \norm{L(u)}_{L^{2}}^{2}+c\norm{u}_{L^{2}}^{2}.
  \end{equation*}
  To conclude the elliptic estimate for $W^{1,2}$, we will try to bound
  \begin{equation*}
    \sum_{\ell}\abs{\ell}^{2}\abs{\hat{u}(\ell)}^{2}\le C \sum_{i,\ell}\abs{\ip{\ell\intprod A^{*}(s_{i}),\hat{u}(\ell)}}^{2}.
  \end{equation*}
  \begin{claim}
    There is a constant $c$ so that for all vectors $v$
    \begin{equation*}
      c\abs{\ell}^{2}\abs{v}^{2}\le \sum_{i}\abs{\ip{\ell\intprod A^{*}(s_{i}),v}}^{2}.
    \end{equation*}
    This is where ellipticity of $L$ comes in. 
  \end{claim}
  \begin{proof}
    Since $L$ is elliptic, the map $v\mapsto A(\ell\ocross v)$ is an isomorphism for each non-zero $\ell$. 

    The adjoint of $A$ is defined by formally taking the adjoint of the map
    \begin{equation*}
      A:\Hom(TM,\C^{n})\to \C^{n},
    \end{equation*}
    so
    \begin{equation*}
      \ip{A(\ell\ocross v),s_{i}}=\ip{\ell\ocross v,A^{*}(s_{i})}=\ip{v,\ell\intprod A^{*}(s_{i})}.
    \end{equation*}
    Therefore we conclude that
    \begin{equation*}
     v \mapsto \ip{v,\ell\intprod A^{*}(s_{i})}s_{i}
    \end{equation*}
    is an injection. Define $c$ by
    \begin{equation*}
      c=\inf_{\abs{v}=\abs{\ell}=1}\sum_{i}\abs{\ip{v,\ell\intprod A^{*}(s_{i})}}^{2},
    \end{equation*}
    and note that $c$ is the infimum of a continuous function on a compact set. Since $c$ is never vanishing we conclude $c(v)>0$. By bilinearity, we conclude
    \begin{equation*}
      c\abs{\ell}^{2}\abs{v}^{2}\le \sum_{i}\abs{\ip{v,\ell\intprod A^{*}(s_{i})}}^{2}.
    \end{equation*}
    Here it was useful to temporarily allow $\ell$ to be a non-integer. This completes the proof.
  \end{proof}
  \begin{cor}[interior estimates for $p=2$]
    Let $L$ be a constant coefficient first order elliptic operator on the torus. Then for all $u\in W^{1,2}$, we have
    \begin{equation*}
      \norm{u}_{W^{1,2}}\le C\left[\,\norm{Lu}_{L^{2}}+\norm{u}_{L^{2}}\,\right].
    \end{equation*}
  \end{cor}
  \begin{xca}
    Prove that the interior elliptic estimates for the trivial
    complex bundle $\C^{n}$ imply the interior estimates for the
    trivial real bundle $\R^{n}$.
  \end{xca}
  
  We will see later that the interior elliptic estimates + a general elliptic regularity theorem are enough to construct a parametrix (in general). We can be more explicit in the constant coefficient case and explicitly describe a parametrix. We have shown above that
  \begin{equation*}
    (e_{\ell},L(u))=\sum_{i}\ip{i\ell\intprod A^{*}(s_{i}),\hat{u}(\ell)}s_{i}+\ip{\nabla^{*}A^{*}(s_{i})+B(s_{i}),\hat{u}(\ell)}s_{i}.
  \end{equation*}
  Consider
  \begin{equation*}
    v\mapsto \sum_{i}\ip{i\ell\intprod A^{*}(s_{i}),v}s_{i}+\sum_{i}\ip{\nabla^{*}A^{*}(s_{i})+B(s_{i}),v}s_{i}.
  \end{equation*}
  as a map $\C^{n}\to \C^{n}$. We have shown that the norm (squared) of the first term is bounded below by $c\abs{\ell}^{2}\abs{v}^{2}$. It is clear that the norm of the second term is bounded above by $C\abs{v}$, and hence for all but finitely many $\ell$, the map
  \begin{equation*}
    v\mapsto \sum_{i}\ip{i\ell\intprod A^{*}(s_{i}),v}s_{i}+\sum_{i}\ip{\nabla^{*}A^{*}(s_{i})+B(s_{i}),v}s_{i}.
  \end{equation*}
  is invertible. Indeed, by pick $\ell$ sufficiently large, we may suppose that there is a neighborhood of infinity $\Lambda\subset \Z^{d}$ so that
  \begin{equation*} 
    \abs{\ell}\abs{v}<C\abs{\sum_{i}\ip{i\ell\intprod A^{*}(s_{i}),v}s_{i}+\sum_{i}\ip{\nabla^{*}A^{*}(s_{i})+B(s_{i}),v}s_{i}},
  \end{equation*}
  for all $\ell\in \Lambda$, and all $v$. Given any collection of vectors $b_{\ell}$, we can find unique $a_{\ell}$, for $\ell\in \Lambda$ so that
  \begin{equation*}
    (e_{\ell},L(a_{\ell}))=b_{\ell},
  \end{equation*}
  and moreover,
  \begin{equation*}\tag{$\ast$}
    \abs{a_{\ell}}\le C\frac{\abs{b_{\ell}}}{\ell}\text{ for $\ell\in \Lambda$}.
  \end{equation*}
  Suppose now that $w=\sum_{\ell}b_{\ell}e_{\ell}$, and $w$ lies in $L^{2}$. Let $\Pi$ be finite dimensional subspace spanned by $e_{\ell}$, $\ell\not\in \Lambda$.

  Define
  \begin{equation*}
    u:=P(w)=\sum_{\Lambda}a_{\ell}e_{\ell}\text{ so that }L(u)-w\text{ lies in $\Pi$}.
  \end{equation*}
  This prescription defines a bounded linear map $L^{2}\to W^{1,2}$, since we have shown ($\ast$).

  Note that $LP(w)=w-F(w)$, where $F$ is a finite rank operator (projection onto $\Pi$). Similarly, observe that $PL(u-F(u))=u-F(u)$, so
  \begin{equation*}
    PL(u)=u-G(u),
  \end{equation*}
  where $G$ is some other finite rank operator. Therefore we have inverted $L$ up to finite rank operators. Observe that
  \begin{equation*}
    \norm{u}_{W^{1,2}}\le \norm{u-G(u)}_{W^{1,2}}+\norm{G(u)}_{W^{1,2}}\le C\norm{Lu}_{L^{2}}+\norm{G(u)}_{W^{1,2}}.
  \end{equation*}
  Note that for $u$ restricted to the orthogonal complement of $\ker G$, we know that $G$ is continuous with respect to any topology on it's finite rank target. We therefore have
  \begin{equation*}
    \norm{G(u)}_{W^{1,2}}\le C\norm{u}_{L^{2}},
  \end{equation*}
  for $u\perp \ker G$, and hence this holds for all $u$. Thus we conclude
  \begin{equation*}
    \norm{u}_{W^{1,2}}\le C\left[\,\norm{Lu}_{L^{2}}+\norm{u}_{L^{2}}\,\right].
  \end{equation*}
  Thus we see that the existence of a parametrix implies the interior estimates.         
\begin{thm}
  Let $L$ be a first order constant coefficient elliptic operator on $\T^{d}$. Then for all $u\in W^{1,p}$, and all $p>0$, we have
  \begin{equation*}
    \norm{u}_{W^{1,p}}\le C\left[\,\norm{Lu}_{L^{p}}+\norm{u}_{L^{p}}\,\right].
  \end{equation*}  
\end{thm}
\begin{remark}
  We won't prove this theorem, because I think it is quite hard. Even for the standard Cauchy-Riemann operator $\bd_{x}+i\bd_{y}$, I think one needs to use ``Calderon-Zygmund'' theory.

  One method to prove it would be to try to show that our parametrix $P$ is bounded from $L^{p}\to W^{1,p}$, but this seems hard.

  However, assuming this theorem, we can prove the interior elliptic estimates in general without too much additional work.

  It is desirable to have the elliptic estimates for $p>2$, because the Sobolev spaces $W^{1,p}(\Sigma,-)$ are made of continuous functions when $p>2$ -- this is important for developing the non-linear theory. We could try to develop the non-linear theory using the spaces $W^{2,2}(\Sigma,-)$. One meta-reason this is difficult is that ``the first estimate is the hardest to establish,'' and it is generally easier to establish estimates which begin with lower regularity.

  Note that the spaces $W^{1,2}(\S^{1},-)$ and $W^{1,2}([0,1],-)$ consist of continuous function (when $p=2$). This implies that non-linear theory for elliptic equations on one-dimensional manifolds can be approached using the $W^{k,2}$ spaces. Non-linear elliptic equations on one-dimensional manifolds can be thought of as the theory of ordinary differential equations.
\end{remark}
\begin{xca}
  Construct a parametrix $P:L^{2}(\T^{2})\to W^{1,2}(\T^{2})$ for the constant coefficient elliptic differential operator
  \begin{equation*}
    u\mapsto \conj{\bd} u=u_{x}+iu_{y}.
  \end{equation*}
  Hint: the fourier coefficients of $Lu$ are
  \begin{equation*}
    (e_{\ell},Lu)=i(\ell_{1}+i\ell_{2})\hat{u}(\ell).
  \end{equation*}
  Therefore we can invert $L$ whenever $\ell\ne 0$. Conclude a unique parametrix $P:L^{2}\to W^{1,2}$ so that
  \begin{equation*}
    u-LPu=\int u\,\d\Vol,
  \end{equation*}
  and so that
  \begin{equation*}
    u-PLu=\int u\,\d\Vol.
  \end{equation*}
  Conclude that the index of $\conj{\bd}$ is $0$.
\end{xca}

\begin{clear}{Interior elliptic estimates in the non-constant
    coefficient case}  
First we will generalize to the case when we have a differential operator on the torus which does not have constant coefficients. Write
\begin{equation*}
  L(u)=A(\nabla u)+B(u).
\end{equation*}
Recall that we are trying to prove
\begin{equation*}
  \norm{u}_{W^{1,p}}\le C\left[\norm{Lu}_{L^{p}}+\norm{u}_{L^{p}}\right].
\end{equation*}
Since
\begin{equation*}
  \norm{A(\nabla u)}_{L^{p}}\le \norm{L(u)}_{L^{p}}+C\norm{u}_{L^{p}},
\end{equation*}
it suffices to bound $\norm{\nabla u}_{L^{p}}$ by $\norm{A(\nabla u)}_{L^{p}}$.

Now cover $\T^{d}$ by $N^{d}$ overlapping squares $U_{\alpha}$ of size $(2/N)^{d}$, and let $x_{\alpha}\in U_{\alpha}$ be the center of the square.

Let $L_{\alpha}(u)=A_{\alpha}(\nabla u)+B_{\alpha}(u)$. This defines a constant coefficient differential operator on the entire torus $\T^{d}$.

Choose a partition of unity $\rho_{\alpha}$ for $U_{\alpha}$, with the property that
\begin{equation*}
  \abs{\nabla \rho_{\alpha}}<CN,
\end{equation*}
where $C$ is independent of $N$.

Suppose we can prove that
\begin{equation*}\tag{$\ast$}
  \norm{\rho_{\alpha}u}_{W^{1,p}}\le C_{\alpha}\left[\norm{L\rho_{\alpha}u}_{L^{p}}+\norm{u}_{L^{p}}\right]
\end{equation*}

Then we estimate
\begin{equation*}
  \norm{u}_{W^{1,p}}\le \sum_{\alpha}\norm{\rho_{\alpha}u}_{W^{1,p}}\le \sum_{\alpha}C_{\alpha}\left[\norm{L\rho_{\alpha}u}_{L^{p}}+\norm{u}_{L^{p}}\right].
\end{equation*}
Now write
\begin{equation*}
  L\rho_{\alpha}u=\rho_{\alpha}Lu+[L,\rho_{\alpha}]u
\end{equation*}
and thereby conclude
\begin{equation*}
  \norm{L\rho_{\alpha}u}_{L^{p}}+\norm{u}_{L^{p}}\le \norm{\rho_{\alpha}Lu}_{L^{p}}+\norm{[L,\rho_{\alpha}]u}_{L^{p}}+\norm{u}_{L^{p}}\le \left[\norm{Lu}_{L^{p}}+c(N)\norm{u}_{L^{p}}\right].
\end{equation*}
The constant $c(N)$ may be quite large (it involves the derivatives of $\rho_{\alpha}$, which are bounded by $N$). We ultimately conclude
\begin{equation*}
  \norm{u}_{W^{1,p}}\le C(N) \left[\norm{Lu}_{L^{p}}+\norm{u}_{L^{p}}\right],
\end{equation*}
for some large constant $C(N)$. Therefore it suffices to prove that ($\ast$) holds for some finite $N$.

Now is where we use the constant coefficient operators $L_{\alpha}$. Consider the operators $L_{x}=A_{x}(\nabla u)+B_{x}(u)$ as $x$ ranges over $\T^{d}$. Each one satisfies an estimate of the form
\begin{equation*}
  \norm{u}_{W^{1,p}}\le C_{x}(\norm{L_{x}u}_{L^{p}}+\norm{u}_{L^{p}}).
\end{equation*}
We claim that there is a constant $C$ independent of $x$ so that
\begin{equation*}
  \norm{u}_{W^{1,p}}\le C(\norm{L_{x}u}_{L^{p}}+\norm{u}_{L^{p}})
\end{equation*}
holds for all $x$. This is a compactness argument. To see why, observe that for each $x\in \T^{d}$, there is a ball $B$ around $x$ and a constant $C$ so that $y\in B$ implies
\begin{equation*}
  \norm{u}_{W^{1,p}}\le C(\norm{L_{y}u}_{L^{p}}+\norm{u}_{L^{p}}).
\end{equation*}
Then by covering $\T^{d}$ by finitely many such balls and taking the maximum of the constants $C$ proves the claim.

We know that
\begin{equation*}
  \norm{\rho_{\alpha}u}_{W^{1,p}}\le C(\norm{L_{\alpha}\rho_{\alpha}u}_{L^{p}}+\norm{u}_{L^{p}})\le C(\norm{L\rho_{\alpha}u}_{L^{p}}+\norm{(L_{\alpha}-L)\rho_{\alpha}u}_{L^{p}}+\norm{u}_{L^{p}}).
\end{equation*}
We estimate

\begin{equation*}
  \begin{aligned}
    \norm{(L_{\alpha}-L)\rho_{\alpha}u}_{L^{p}}\le \frac{1}{N}\norm{L}_{C^{1}}\norm{\rho_{\alpha}u}_{W^{1,p}}.
  \end{aligned}
\end{equation*}
Combining everything, we obtain
\begin{equation*}
  \norm{\rho_{\alpha}u}_{W^{1,p}}\le C\left[\norm{L\rho_{\alpha}u}+\frac{\norm{L}_{C^{1}}}{N}\norm{\rho_{\alpha}u}_{W^{1,p}}+C(N)\norm{u}_{L^{p}}\right].
\end{equation*}
Picking $N$ large enough so $C\norm{L}_{C^{1}}/N<1/2$, we conclude
\begin{equation*}
  \norm{\rho_{\alpha}u}_{W^{1,p}}\le 2C\left[\norm{L\rho_{\alpha}u}+C(N)\norm{u}_{L^{p}}\right],
\end{equation*}
as desired. This completes the proof of the elliptic estimates for
first order operators on $\T^{d}$.
\end{clear}

We can now complete the proof of the interior elliptic estimates:
\begin{thm}[interior elliptic estimates for $W^{1,p}$]
  Let $Y$ be a compact manifold with boundary, and let $L$ be a first
  order elliptic operator $E_{1}\to E_{2}$. There is a constant $C(p)$
  so that for all $u\in
  W^{1,p}_{0}(Y,E_{1})$, we have
  \begin{equation*}
    \norm{u}_{W^{1,p}}\le C(\norm{Lu}_{L^{p}}+\norm{u}_{L^{p}}).
  \end{equation*}
\end{thm}
\begin{proof}
  First we show that it holds locally on $Y$. Cover $Y$ by finitely many compact
  balls (or half balls) with boundary $U_{\alpha}$, each equipped with an embedding
  $\varphi_{\alpha}:U_{\alpha}\to \T^{d}$ trivializing $E_{1},E_{2}$.

  Using the connection and metric $U_{\alpha}$ inherits from
  $\T^{d}$ and the trivial bundle on $\T^{d}$, we can write
  \begin{equation*}
    L=A_{\alpha}(\nabla u_{\alpha})+B_{\alpha}(u_{\alpha})\text{ for
      $u_{\alpha}\in W^{1,p}_{0}(U_{\alpha})$.}
  \end{equation*}
  The tensors $A_{\alpha},B_{\alpha}$ are defined on the codimension
  $0$ image of $U_{\alpha}$ in $\T^{d}$. Taking a tubular
  neighborhood of $U_{\alpha}\subset \T^{d}$, we can extend
  $A_{\alpha},B_{\alpha}$ smoothly to global sections on $\T^{d}$ of the appropriate
  bundles. Then we conclude for $u\in W^{1,p}_{0}(U_{\alpha})\subset W^{1,p}_{0}(\T^{d})$ 
  \begin{equation*}\tag{$\ast$}
    \norm{u}_{W^{1,p}}\le C(\norm{Lu}_{L^{p}}+\norm{u}_{L^{p}}),
  \end{equation*}
  this implies the same result when we use the norms that
  $W^{1,p}_{0}(B_{\alpha})$ inherits from $W^{1,p}_{0}(Y)$, since the
  Sobolev norms defined on a compact manifold do not depend on the
  metric/connection data used (up to norm equivalence).

  Now let $\rho_{\alpha}$ be a partition of unity for the balls
  $U_{\alpha}$. We estimate
  \begin{equation*}
    \norm{u}_{W^{1,p}}\le
    \sum_{\alpha}\norm{\rho_{\alpha}u}_{W^{1,p}}\le C\sum_{\alpha} \norm{\rho_{\alpha}Lu}_{p}+\norm{[L,\rho_{\alpha}]u}_{L^{p}}+\norm{\rho_{\alpha}u}_{L^{p}},
  \end{equation*}
  for some large $C$ (the maximum of the constants we were
  guaranteed in ($\ast$)). We can estimate the three terms appearing
  in the sum
  \begin{equation*}
    \begin{aligned}
      \sum_{\alpha}\norm{\rho_{\alpha}Lu}_{L^{p}}&\le (\text{number of
        balls})\norm{Lu}_{L^{p}}\\
      \sum_{\alpha}\norm{[L,\rho_{\alpha}]u}_{L^{p}}&\le
      \left[\sum_{\alpha}\norm{[L,\rho_{\alpha}]}_{C^{0}}\right]\norm{u}_{L^{p}}\\
      \sum_{\alpha}\norm{\rho_{\alpha}u}_{L^{p}}&\le (\text{number of
        balls})\norm{u}_{L^{p}},
    \end{aligned}    
  \end{equation*}
  we conclude for $u\in W^{1,p}_{0}$ that
  \begin{equation*}
    \norm{u}_{W^{1,p}}\le C(\norm{Lu}_{L^{p}}+\norm{u}_{L^{p}}).
  \end{equation*}
  This completes the proof.    
\end{proof}
\begin{xca}
  Prove that if $u\in W^{1,p}_{0}(Y)$ then
  $\rho_{\alpha}u\in W^{1,p}_{0}(B_{\alpha})$. Hint: the boundary case
  requires some proof since, since $\rho_{\alpha}$ is potentially
  non-vanishing on the boundary face $B_{\alpha}\cap \bd Y$.
\end{xca}

\subsection*{Elliptic bootstrapping}

\begin{thm}
  Let $u\in L^{p}(Y)$ be supported away from $\bd Y$, and suppose that
  $Lu\in L^{p}(Y)$ (where $Lu$ is defined by duality using the adjoint
  $L^{*}$). Then $u\in W^{1,p}_{0}(Y)$. 
\end{thm}
\begin{clear}{Remark}
  Here $u$ being ``supported away from $\bd Y$'' is equivalent to the existence
  of a bump function $\rho$ supported in $\text{int}(Y)$ so that $\rho
  u=u$.
\end{clear}
\begin{clear}{Remark}
  The idea for the proof (in the case when $Y=B(1)$) will be to consider the convolutions
  $\rho_{\delta}\ast u$ (which are well defined elements of
  $W^{1,p}_{0}(B(1))$ for $\delta$ sufficiently small because of our
  assumption on the support of $u$). If we can show that
  $\rho_{\delta}\ast u$ is uniformly bounded in $W^{1,p}_{0}$, then
  reflexivity of $W^{1,p}_{0}$, and the fact that $\rho_{\delta}\ast
  u\to u$ in $L^{p}$, we can deduce that $u$ is in fact in
  $W^{1,p}_{0}$.
\end{clear}
\begin{defn}
  Let $T_{\delta}u=\rho_{\delta}\ast u$, i.e.
  \begin{equation*}
    T_{\delta}(u)=\int_{\R^{d}} u(y)(\tau^{-x}\rho_{\delta})(y)\,\d y,
  \end{equation*}
  where $\rho:B(1)\to [0,\infty)$ is a radially symmetric bump
  function with integral $1$. This formula uses the antipodal symmetry of
  $\rho_{\delta}$.

  This defines a continuous linear
  map $L^{p}(B(1))\to C^{\infty}(B(1+\delta))$, and moreover
  \begin{equation*}
    \rho_{\delta}\ast u\to u\text{ in $L^{p}$}.
  \end{equation*}
\end{defn}

The key part of the proof is the following ``commutator estimate''
\begin{lemma}
  Let $L$ be a first order differential operator on $\R^{d}$. There is
  a constant $C$ independent of $\delta$ so that for all $u\in L^{p}(B(1))$ we have
  \begin{equation*}
    \norm{[L,T_{\delta}]u}_{L^{p}}\le C\norm{u}_{L^{p}}.
  \end{equation*}
\end{lemma}
\begin{proof}[of lemma]
    We will prove the case when $L=A(\nabla u)$. The reader can prove
    the case $L=B(u)$, and thereby obtain it for all first order
    differential operators. Let $g$ be a test section in $L^{q}$.

    We compute
    \begin{equation*}
      \ip{A(\nabla T_{\delta}(u)),g}=\ip{T_{\delta}(\nabla
        u),A^{*}g}=\int \ip{\nabla u(y), \tau^{-x}\rho_{\delta}(y)A^{*}_{x}g(x)}\,\d y.
    \end{equation*}
    Using the adjoint of $\nabla$ (with respect to the $y$-variable,
    we consider $x$ to be fixed for the moment), we conclude
    \begin{equation*}
     \ip{LT_{\delta}u,g}=\ip{A(\nabla T_{\delta}(u)),g}=\int \ip{u(y), \tau^{-x}[\nabla^{*},\rho_{\delta}]A^{*}_{x}g(x)}\,\d y,
   \end{equation*}
   where we use the fact that $\nabla$ commutes with $\tau^{-x}$ and
   that $\nabla^{*}A^{*}_{x}g(x)=0$, since we consider $x$ fixed.

   \begin{xca}
     Prove that the symbol of $\nabla^{*}:\Hom(TM,\R^{n})\to \R^{n}$ is
     \begin{equation*}
       [\nabla^{*},f]=\text{grad}(f)\intprod (-).
     \end{equation*}
     This is a tensor, and hence $\nabla^{*}$ is a first order
     operator.          
   \end{xca}
   \begin{xca}
     Prove that if $w$ is a constant section of $\Hom(TM,\R^{n})$,
     then $\nabla^{*}w=0$. Hint:
     \begin{equation*}
       \int \ip{\nabla^{*}w,u}\dVol=\int \ip{w,\nabla u}\d\Vol
     \end{equation*}
     Write
     \begin{equation*}
       w=\sum \d x_{i}\ocross v_{i}\text{ and }\nabla u=\sum \d
       x_{i}\ocross \pd{u}{x_{i}}.
     \end{equation*}
     Conclude that
     \begin{equation*}
       \int \ip{w,\nabla u}\d\Vol=\sum_{i}\int
       \ip{v_{i},\pd{u}{x_{i}}}\,\d\Vol=0,
     \end{equation*}
     since $v_{i}$ are constant (use compatibility of the connection
     with $\ip{-,-}$). 
   \end{xca}
   Continuing with the proof of the claim, we now explore the term
   $T_{\delta}Lu$.
   \begin{equation*}
     \begin{aligned}
     \ip{T_{\delta}Lu,g}&=\int_{\R^{d}}\ip{A_{y}\nabla_{y}u,\tau^{-x}\rho_{\delta}(y)g(x)}\d
     y
     \\
     &=\int_{\R^{d}}\ip{u,\tau^{-x}[\nabla^{*},\rho_{\delta}]A^{*}_{y}g(x)}\d
     y+\int_{\R^{d}}\ip{u,\,\tau^{-x}\rho_{\delta}\cdot \nabla^{*}(A^{*}_{y}\cdot
       g(x))}\d y.
     \end{aligned}
   \end{equation*}
   Therefore
   \begin{equation*}
     \ip{T_{\delta}Lu-LT_{\delta}u,g}=\int_{\R^{d}}\ip{u,\tau^{-x}[\nabla^{*},\rho_{\delta}](A^{*}_{y}-A^{*}_{x})g(x)}\d
     y+\int_{\R^{d}}\ip{u,\,\tau^{-x}\rho_{\delta}\cdot \nabla^{*}(A^{*}_{y}\cdot
       g(x))}\d y.    
   \end{equation*}
   Taking absolute values produces
   \begin{equation*}\tag{$\ast$}
     \abs{\ip{T_{\delta}Lu-LT_{\delta}u,g}}\le
       \left[\norm{\rho_{\delta}}_{C^{1}}\cdot
       \text{osc}(A^{*};\delta)\abs{g(x)}+\norm{A^{*}}_{C^{1}}\abs{g(x)}\,\right]\int_{x+B(\delta)}\abs{u(y)}\,\d
         y.
     \end{equation*}
     where
     \begin{equation*}
       \text{osc}(A^{*};\delta)=\sup_{\abs{x-y}<\delta}\abs{A^{*}(y)-A^{*}(x)}.
     \end{equation*}
     Integrating ($\ast$) over $B(1)$ conclude
     \begin{equation*}
       \abs{(T_{\delta}Lu-LT_{\delta}u,g)}\le \left[\norm{\rho_{\delta}}_{C^{1}}\text{osc}(A^{*};\delta)+\norm{A^{*}}_{C^{1}}\right]\norm{\int_{x+B(\delta)}u(y)\,\d y}_{L^{p}}\norm{g}_{L^{q}},
     \end{equation*}
     and hence, conclude
     \begin{equation*}
       \norm{[T_{\delta},L]u}_{L^{p}}\le C\left[\norm{\rho_{\delta}}_{C^{1}}\text{osc}(A^{*};\delta)+\norm{A^{*}}_{C^{1}}\right]\norm{u}_{L^{p}},
     \end{equation*}
     where we have used the (easy) fact that
     \begin{equation*}
       \norm{\int_{x+B(\delta)}u(y)\,\d y}_{L^{p}}\le \norm{u}_{L^{p}}.
     \end{equation*}
     This can be proved by pairing with a test function. Now observe
     that $\text{osc}(A^{*};\delta)<\norm{A^{*}}_{C^{1}}\delta$, and
     $\norm{\rho_{\delta}}_{C^{1}}=\delta^{-1}\norm{\rho_{1}}_{C^{1}}$, so
     \begin{equation*}
       \norm{[T_{\delta},L]u}_{L^{p}}\le C(1+\norm{\rho_{1}}_{C^{1}})\norm{A^{*}}_{C^{1}}\norm{u}_{L^{p}}.
     \end{equation*}
     This proves the claim, as we can take $C$ to be
     $C(1+\norm{\rho_{1}}_{C^{1}})\norm{A^{*}}_{C^{1}}$ which does not
     depend on $\delta$.     
   \end{proof}

   Now we can prove the theorem.
   \begin{proof}
     We are given $u,Lu\in L^{p}$ and we know that $u$ is supported
     away from the boundary, and we want to show that $u\in
     W^{1,p}_{0}$.

     First suppose that $u$ is supported in $B(r)\subset B(1)$ for
     some $r<1$. For $\delta<r$, $\rho_{\delta}\ast u$ is supported in
     $B(1)$. We compute
     \begin{equation*}
       L(\rho_{\delta}\ast u)=\rho_{\delta}\ast (L u)+[L,T_{\delta}]u,
     \end{equation*}
     and hence
     \begin{equation*}\tag{$\ast$}
       \norm{L(\rho_{\delta}\ast u)}_{L^{p}}\le \norm{Lu}_{L^{p}}+C\norm{u}_{L^{p}},
     \end{equation*}
     where the constant $C$ is given by our ``commutator estimate.''

     By the interior elliptic estimates applied to $\rho_{\delta}\ast
     u\in W^{1,p}_{0}$, together with ($\ast$) we conclude
     \begin{equation*}
       \norm{\rho_{\delta}\ast u}_{W^{1,p}}\le C(\norm{Lu}_{L^{p}}+\norm{u}_{L^{p}}).
     \end{equation*}
     Therefore $\rho_{\delta}\ast u$ is uniformly bounded in
     $W^{1,p}_{0}$. It follows (by reflexivity of $W^{1,p}_{0}$) that
     $\rho_{\delta}\ast u$ converges weakly in $W^{1,p}_{0}$ (after
     taking a subsequence). Since we know that $\rho_{\delta}\ast u$
     converges to $u$ in $L^{p}$, we conclude that $u$ must be equal
     to the weak limit of $\rho_{\delta}\ast u$ in $W^{1,p}_{0}$, and
     hence $u$ is in $W^{1,p}_{0}$, as desired.

     More generally, suppose that $u$ is defined on an arbitrary
     compact manifold with boundary~$Y$. Since the support of $u$ is
     separated from the boundary, we can find balls
     $$B_{1}(1),\cdots,B_{n}(1)\subset \text{int}(Y)$$ so that the balls
     $B_{1}(r),\cdots,B_{n}(r)$ cover the support of $u$. Picking a
     partition of unity $\rho_{1},\cdots,\rho_{n}$ so $\rho_{i}$ is
     supported in $B_{i}(r)$, conclude that
     \begin{equation*}
       L\rho_{i}u=\rho_{i}Lu+[L,\rho_{i}]u\in L^{p}\implies
       \rho_{i}u\in W^{1,p}_{0}\implies u\in W^{1,p}_{0}.
     \end{equation*}     
     This proves the theorem.     
   \end{proof}
   \subsection*{Differentiating the equation}
   The goal of this section will be to prove the following
   result
   \begin{thm}[interior bootstrapping]
     Let $Y$ be a compact manifold with boundary, and $L:E_{1}\to
     E_{2}$ a first order elliptic operator. Then for sections $u$
     supported away from $\bd Y$ we have
     \begin{equation*}
       u\in L^{p}(Y)\text{ and }Lu\in W^{k,p}_{0}(Y)\implies u\in W^{k+1,p}_{0}(Y).
     \end{equation*}
   \end{thm}
   
   The key idea needed to prove this is described in the following exercise:
   \begin{xca}
     If $L$ is elliptic, then there exists an elliptic first order operator $$L':\Hom(TM,E_{1})\to
     \Hom(TM,E_{2})$$ so that
     \begin{equation*}
       \nabla L(u)=L'(\nabla u)+\text{first order operator}.
     \end{equation*}
     Hint: writing $L(u)=A(\nabla u)+B(u)$, observe that
     \begin{equation*}
       \nabla L(u)=A(\nabla \nabla u)+B(\nabla u)+\nabla A (\nabla
       u)+\nabla B(u),
     \end{equation*}
     therefore we can define
     \begin{equation*}
       L'(\Phi)=A(\nabla \Phi)+B(\Phi).
     \end{equation*}
     We claim that $L'$ is still elliptic. To see why, compute
     \begin{equation*}
       [L',f](\Phi)=A(\d f \ocross \Phi),
     \end{equation*}
     and since $A(\xi\ocross -):E_{1}\to E_{2}$ is an isomorphism if
     $\xi\ne 0$, we conclude the induced map on $\Hom(TM,E_{1})\to
     \Hom(TM,E_{2})$ is still an isomorphism.

     We adopt the convention that we have chosen
     extensions of $L$ to elliptic operators $$\Hom(TM^{\ocross
       n},E_{1})\to \Hom(TM^{\ocross n}E_{2})$$ so that $[L,\nabla]$ is a
     first order operator. 
   \end{xca}
   \begin{proof}[interior bootstrapping]
     We will prove that
     \begin{equation*}\tag{$\ast$}
       u\text{ supported away from $\bd Y$ and $u,Lu\in
         W^{k,p}_{0}$}\implies\text{ $u\in W^{k+1,p}_{0}$}.
     \end{equation*}
     by induction. We already know it for $k=0$. Now suppose that
     $u,Lu\in W^{k,p}_{0}$. Then $\nabla u\in W^{k-1,p}_{0}$, and
     \begin{equation*}
       L'\nabla u=\nabla L u+[L',\nabla]u\in W^{k-1,p}_{0}.
     \end{equation*}
     Since $L'$ is elliptic, and we suppose that we know the $k-1$
     case of ($\ast$) for all elliptic operators (in particular, for $L'$), we conclude that
     \begin{equation*}
       \nabla u\in W^{k,p}_{0},
     \end{equation*}
     and hence $u\in W^{k+1,p}_{0}$, as desired. This proves ($\ast$)
     for all $k$. It is clear that knowing $(\ast)$ for all $k$
     implies
     \begin{equation*}
       u\in L^{p}(Y)\text{ and }Lu\in W^{k,p}_{0}(Y)\implies u\in W^{k+1,p}_{0}(Y).
     \end{equation*}
     for sections $u$
     supported away from $\bd Y$. This completes the proof.
   \end{proof}

   Using the same ``differentiating the equation'' trick, we can
   upgrade the interior elliptic estimates:
   \begin{thm}[interior elliptic estimates, for all $k$]
     Let $L$ be an elliptic differential operator. There is a constant
     $C_{k}$ (depending on $L$ and $k$) so that for all
     $u\in W^{k+1,p}_{0}(Y)$ we have
     \begin{equation*}\tag{$\ast$}
       \norm{u}_{W^{k+1,p}}\le C_{k}(\norm{Lu}_{W^{k,p}}+\norm{u}_{L^{p}}).
     \end{equation*}
   \end{thm}
   \begin{proof}
     The case when $k=0$ is already known. Suppose we have found
     constants $C_{i}$ so that ($\ast$) holds for $i<k$.

     Let $u\in W^{k+1,p}_{0}(Y)$. Then $\nabla u\in W^{k,p}_{0}(Y)$,
     and hence we can apply the $k-1$ elliptic estimate to $\nabla u$
     \begin{equation*}
       \norm{\nabla u}_{W^{k,p}}\le C_{k-1}(\norm{L'\nabla
         u}_{W^{k-1,p}}+\norm{\nabla u}_{L^{p}}).
     \end{equation*}
     It follows that
     \begin{equation*}
       \begin{aligned}
       \norm{\nabla u}_{W^{k,p}}&\le C_{k-1}(\norm{\nabla
         (Lu)}_{W^{k-1,p}}+\norm{[L,\nabla]u}_{W^{k-1,p}}+\norm{\nabla
         u}_{L^{p}})\\
       &\le
       C'(\norm{Lu}_{W^{k,p}}+\norm{u}_{W^{k,p}}+\norm{u}_{W^{1,p}})\\
       &\le C_{k}(\norm{Lu}_{W^{k,p}}+\norm{u}_{L^{p}}),
     \end{aligned}     
   \end{equation*}
   where we have used the fact that $[L,\nabla]$ is first order
   operator, and then used the $0,k-1$ cases of the elliptic
   estimates. It is clear that constant $C_{k}$ does not depend on the
   $u$ we started with. This completes the proof.   
 \end{proof}
 \subsection*{Local elliptic regularity}
 \begin{defn}
   Let $Y$ be a potentially non-compact manifold. We define
   \begin{equation*}\tag{$\ast$}
     W^{k,p}_{\text{loc}}(Y)=\lim W^{k,p}(Z)\text{ where $Z$ compactly
       supported in $\text{int}(Y)$}.
   \end{equation*}
   Note that the set $W^{k,p}_{\text{loc}}(Y)$ can be thought of as
   a subset of distributions $C^{\infty}_{c}(Y,E_{1})\to \R$ (or
   $\C$). The definition in $(\ast)$ also tells us the topology of
   $W^{k,p}_{\text{loc}}(Y)$.
 \end{defn}
 \begin{thm}\label{thm:localellipticregularity}
   Let $L$ be an elliptic first order operator on $Y$ (potentially
   non-compact). Then
   \begin{equation*}\tag{1}
     u\in L^{p}_{\text{loc}}(Y)\text{ and }Lu\in
     W^{k,p}_{\text{loc}}(Y)\iff u\in W^{k+1,p}_{\text{loc}}(Y)
   \end{equation*}
   and the map
   \begin{equation*}\tag{2}
     u\in W^{k+1,p}_{\text{loc}}(Y)\to u\cross Lu\in
     L^{p}_{\text{loc}}(Y)\cross W^{k,p}_{\text{loc}}(Y),
   \end{equation*}
   is a closed embedding.
 \end{thm}
 \begin{clear}{Remark}
   The fact that (2) defines a closed embedding is closely related to
   the interior elliptic estimates.
 \end{clear}
 \begin{proof}
   The proof of (1) is quite easy, since we already know a similar
   statement in the interior case. Here is the argument: if $u\in
   L^{p}_{\text{loc}}$ and $Lu\in W^{k,p}_{\text{loc}}$, then for any
   test function $\rho$ we have
   \begin{equation*}\tag{$\ast$}
     \rho u\in L^{p}\text{ and }L\rho u=\rho Lu+[L,\rho]u\in L^{p},
   \end{equation*}
   (since $[L,\rho]$ is a tensor with compact support, $[L,\rho]u\in
   L^{p}$). By interior bootstrapping we conclude $\rho u\in
   W^{1,p}_{0}$. By taking a locally finite partition of unity, we
   conclude $u\in W^{1,p}_{\text{loc}}$. Now we repeat the argument,
   but now that we know that $u\in W^{1,p}_{\text{loc}}$, we can conclude
   that $L\rho u\in W^{1,p}_{0}$, and hence conclude $\rho u\in
   W^{2,p}_{0}$, etc. We conclude that (1) holds.

   To prove (2), we will use the interior elliptic estimates. It is
   easy to show that $W^{k,p}_{\text{loc}}$ is a metrizable space, and
   hence to prove that the map in (2) is a closed embedding, it
   suffices to consider sequences of elements. Let $u_{n}$ be a
   sequence in $W^{k+1,p}_{\text{loc}},$ and suppose that
   $u_{n}\to u_{\infty}$ in $L^{p}_{\text{loc}}$ and $Lu_{n}$
   converges in $W^{k,p}_{\text{loc}}$. Let $\rho$ be some bump
   function, say $\rho=1$ on the compact set $Z$. By the interior
   elliptic estimates we conclude that $\rho u_{n}$ is Cauchy in
   $W^{1,p}$, and hence $u_{n}$ converges in 
   $W^{1,p}_{\text{loc}}$. By repeating the argument, conclude that
   $\rho u_{n}$ is Cauchy in $W^{2,p}$, etc. Ultimately deduce that
   $u_{n}$ converges in $W^{k+1,p}_{\text{loc}}$. This completes the proof.         
 \end{proof}
 \begin{defn}
   We define the topology on the space $C^{\infty}$ by a limit 
   \begin{equation*}
     C^{\infty}_{\text{loc}}(Y)=\lim_{k\to\infty}C^{k}_{\text{loc}}(Y),
   \end{equation*}
   where $\text{loc}$ indicates that we are not thinking of $C^{k}$ as
   a normed space, but rather as the limit of the normed spaces
   $C^{k}(Z)$ as $Z$ ranges over compact subdomains of $Y$.

   An easy consequence of the Morrey embedding theorem, we conclude
   that
   \begin{equation*}\tag{$\ast$}
     C^{\infty}_{\text{loc}}(Y)\simeq \lim_{k}W^{k,p}_{\text{loc}}(Y).
   \end{equation*}
 \end{defn}
 \begin{thm}
   Let $Y$ be a manifold, and let $L$ be a first order elliptic
   operator on $Y$. Then the map
   \begin{equation*}
     u\in C^{\infty}_{\text{loc}}(Y)\mapsto u\cross Lu\in
     L^{p}_{\text{loc}}\cross C^{\infty}_{\text{loc}}(Y)
   \end{equation*}
   is a closed embedding. 
 \end{thm}
 \begin{proof}
   Let $u_{n}\in C^{\infty}_{\text{loc}}(Y)$ be a sequence so that
   $u_{n}$ converges in $L^{p}_{\text{loc}}$ and $Lu_{n}$ converges in
   $C^{\infty}_{\text{loc}}$.

   Clearly $Lu_{n}$ converges in $W^{k,p}_{\text{loc}}$, for every
   $k$, and hence $u_{n}$ converges in $W^{k,p}_{\text{loc}}$ for
   every $k$ by Theorem \ref{thm:localellipticregularity}. By
   ($\ast$), we conclude that $u_{n}$ converges in
   $C^{\infty}_{\text{loc}}$, as desired.
 \end{proof}
 \begin{example}
   Let $u_{n}$ be a sequence of holomorphic functions in $D(1)$ so
   that $u_{n}$ converges to $u_{\infty}$ in $L^{p}$. Then
   $u_{\infty}$ is $C^{\infty}$ smooth, holomorphic, and $u_{n}\to
   u_{\infty}$ converges in $C^{\infty}_{\text{loc}}$.

   To see that $u_{\infty}$ is $C^{\infty}$ smooth, use the fact that
   $u\in C^{\infty}_{\text{loc}}\mapsto u\cross \conj\bd u\in
   L^{p}\cross C^{\infty}_{\text{loc}}$ is a closed
   embedding. Therefore the sequence $u_{n}$ of holomorphic functions
   converges in $C^{\infty}_{\text{loc}}$. The limit must obviously
   equal $u$, and hence $u\in C^{\infty}_{\text{loc}}$ and
   $u_{n}\to u$ in $C^{\infty}_{\text{loc}}$. It is now clear that $u$
   is holomorphic.
 \end{example}
  \subsection*{Fredholm theory of elliptic operators on closed
   manifolds.}
 In this section we will work with $p=2$. It will simplify things without
 any loss of generality concerning the final statement we will prove.

 Consider an elliptic operator $L:E_{1}\to E_{2}$ on a closed manifold
 and its adjoint $$L^{*}:E_{2}\to E_{1},$$ and consider the spaces
 $W^{k,2}(E_{i})$, $k=0,1,2,\cdots.$%  We define $W^{\infty,2}=\lim
 % W^{k,2}$, so $W^{\infty,2}\simeq C^{\infty}$.
 
 Let's agree to write $L_{k}$ for the induced map $W^{k+1,2}\to
   W^{k,2}$.
 \begin{lemma}
   The map $L_{k}$ has closed range and the kernel of $L_{k}$ is finite dimensional.
 \end{lemma}
 \begin{proof}
   Since
   \begin{equation*}
     \norm{u}_{W^{k+1,2}}\le C_{k}(\norm{Lu}_{W^{k}}+\norm{u}_{L^{p}}),
   \end{equation*}
   we can easily deduce that $L_{k}$ has finite kernel and closed
   range. We leave the proof to the reader, with the hint that to
   prove first that $L$ has finite dimensional kernel, and then to
   consider $L$ restricted to a complement of the kernel.
 \end{proof}
 
 \begin{lemma}
   The cokernel of $L_{k}$ is
   canonically identified with $\Hom(\ker L^{*}_{0},\R)$ via the $L^{2}$
   inner product, and so the cokernel of $L_{k}$ is also finite
   dimensional. In other words:
   \begin{equation*}
     \im L_{k}=(\ker L^{*}_{0})^{\perp}\cap W^{k,2}.
   \end{equation*}
 \end{lemma}
 \begin{proof}
   It is clear that if $w=Lu$ for $u\in W^{k+1,2}$, then $w\in (\ker
   L^{*}_{0})^{\perp}$. The reverse inclusion is a bit deeper. Suppose
   that $w\in (\ker L^{*}_{0})^{\perp}\cap W^{k,2}$.

   First we will prove that $w\in \im L_{0}$ (i.e.\ there is a
   $W^{1,2}$ section $u$ so $L_{0}u=w$). 

   Let's suppose that $w\not\in \im L_{0}$ in search of a
   contradiction. Since $\im L_{0}$ has closed range in $L^{2}$, we
   can find a section $v$ so that $(w,v)\ne 0$, but $(\im
   L_{0},v)=0$. It follows that $L^{*}v=0$ weakly and hence by
   elliptic regularity, $v$ is of class $W^{1,2}$ and
   $v\in \ker L^{*}_{0}$, contradicting the fact that $w\in (\ker
   L_{0}^{*})^{\perp}$. Therefore $w=Lu$ for $u\in W^{1,2}$. By
   elliptic regularity, since $u\in L^{p}$ and $Lu\in W^{k,2}$, we
   conclude $u\in W^{k+1,2}$, and hence $w\in \im L_{k}$, as desired.   
 \end{proof}
 
 \begin{prop}
   For all $k$,
   \begin{equation*}
     \ker L_{k}=\set{u\in C^{\infty}:Lu=0}=\set{u\in L^{2}:Lu=0\text{ weakly}}.
   \end{equation*}
 \end{prop}
 \begin{proof}
   It is clear that
   \begin{equation*}
     \set{u\in C^{\infty}:Lu=0}\subset \ker L_{k}\subset \set{u\in L^{2}:Lu=0\text{ weakly}}.
   \end{equation*}
   Therefore it suffices to show that
   \begin{equation*}
     \set{u\in L^{2}:Lu=0\text{ weakly}}\subset \set{u\in C^{\infty}:Lu=0}.
   \end{equation*}
   This is a straightforward consequence of elliptic regularity:
   \begin{equation*}
     u\in L^{2}_{\text{loc}}\text{ and }Lu\in
     C^{\infty}_{\text{loc}}\implies u\in C^{\infty}_{\text{loc}}.
   \end{equation*}
   This completes the proof.
 \end{proof}
 \begin{cor}
   Let $L$ be a first order elliptic operator on a closed manifold
   $Y$. There are canonical $L^{2}$-orthogonal decompositions
   \begin{equation*}
     W^{k,2}(E_{2})=\im L_{k}\oplus \ker L^{*}\text{ and
     }W^{k,2}(E_{1})=\im L^{*}_{k}\oplus \ker L,
   \end{equation*}
   including the case when $k=\infty$ (where $W^{\infty,2}=C^{\infty}$).   
 \end{cor}
 \begin{proof}
   We already know that $\ker L^{*}\subset W^{k,2}(E_{2})$ and $\ker
   L\subset W^{k,2}(E_{1})$ for all $k$. To prove that
   \begin{equation*}
     W^{k,2}(E_{2})=\im L_{k}\oplus \ker L^{*},
   \end{equation*}
   it suffices to observe that if $u\in W^{k,2}(E_{2})$ is orthogonal
   to $\ker L^{*}$, then $u\in \im L_{k}$.

   Now when $k=\infty$, we once again notice that it suffices to prove
   that if $u\in C^{\infty}(E_{2})$, and $u\perp \ker L^{*}$, then
   $u\in \im L$. This is easy, since we know that $u=Lw$ for some
   $w\in W^{1,2}$, and hence by regularity $w$ is actually in
   $C^{\infty}(E_{2})$, as desired. This completes the proof.
 \end{proof}
 We may care about similar decompositions of $W^{k,p}$ in the case
 when $p\ne 2$. A very similar argument proves the following:
 \begin{thm}
   Consider the split injections
   \begin{equation*}
     \ker L\to W^{k,p}(E_{1})\text{ and }\ker L^{*}\to W^{k,p}(E_{2}),
   \end{equation*}
   where the splittings are given by the $L^{2}$ orthogonal
   projections, i.e.
   \begin{equation*}
     u\mapsto (u,v_{1})v_{1}+\cdots+(u,v_{n})v_{n}.
   \end{equation*}
   These splittings are obviously continuous in the $W^{k,p}$
   topology. With respect to these splittings, we have
   \begin{equation*}
     W^{k,p}(E_{2})=\im L_{k}\oplus \ker L^{*}\text{ and }W^{k,p}(E_{1})=\im L_{k}^{*}\oplus \ker L.
   \end{equation*}
 \end{thm}

 
 \begin{example}
   Consider $L=\conj\bd$ on $\T^{2}$. As we have seen, this is an
   elliptic operator, and $L^{*}=-\bd$. It is easy to show using
   Fourier coefficients that
   \begin{equation*}
     \ker L=\ker L^{*}=\C\text{ (the constants)}.
   \end{equation*}
   As corollary of the preceding decomposition theorem, we conclude
   that for $u\in W^{k,p}$
   \begin{equation*}
     \text{if $\int u\,\dVol=0$ then $u=\conj\bd w$ for unique $W^{k+1,p}$ satisfying
       $\int w\, \dVol=0$}.
   \end{equation*}
   
   As a consequence, we can prove the following local surjectivity
   result: the map of sheaves
   \begin{equation*}
     \conj\bd:W^{k+1,p}_{\text{loc}}\to W^{k,p}_{\text{loc}}.
   \end{equation*}
   is locally surjective (on any Riemann surface).

   \begin{proof}
     It suffices to prove that for $w\in W^{k,p}_{\text{loc}}(D(1))$,
     we can find $u\in W^{k+1,p}_{\text{loc}}(D(r))$ so $\conj\bd u=w$
     on $D(r)$. Let $\rho$ be a bump function equal to $1$ on a
     neighborhood of $D(r)$ and which is supported in $D(1)$. Then
     $\rho u\in W^{k,p}_{0}(D(1))$. By embedding $D(1)$ in the torus,
     it follows that there is a $W^{k+1,p}(\T^{2})$ section $v$ so
     that
     \begin{equation*}
       \conj\bd v=\rho u+c
     \end{equation*}
     for some constant $c$. Let $\varphi(z)=cz$ on a neighborhood of
     $D(r)$. Then
     \begin{equation*}
       \conj\bd(v-\rho \varphi)=u\text{ on $D(r)$},
     \end{equation*}
     as desired.    
   \end{proof}  
 \end{example}
  \begin{example}[the parametrix]
   Consider the splitting
   \begin{equation*}
     W^{k,p}(E_{2})=\im L_{k}\oplus \ker L^{*}.
   \end{equation*}
   Let's define $p$ to be the projection onto $\ker L^{*}$, and $q$
   to be the projection onto $\ker L$, then
   \begin{equation*}
     P_{k}=(1-q)\circ L_{k}^{-1}\circ (1-p)
   \end{equation*}
   is well-defined: i.e.\ there is a unique element in
   $u\in W^{k+1,p}(E_{2})$ so that $u\perp \ker L$ and $L_{k}u=(1-p)w$
   for each $w\in W^{k,p}(E_{2})$. The open mapping theorem guarantees
   that $P_{k}$ is continuous, i.e.\ $L_{k}^{-1}$ is continuous on
   $(\ker L^{*})^{\perp}\to (\ker L)^{\perp}$.

   It is clear, by uniqueness, that $P_{k}=P_{k+1}$ on $W^{k+1,p}\subset
   W^{k,p}$. Therefore we obtain a well-defined continuous map
   $P:C^{\infty}\to C^{\infty}$ which equals $P_{k}$ on
   $W^{k,p}$. Note that
   \begin{equation*}
     \begin{aligned}
       1-PL&=\text{ projection onto $\ker L$}\\
       1-LP&=\text{ projection onto $\ker L^{*}$}.
     \end{aligned}
   \end{equation*}
 \end{example}
 \subsection*{Homotopy invariance of the Fredholm index.}
 Let $L:E_{1}\to E_{2}$ be an elliptic operator on a closed manifold
 $Y$. Define the index of $L$ to be $\dim \ker L-\dim \ker L^{*}$.

 The goal of this section will be to show that the index is
 constant on a continuously varying family of elliptic operators.

 More generally, if $X,Y$ are any Banach spaces, consider the space
 $\mathscr{F}(X,Y)$ of operators $L:X\to Y$, equipped with the distance
 induced by the operator norm
 \begin{equation*}
   d(L,L')=\norm{L-L'}
 \end{equation*}
 and which satisfy the Fredholm condition:
 \begin{equation*}
   L\text{ has finite dimensional kernel, closed image, and finite dimensional cokernel}.
 \end{equation*}
 \begin{thm}
   The index function $\mathscr{F}(X,Y)\to \Z$ is continuous. 
 \end{thm}
 There is a slight trick to the proof. It is useful to introduce one
 concept related to Fredholm maps.
 \begin{defn}
   Let $L:X\to Y$ be a bounded linear operator. A
   \textbf{regularization} of $L$ is the data of two finite
   dimensional vector spaces and maps $a,b,c$ so that
   \begin{equation*}\tag{$\ast$}
    \begin{dmatrix}
       L&a\\
       b&c
     \end{dmatrix}:X\oplus V\to Y\oplus W
   \end{equation*}
   is an isomorphism.
 \end{defn}
 \begin{prop}
   A map $L:X\to Y$ is Fredholm if and only if it admits a regularization.
 \end{prop}
 \begin{proof}
   Suppose $L$ is Fredholm. Take $V=\coker L$, $W=\ker L$. Choosing a
   splitting $j$ of $\pi:Y\to \coker L$ and a splitting $p$ of $i:\ker
   L\to X$ gives a map
   \begin{equation*}
     \begin{dmatrix}
       L&j\\
       p&0
     \end{dmatrix}
     :X\oplus \coker L\to Y\oplus \ker L
   \end{equation*}
   which the reader can check is bijective, and hence is an
   isomorphism.

   Conversely, suppose that $L$ admits a regularization of the form
   ($\ast$). Since ($\ast$) is an isomorphism, the map $x\in X\mapsto
   Lx\cross bx\in Y\cross W$ is a closed embedding, and hence we
   conclude a bound
   \begin{equation*}\tag{$1$}
     \norm{x}_{X}\le C(\norm{Lx}_{Y}+\norm{bx}_{W}).
   \end{equation*}
   Since $x\mapsto bx$ is a finite rank operator (in particular, is
   compact), ($1$) implies that $L$ has finite dimensional kernel
   and closed image.

   We will prove that $\coker L$ is finite dimensional arguing by
   contradiction. Suppose that $\coker L$ is infinite
   dimensional. Then we can find unit vectors $y_{1},y_{2},y_{3},\cdots$ in
   $Y$ so that $$\norm{y_{i}-\im
     L-\text{span}(y_{1},\cdots,y_{i-1})}>1/2.$$

   Since $(\ast)$ is an isomorphism, we can find vectors $x_{i},v_{i}$
   so $\norm{x_{i}}+\norm{v_{i}}$ is bounded so that
   \begin{equation*}
     Lx_{i}+a(v_{i})=y_{i}\implies \norm{a(v_{i})-a(v_{j})}>1/2\text{
       for all $i\ne j$},
   \end{equation*}
   in particular
   \begin{equation*}\tag{2}
     \norm{a}\norm{v_{i}-v_{j}}>1/2\text{ for all $i\ne j$}.
   \end{equation*}
   Since any bounded ball in $V$ is precompact, a subsequence of
   $v_{1},v_{2},\cdots$ must converge, contradicting (2). Therefore
   $\coker L$ is finite dimensional.   
 \end{proof}
 \begin{prop}
   Suppose that $L:X\to Y$ admits a regularization $a,b,c,V,W$. There
   is $\epsilon>0$ so that any map $L'$ within $\epsilon$ of $L$
   is also regularized by $a,b,c,V,W$.
 \end{prop}
 \begin{proof}
   This is easy, since we know that being an isomorphism is open
   condition in the space of maps, so that there is $\epsilon$ so that
   \begin{equation*}
     \norm{\Lambda-
       \begin{dmatrix}
         L&a\\
         b&c
       \end{dmatrix}
}<\epsilon\implies \Lambda\text{ is an isomorphism}.
\end{equation*}
In particular, if $\norm{L'-L}<\epsilon$, then 
\begin{equation*}
  \norm{\begin{dmatrix}
         L'&a\\
         b&c
       \end{dmatrix}-\begin{dmatrix}
         L&a\\
         b&c
       \end{dmatrix}}<\epsilon\implies \begin{dmatrix}
         L'&a\\
         b&c
       \end{dmatrix}\text{ is an isomorphism}.
     \end{equation*}
     As a corollary, the space of Fredholm maps $\mathscr{F}(X,Y)$ is
     open in the operator norm.
   \end{proof}
   \begin{prop}
     Let $L:X\to Y$ be a Fredholm operator and suppose
     $a,b,c,V,W$ is a regularization for $L$. Then
     \begin{equation*}
       \dim \ker L-\dim \coker L=\dim W-\dim V.
     \end{equation*}     
   \end{prop}
   \begin{proof}
     Since $X$ is Fredholm, we can split $X,Y$ as follows
     \begin{equation*}
       X=X'\oplus \ker L\text{ and }Y=\im L\oplus \coker L,
     \end{equation*}
     where $X'$ is complimentary to $\ker L$, and so that
     $L'=L|_{X'}:X'\to \im L$ is an isomorphism. Then the matrix of
     the regularization takes the form
     \begin{equation*}
       R=
       \begin{dmatrix}
         L'&0&a_{1}\\
         0&0&a_{2}\\
         b_{1}&b_{2}&c
       \end{dmatrix}:X'\oplus \ker L\oplus V\to \im L\oplus \coker
       L\oplus W
     \end{equation*}
     We will construct an isomorphism $\varphi:\ker L\oplus V\to \coker
     L\oplus W$. We begin by computing
     \begin{equation*}
              \begin{dmatrix}
         L'&0&a_{1}\\
         0&0&a_{2}\\
         b_{1}&b_{2}&c
       \end{dmatrix}
       \begin{dmatrix}
         x\\k\\v
       \end{dmatrix}
=       \begin{dmatrix}
         L'x+a_{1}(v)\\0\\0
       \end{dmatrix}+
       \begin{dmatrix}
         0\\
         \alpha(x,k,v)\\
         \beta(x,k,v)
       \end{dmatrix}.
     \end{equation*}
     Motivated by this computation, define $x(v)\in X'$ \emph{uniquely} by the
     equation $L'x(v)+a_{1}(v)=0$. Then define $\varphi$ by
     \begin{equation*}
       \varphi
       \begin{dmatrix}
         k\\v
       \end{dmatrix}
       =R
       \begin{dmatrix}
         x(v)\\k\\v
       \end{dmatrix}=\begin{dmatrix}
         0\\
         \alpha(x(v),k,v)\\
         \beta(x(v),k,v)
         \end{dmatrix}
       \end{equation*}    
     We claim that $\varphi$ is a bijection. Injectivity is easy: if
     $\varphi(k,v)=0$, then
     \begin{equation*}
       R
       \begin{dmatrix}
         x(v)\\k\\v
       \end{dmatrix}=0\implies
       \begin{dmatrix}
         k\\v
       \end{dmatrix}=0.
     \end{equation*}
     Conversely, since $R$ is surjective, there is some $x,k,v$ so
     that
     \begin{equation*}
       R \begin{dmatrix}
         x\\k\\v
       \end{dmatrix}=
       \begin{dmatrix}
         0\\a\\b
       \end{dmatrix}\implies x=x(v)\implies \varphi
       \begin{dmatrix}
         k\\v
       \end{dmatrix}=
       \begin{dmatrix}
         a\\b
       \end{dmatrix}.
     \end{equation*}
     Therefore $\varphi$ is an isomorphism $\ker L\oplus V\to \coker
     L\oplus W$. It follows that
     \begin{equation*}
       \dim \ker L+\dim V=\dim \coker L+\dim W\implies \dim \ker
       L-\dim \coker L=\dim W-\dim V,
     \end{equation*}
     as desired.
   \end{proof}
   \begin{thm}[homotopy invariance of the Fredholm index]
     Let $X,Y$ be Banach spaces and consider the space
     $\mathscr{F}(X,Y)$ of Fredholm
     maps $X\to Y$. Then the index function $\mathscr{F}(X,Y)\to \Z$
     defined by
     \begin{equation*}
       \text{index}(L)=\dim \ker L-\dim \coker L
     \end{equation*}
     is continuous.
   \end{thm}
   \begin{proof}
     If $L$ has index $n$, then we can regularize $L$ to an isomorphism
     \begin{equation*}
       \begin{dmatrix}
         L&a\\
         b&c
       \end{dmatrix}:X\oplus V\to Y\oplus W.
     \end{equation*}
     In particular
     \begin{equation*}\tag{1}
       \text{index}(L)=\dim W-\dim V
     \end{equation*}
     
     For $L'$ sufficiently close to $L$, we know that
     \begin{equation*}
       \begin{dmatrix}
         L'&a\\
         b&c
       \end{dmatrix}:X\oplus V\to Y\oplus W.
     \end{equation*}
     is still an isomorphism. It follows that
     \begin{equation*}\tag{2}
       \text{index}(L')=\dim W-\dim V
     \end{equation*}
     Comparing (1) and (2), we see that
     $\text{index}(L)=\text{index}(L')$, as desired.
   \end{proof}
   \begin{cor}
     Let $L_{t}$ be a continuous family of Elliptic operators on a
     closed manifold $Y$. It is easy to show that $L_{t}$ is
     continuous in the operator norm topology on $W^{1,p}(Y)\to
     L^{p}(Y)$. It follows that
     \begin{equation*}
       \text{index}(L_{1})=\text{index}(L_{0}).
     \end{equation*}
   \end{cor}   
 \end{document}
