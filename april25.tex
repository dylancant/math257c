\documentclass{amsart}
\usepackage{preamble}

\begin{document}
\section*{\textbf{April 25 (Dylan/Lie)}}

Back to Morse theory to complete the analytic program. Recall the
outline we intend to follow. The reader is encouraged to read the
first section of Schwarz's book for more details on this outline.

\begin{enumerate}
\item Functional setup.
\item Fredholm/index theory.
\item Transversality.
\item Compactness.
\item Gluing
\item Orientation (not today).
\end{enumerate}

Let's recall the setting: suppose that $M$ is a closed manifold, $f$
is a Morse function, and $g$ is a Riemannian metric on $M$. The last
time we were considering this setup, we proved the ``local stable
manifold theorem.'' In this lecture, we intend to prove a more
``global'' theorem -- we will explore the moduli space of flow lines
joining two critical points of $f$ in $M$. 

Let $\grad$ denote the gradient with respect to $g$ of the Morse
function $f$. A \textbf{flow line} $\gamma:\R\to M$ is a $C^{\infty}$
function satisfying
\[
  \gamma'+\grad_f\circ \gamma=0.
\]
If $x,y$ are critical points of $f$, and $\gamma$ is a flow line
satisfying the asymptotic conditions $\gamma(-\infty)=x$,
$\gamma(\infty)=y$, then we call $\gamma$ a \textbf{connecting trajectory}.

Define the moduli space of (parametrized) flow lines joining $x$ to $y$ $$\M(x,y)=\{\text{connecting trajectories between x and y}\}.$$ This space $\M$
is what we will study using analysis techniques. Unfortunately,
because translation reparametrization preserves connecting
trajectories, $\M(x,y)$ is infinite (if it is non-empty), so
ultimately we will want to consider the quotient
\begin{equation*}
  \hat{\mathscr{M}}=\mathscr{M}/\R.
\end{equation*}
These moduli spaces feature in the definition of the ``Morese
complex;'' one can define a chain complex $(CM_\ast,\bd)$ (over
$\Z/2\Z$), which is freely generated in each degree
$$CM_{i}:=\Z/2\Z\,\langle \text{\text{critical points of Morse index
  i}}\rangle,$$ and whose boundary map $\bd$ is defined by ``counting flow lines''
$$\bd:CM_{i+1}\to CM_i:\langle \bd x,y\rangle=\#\hat{\M}(x,y).$$

\begin{clear}{Remark}
Without continuation maps, homotopies between continuation maps etc,
the theory is incomplete. We won't consider these ideas yet -- for now
we are preoccupied with establishing foundations for the construction
of the prerequisite moduli spaces.
\end{clear}

\begin{clear}{Strategy for construction of $\mathscr{M}(x,y)$}  
The idea is to construct a Banach manifold $\mathscr{P}$ so that

$$C^{\infty}_{x,y}(\closure{\R},M)\subset\mathscr{P}\subset
C^{0}_{x,y}(\closure{\R},M)$$ (where $\closure{\R}\simeq [-1,1]$ via
$t\mapsto \frac{2}{\pi}\arctan(t)$ is the two point
compactification of $\R$ with smooth structure inherited from
$[-1,1]$) with the following properties:

(i) Every gradient flow line joining $x$ to $y$ lies in $\mathscr{P}$.

(ii) The section
\begin{equation*}
  s:\gamma \mapsto \bd_{t}\gamma+\grad\circ \gamma
\end{equation*}
extends a smooth section of a Banach bundle $\mathscr{E}\to
\mathscr{P}$, and $s^{-1}(0)$ consists precisely of the gradient flow
lines joining $x$ to $y$, i.e.\ $s^{-1}(0)=\mathscr{M}(x,y)$.

(iii) The linearization of $s$ at $s^{-1}(0)$ is Fredholm with index
equal to $\text{index}(x)-\text{index}(y)$.

(iv) Observe that the section $s$ is actually a family of sections
parametrized by the underlying Riemannian metric $g$. It can be shown that
for generic metric $g$, the section $s:\mathscr{P}\to \mathscr{E}$ is
transverse to the zero section, and hence $s^{-1}(0)$ is a smooth
submanifold of $\mathscr{P}$ with dimension $\text{index}(x)-\text{index}(y)$.\hfill$\lrcorner$
\end{clear}

To motivate the choice/construction of the path space $\mathscr{P}$,
we recall the key analytical result which will enable us to establish
(iii).
\begin{thm}\label{thm:a251}
  Let $A:\R\to \R^{n\cross n}$ be a family of matrices so that
  $A(-\infty)$ and $A(+\infty)$ converge, and $A(-\infty)$,
  $A(+\infty)$ have no imaginary eigenvalues. Then the differential
  operator
  \begin{equation*}\tag{$\ast$}
    L=\bd_{t}+A(t)
  \end{equation*}
  induces a Fredholm operator
  \begin{equation*}
    L:W^{1,p}(\R,\R^{n})\to L^{p}(\R,\R^{n}),
  \end{equation*}
  whose index equals the spectral flow of the operators $A(t)$.\hfill $\square$
\end{thm}
The key here is that this result is stated for the spaces
$W^{1,p}(\R,\R^{n})$ and \emph{so we want the tangent space
  $T\mathscr{P}_{\gamma}$ to be isomorphic to
  $W^{1,p}(\R,\R^{n})$}, because, as we will see later, the linearization of
$\bd_{t}+\text{grad}$ at a gradient flow line is of the form
($\ast$).

\begin{clear}{Some ideas used in the construction of $\mathscr{P}$}
  When discussing ``Banach manifolds of maps,'' Schwarz's book
  (and Wendl's notes on Holomorphic curves) reference the 1967 paper
  ``Geometry of Manifolds of Maps'' by Halldor Eliasson. The techincal
  tool Eliasson introduces is the notion of a \textbf{Manifold model}
  which should be thought as the necessary data needed to construct a
  Banach manifold of maps.

  The definition of ``Manifold model'' we will give below attempts to
  incorporate asymptotic boundary conditions in some systematic
  way. For the purposes of constructing a \emph{path space} we should
  take $\closure{X}=\closure{\R}$.

  \begin{defn}
  Let $\closure{X}$ be a compact manifold with boundary and let $X\subset \closure{X}$ be an open set. Let's suppose that $\closure{X}=X\cup \bd X$, for a closed submanifold $\bd X$. Let $C^{\infty}_{0}(\closure{X})$ be the set of continuous functions vanishing on $\bd X$. 

  A \textbf{manifold model} for $(X,\closure{X})$ is a functor $W:\text{Bun}(\closure{X})\to \text{Ban}$ and the data of two natural continuous dense inclusions
  \begin{equation*}
    C^{\infty}_{0}(\closure{X},E)\to W(E)\to C^{0}_{0}(\closure{X},E),
  \end{equation*}
  where $C^{k}_{0}$ are functions vanishing on $\bd X$. 

  (Banach Algebra) We require the continuous multiplication maps
  \begin{equation*}
    C^{\infty}_{0}(\closure{X},E_{1})\ocross C^{\infty}(\closure{X},\Hom(E_{1},E_{2}))\to C^{\infty}_{0}(\closure{X},E_{2})
  \end{equation*}
  (and similarly with $\infty\mapsto 0$) to induce continuous multiplication maps
  \begin{equation*}
    W(E_{1})\ocross W(\Hom(E_{1},E_{2}))\to W(E_{2})\text{ and }W(E_{1})\ocross C^{\infty}(\Hom(E_{1},E_{2}))\to W(E_{2})
  \end{equation*}

  (Composition is continuous) Because of the inclusion into $C^{0}_{0}(U)$, the set $W(U)$ is well-defined and open for any precompact $U\subset E$. Assuming that $C^{0}_{0}(U)$ is non-empty implies that $U$ contains a compactly supported smooth section. 

  We require that there are continuous composition maps $f\circ :W(U)\to W(E_{2})$ for every smooth fiber preserving map $\closure{U}\to E_{2}$ which vanishes on $0\cap \bd X$ (where $0$ is the zero section) making the following diagram commute
  \begin{equation*}
    \begin{tikzcd}
      C^{\infty}_{0}(\closure{X},U)\arrow[r]\arrow[d,"f\circ"]& W(U)\arrow[r]\arrow[d,"f\circ"]& C^{0}_{0}(\closure{X},U)\arrow[d,"f\circ "]\\
      C^{\infty}_{0}(\closure{X},E_{2})\arrow[r]& W(E_{2})\arrow[r]& C^{0}_{0}(\closure{X},E_{2}).
    \end{tikzcd}
  \end{equation*}
\end{defn}
\begin{defn}
  Let $M$ be a smooth manifold with Riemannian metric $g$ and any fiberwise convex open neighborhood $\mathscr{D}\subset TM$ of the zero section on which $\exp:TM\to M\cross M$ is an open embedding. For each smooth map $\gamma:\closure{X}\to M$ we can form the pullback $\gamma^{*}\mathscr{D}\subset \closure{X}\cross TM$, which admits an open embedding
  \begin{equation*}
    \exp_{\gamma}:\gamma^{*}\mathscr{D}\to \closure{X}\cross M,
  \end{equation*}
  obtained by applying $\exp:TM\to M$ to the definition of $\gamma^{*}\mathscr{D}$. We can think of $\exp_{\gamma}$ as a fiber preserving smooth map which sends $0\cap \bd X$ to $\gamma(\bd X)$. Fixing $\gamma|_{\bd X}=b$, we obtain a well-defined inclusion
  \begin{equation*}
    \exp_{\gamma}\circ:W(\gamma^{*}\mathscr{D})\to C^{0}_{b}(\closure{X},M),
  \end{equation*}
  where $C^{k}_{b}$ denotes functions which agree with $b$ on $\bd
  X$.
  
  We define
  \begin{equation*}
    W(\closure{X},M)=\bigcup \set{W(\gamma^{*}\mathscr{D}):\gamma\in C^{\infty}_{b}(\closure{X},M)}.
  \end{equation*}
  This set of maps does not depend on the Riemannian metric or the set $\mathscr{D}$.
\end{defn}
\begin{thm}
  For every manifold model $W$, and every smooth map $b:\bd X\to M$, there is a Banach manifold structure on $W(\closure{X},M)$ equipped with dense inclusions
  \begin{equation*}
    C^{\infty}_{b}(\closure{X},M)\subset W(\closure{X},M)\subset C^{0}_{b}(\closure{X},M)
  \end{equation*}
  and so that for any $\gamma\in C^{\infty}_{b}(\closure{X},M)$, and any choice of Riemannian metric $g$ or open set $\mathscr{D}$ the map
  \begin{equation*}
    \exp_{\gamma}:W(\gamma^{*}\mathscr{D})\to W(\closure{X},M)
  \end{equation*}
  is a smooth open embedding. 
\end{thm}
\begin{clear}{Remark}
  It follows that we can fix any $g,\mathscr{D}$ and obtain an atlas of smooth coordinate charts on $W(\closure{X},M)$. Indeed, if one knows that $C^{\infty}(\closure{X},M)$ has a countable subset dense in the $C^{0}$, one concludes that $W(\closure{X},M)$ has a countable atlas. 
\end{clear}
% \begin{xca}
%   Show that $C^{\infty}_{b}(\closure{X},M)$ has a countable subset which is dense in $C^{0}$. Hint: use Arz\'ela-Ascoli and a compact exhaustion of $M$. More Hints: fix a sequence $\epsilon_{n}\to 0$ and define $C^{\infty,\epsilon_{n}}_{b}(\closure{X},K)$ to be the maps whose first derivatives are bounded above by $\epsilon_{n}$. This set has compact closure in $C^{0}_{b}(\closure{X},M)$, and hence we can find a countable dense subset for $C^{\infty,\epsilon_{n}}_{b}(\closure{X},K)$ in $C^{0}$. By taking the union of all these sets, over all $K$ and $\epsilon_{n}$, we obtain a countable dense subset for all of $C^{0}$.
% \end{xca}
% \begin{xca}
%   Prove that any manifold model $W$ is seperable. Hint: it suffices to prove that
%   \begin{equation*}
%     C^{\infty}_{0}(\closure{X},E)\text{ has a countable subset dense in }C^{0}_{0}(\closure{X},E).
%   \end{equation*}
%   For this, use Arz\'ela Ascoli (as above) and a compact exhaustion of $E$ by disk bundles.
% \end{xca}
% \begin{clear}{Remark}
%   The preceding discussion and exercises establish that
%   $W(\closure{X},M)$ has a countable atlas by open subsets of Banach
%   spaces, which is important because it enables us to use Sard's
%   theorem. The seperability of the local charts is not strictly
%   necessary to apply Sard's theorem, but it is still good to know.
% \end{clear}
\end{clear}
\begin{clear}{The manifold model Schwarz uses to construct his path
    space $\mathscr{P}$}
  As we mentioned above, we really want the tangent spaces to our
  path space to be the $W^{1,p}(\R,\gamma^{*}TM)$ spaces (the
  $W^{1,2}$ spaces suffice), because it is for these Banach spaces
  that Theorem \ref{thm:a251} applies.

  % Here is a detailed construction
  % of Schwarz's manifold model.
  \begin{defn}
  Let $\conj{\R}=\R\cup\set{-\infty,+\infty}$ with the differentiable structure induced by the bijection
  \begin{equation*}
    \conj{\R}\to [-1,1]\ \ \ \ \varphi:t\mapsto \frac{2}{\pi}\arctan t
  \end{equation*}
  We will let $t$ to be the smooth function defined on $\R\subset \conj{\R}$.

  If $u:\conj{\R}\to \R^{n}$ is a $C^{1}$ function, then the derivative $\bd_{t}u$ is defined as usual on $\R$ and is defined to be $0$ on $\pm \infty$. This is a $C^{1}$ function on $\conj{\R}$. To see why, we compute on $(-1,1)$
  \begin{equation*}
    \bd_{t}u(\varphi^{-1}(s))=\frac{\bd_{s}(u(\varphi^{-1}(s)))}{\bd_{s}\varphi^{-1}(s)},
  \end{equation*}
  and use the fact that
  \begin{equation*}\tag{$\ast$}
    \varphi^{-1}(s)=\tan\frac{\pi}{2}s\implies \bd_{s}\varphi^{-1}(s)=\frac{\pi/2}{\cos^{2}(\pi s/2)}=\frac{\pi}{2}(1+\varphi^{-1}(s)^{2})
  \end{equation*}
  and thereby deduce for $u:\conj{\R}\to \R^{n}$ smooth we have
  \begin{equation*}
    \bd_{t}u(\varphi^{-1}(s))=\frac{2}{\pi}\bd_{s}u(\varphi^{-1}(s)) \cos^{2}(\pi s/2)
  \end{equation*}
  so $\bd_{t}u$ extends to to the boundary as a $C^{\infty}$ function. We can conclude from ($\ast$) that if $u$ is $C^{\infty}$ on $\conj{\R}$ then
  \begin{equation*}
    \abs{\bd_{t}u}\le \frac{\text{const}}{1+t^{2}}.
  \end{equation*}
  In particular, if $u$ is $C^{\infty}$ on $\conj{\R}$, then $\bd_{t}u$ is integrable on $\R$ (with its standard measure $\d t$). If $u$ is $C^{\infty}$ on $\conj{\R}$ and $u(\pm\infty)=0$, then $u$ is in $L^{p}$ for all $p$. To see why, observe that for $t_{1}<0$ we can compute for $t_{0}<t_{1}$
  \begin{equation*}
    u(t_{1})-u(t_{0})=\int_{t_{0}}^{t_{1}}\bd_{t}u\,\d t\implies \abs{u(t_{1})}\le \abs{u(t_{0})}+\frac{1}{\abs{t_{1}}}-\frac{1}{\abs{t_{0}}}.
  \end{equation*}
  Taking the limit as $t_{0}\to 0$ concludes that $\abs{u(t_{1})}\le
  \abs{t_{1}}^{-1}$, and hence $u$ is in $L^{p}(\R)$. It follows that
  $C^{\infty}_{0}(\conj{\R})\subset W^{k,p}(\R)$ for all $p>1$, where
  we agree to define $$C^{\infty}_{0}(\conj{\R})=\text{smooth
    functions vanishing at $\pm \infty$}.$$

  If $E$ is a smooth vector bundle on $\conj{\R}$, we define
  \begin{equation*}
    W^{k,p}_{\R}(E)\simeq W^{k,p}(\R^{n})
  \end{equation*}
  by choosing any smooth trivialization of $E$ over $\conj{\R}$.
\end{defn}
% \begin{defn}
%   Let $\text{Bun}(\conj{\R})$ be the category of bundles $E$ equipped with fiber metrics $\ip{-,-}$ and connections $\nabla$. Every such bundle $E$ is actually isomorphic to the standard bundle $(\R^{n},\d)$, in the sense that there is a flat section of
%   \begin{equation*}
%     g\in \text{O}(E,\R^{n})\subset \Hom(E,\R^{n}),\text{ $n=\dim E$};
%   \end{equation*}
%   such a flat section induces an metric isomorphism of bundles preserving the connections, in the sense that $d\circ g=g\circ \nabla$. To be a bit more precise, observe that
%   \begin{equation*}\tag{$\ast$}
%     h\in \Gamma\Hom(E,\R^{n})\mapsto \d\circ h-h\circ \nabla,
%   \end{equation*}
%   is a connection; this is well-defined 
%   since
%   \begin{equation*}
%     [\d\circ h-h\circ \nabla,f]=[\d,f]\circ h-h\circ [\nabla,f]=\d f\ocross h-\d f\ocross h=0.
%   \end{equation*}
%   Since any connection on any bundle over $[-1,1]$ has flat sections, we conclude a flat section $g$ of $(\ast)$ will satisfy $\d \circ g=g\circ \nabla$.
  
%   Let NBan be the category of Banach spaces. We will prefer to use $\text{Ban}$ to be the category of ``Banachable'' topological vector spaces.

%   Using our previous observations, we can define a functor
%   \begin{equation*}
%     E\in \text{Bun}(\conj{\R})\mapsto W^{k,p}_{\R}(E)\in \text{NBan},
%   \end{equation*}
%   by choosing an isomorphism $\varphi_{E}:(E,\ip{-,-},\nabla)\to (\R^{n},\ip{-,-},\d)$, and defining
%   \begin{equation*}
%     W^{k,p}_{\R}(E)=W^{k,p}(\R^{n}),
%   \end{equation*}
%   and extending to morphisms $\varphi:E_{1}\to E_{2}$ by conjugation
%   \begin{equation*}
%     \varphi_{E_{2}}\circ \varphi\circ \varphi_{E_{1}}^{-1}:W^{k,p}(\R^{n})\to W^{k,p}(\R^{n}).
%   \end{equation*}
%   This functor is well-defined up to natural isomorphism (in the category of normed Banach spaces -- this requires the isomorphisms $\varphi_{E}$ to be flat sections $\Gamma\mathrm{O}(E,\R^{n})$).

%   Now we make the observation any section of $g\in \mathrm{GL}(E_{1},E_{2})$ induces an isomorphism $$g:W^{k,p}_{\R}(E_{1})\to W^{k,p}_{\R}(E_{2})\text{ in Ban}$$
  
%   It suffices to prove this when $E_{1}=E_{2}=\R^{n}$. It follows because the $g$ are uniformly bounded in $C^{\infty}$, and hence $g\circ (-)$ induces a bounded invertible transformation $W^{k,p}(\R^{n})\to W^{k,p}(\R^{n})$ for all $k,p$. Finally, note that by our earlier work, there are continuous inclusions, for $k\ge 1$ and $p>1$
%   \begin{equation*}
%     C^{k}_{0}(\conj{\R},E)\to W^{k,p}_{\R}(E),
%   \end{equation*}
%   and for $k\ge 0$ and $p\in (1,\infty)$ a continuous inclusion
%   \begin{equation*}
%     W^{k+1,p}_{\R}(E)\to C^{k}_{0}(\conj{\R},E).
%   \end{equation*}
% \end{defn}
\begin{clear}{Remark}
  It is not too hard to show that $E\mapsto W^{k,p}_{\R}(E)$ is a
  Manifold model, and hence can be used to set up a path space
  $\mathscr{P}$ whose tangent spaces at a curve $\gamma$ are precisely
  $W^{k,p}_{\R}(\gamma^{*}TM)$.
\end{clear}
\end{clear}
% \begin{clear}{Remark}
% Recalling our ``outline'' (i)-(iv)
%   of how to construct $\mathscr{M}(x,y)$, we still need to show that
%   gradient flow lines are actually elements inside the Banach manifold
%   $\mathscr{P}=W^{k,p}_{x,y}(\closure{\R},M)$ constructed using
%   Schwarz's manifold model.

%   Since $C^{\infty,p}_{x,y}(\closure{\R},M)\subset \mathscr{P}$, it
%   suffices to show that gradient flow lines $\R\to M$ extend to smooth
%   maps $\closure{\R}\to M$. This is not immediate! It is proved at the
%   end of these notes.
% \end{clear}

\begin{clear}{Bundles on $\mathscr{P}$}
  % We will merely sketch the results. The way that Schwarz constructs
  % bundles on his path space $\mathscr{P}$ is also taken from
  % Eliasson's paper. 
  The main idea is the following:
  \begin{defn}
    A \textbf{section functor} over $W$ will be a functor
    \begin{equation*}
      L:\text{Bun}(\closure{X})\to \text{Ban},
    \end{equation*}
    equipped with a continuous bilinear maps
    \begin{equation*}
      L(E_{1})\ocross W(\Hom(E_{1},E_{2}))\to L(E_{2})\text{ and }L(E_{1})\ocross C^{\infty}(\Hom(E_{1},E_{2}))\to L(E_{2})
    \end{equation*}
    which extends the data inherent in the functor (recall the morphisms in $\text{Bun}(\closure{X})$ are smooth sections of $\Hom(E_{1},E_{2})$).
  \end{defn}
  \begin{example}
    The functor $E\mapsto L^{p}_{\R}(E)$ is a section functor over
    $W_{\R}^{1,p}$. 
  \end{example}
  \begin{thm}
    Given any section functor $L$ and any bundle $E\to M$, there is a smooth Banach bundle $$\mathscr{E}=L(W_{b}(\closure{X},M)^{*}E)\to W_{b}(\closure{X},M),$$

  If we let $\mathscr{P}=W_{b}(\closure{X},M)$, then we have
  \begin{equation*}
    \mathscr{E}=L(\mathscr{P}^{*}E)\to \mathscr{P}.
  \end{equation*}

  Furthermore, for all $u\in W_{b}(\closure{X},M)$ there are canonical identifications
  \begin{equation*}
    \mathscr{E}_{u}\to L(u^{*}E).
  \end{equation*}
\end{thm}
\begin{clear}{Remark}
  The culmination of all this work is that we have a smooth bundle
  $L^{p}(\mathscr{P}^{*}TM)$ whose fiber at a curve $u$ is
  $L^{p}(u^{*}TM)$. One can show that the ``gradient flow'' section
  $\gamma\mapsto \bd_{t} \gamma+\text{grad}\gamma$ (defined for smooth
  sections) extends to a smooth section of
  \begin{equation*}
    L^{p}(\mathscr{P}^{*}TM)\to \mathscr{P},
  \end{equation*}  
\end{clear}
\begin{prop}
  Let $M=\R^{n}$, and let $g$ be an arbitrary Riemannian metric on
  $\R^{n}$. Let $f$ be a Morse function, and let $x,y$ be two critical
  points. The linearization of the smooth section $\gamma\mapsto
  \gamma'+\text{grad}\circ \gamma$ at a flow line is the section
  \begin{equation*}\tag{$\ast$}
    \eta\in W^{1,p}(\gamma^{*}TM)\mapsto
    \eta'+\d\text{grad}_{\gamma}\cdot \eta\in L^{p}(\gamma^{*}TM)
  \end{equation*}
\end{prop}
\begin{clear}{Remark}
  The linearization ($\ast$) of
  $\gamma\mapsto \gamma'+\text{grad}\circ \gamma$ is precisely the
  kind of operator we were considering in Theorem \ref{thm:a251}. As a
  consequence, we conclude: the section $\gamma\mapsto
  \gamma'+\text{grad}\circ \gamma$ has Fredholm linearizations at the
  gradient flow lines, and moreover, the index is precisely
  \begin{equation*}
    \text{index}=\text{spectral flow of $t\mapsto \d\grad_{\gamma(t)}$}=\text{index}(x)-\text{index}(y).
  \end{equation*}
\end{clear}
\end{clear}
\begin{clear}{Generic metrics make $\gamma\mapsto
    \gamma'+\text{grad}\circ \gamma$ a transverse section of
    $L^{p}(\mathscr{P}^{*}TM)\to \mathscr{P}$}
  Recalling our outline (i)-(iv), we have still not said anything
  about how \emph{generic metrics} will ``cut'' $\mathscr{M}(x,y)$
  transversally. In this section we will introduce an important
  theorem called the \textbf{parametric transversality theorem}.

  \begin{thm}
    Let $\mathscr{E}\to \mathscr{P}$ be a smooth Banach bundle over a
    Banach bundle, and let $\mathscr{G}$ be another Banach
    manifold. Suppose we are given a smooth map
    \begin{equation*}
      \Phi:\mathscr{G}\cross \mathscr{P}\to \mathscr{E}
    \end{equation*}
    so that $\Phi_{g}:\set{g}\cross \mathscr{P}\to \mathscr{E}$ is a
    section for each $g$, i.e.\ $\Phi$ is a ``parametrized family'' of
    sections. Suppose the following conditions are satisfied:

    (i) In some (countable) atlas of trivializations
    $(\psi_{i},U_{i})$, $\psi_{i}:\mathscr{E}|_{U_{i}}\to U_{i}\cross
    E_{i}$, the maps
    \begin{equation*}
      \text{pr}\circ \psi_{i}\circ\Phi:\mathscr{G}\cross U_{i}\to E_{i}
    \end{equation*}
    have $0\in E_{i}$ as a regular value. 

    (ii) The maps $\text{pr}\circ \psi_{i}\circ \Phi_{g}:U_{i}\to
    E_{i}$ are Fredholm maps with index $r$ (for every $g$). 

    Then the conclusion is that there exists a Baire generic set
    $\Sigma\subset \mathscr{G}$ so that for all $g\in \Sigma$, the
    section $\Phi_{g}$ is transverse to the zero section, i.e.\ the
    vertical differential of $\Phi_{g}$ is surjective on
    $\Phi_{g}^{-1}(0)$, and consequently $\Phi_{g}^{-1}(0)$ is a
    smooth submanifold, whose dimension is necessarily $r$.
  \end{thm}

  \begin{example}
    Let's apply this theorem to the following set up. Let $M$ be a
    compact manifold. Define
    \begin{equation*}
      \mathscr{G}=\set{C^{k}\text{ Riemannian metrics on $M$}}.
    \end{equation*}
    Then $\mathscr{G}$ is a convex open subset of the Banach space
    \begin{equation*}
      \mathscr{G}\subset C^{k}(M,\text{Sym}^{2}(T^{*}M)),
    \end{equation*}
    and hence $\mathscr{G}$ has a natural Banach manifold
    structure. Let $\mathscr{P}=W^{1,2}_{x,y}(\closure{R},M)$, and
    $\mathscr{E}=L^{2}(\mathscr{P}^{*}TM)$, as above. 

    One can show that the map
    \begin{equation*}
      \Phi:(g,\gamma)\in \mathscr{G}\cross \mathscr{P}\to
      \gamma'+\text{grad}_{g}\circ \gamma \in L^{2}(\gamma^{*}TM)
    \end{equation*}
    is a smooth family of sections
    $\Phi:\mathscr{G}\cross \mathscr{P}\to \mathscr{E}$. We claim that
    the conditions (i) and (ii) are satisfied.

    To see why, let's suppose that $(g,\gamma)\in \Phi^{-1}(0)$. Then
    $\gamma$ is a flow line joining $x$ to $y$. The linearization of
    $\Phi$ at $(g,\gamma)$ is a linear map
    \begin{equation*}
      \begin{aligned}
        D\Phi_{(g,\gamma)}&:C^{k}(\text{Sym}^{2}(T^{*}M))\cross
        W^{1,p}(\gamma^{*}TM)\to L^{2}(\gamma^{*}TM)\\
        (B,\eta)&\mapsto D_{1}\Phi_{g,\gamma}B+D_{2}\Phi_{g,\gamma}\eta.
      \end{aligned}
    \end{equation*}
    The map $D_{2}\Phi_{g,\gamma}$ is simply the linearization of the
    section $\gamma\mapsto \gamma'+\text{grad}_{g}\circ \gamma$. We
    already showed that this was Fredholm of index
    $\text{index}(x)-\text{index}(y)$. Moreover, the linearization
    $D_{2}\Phi_{g,\gamma}$ is actually an (elliptic) first order
    differential operator
    $W^{1,2}(\gamma^{*}TM)\to L^{2}(\gamma^{*}TM)$, and therefore
    satisfies unique continuation.

    Since $D_{2}\Phi_{g,\gamma}$ has closed image and finite
    dimensional cokernel, the map
    $$(B,\eta)\mapsto D_{1}\Phi_{g,\gamma}B+D_{2}\Phi_{g,\gamma}\eta$$
    still has closed image and finite dimensional cokernel. We claim
    that the cokernel is zero dimensional. If not, then we can find
    $c\in L^{2}$ so that $c\perp \im D\Phi$, in particular $c\in \ker
    D_{2}\Phi_{g,\gamma}^{*}$, and therefore, by unique continuation
    for elements in the kernel of a first order elliptic operator on
    $\R$, we conclude $c$ is everywhere non-zero.

    Now we compute
    \begin{equation*}
      D_{1}\Phi_{g,\gamma}B=D_{B}\grad_{g}\circ \gamma,
    \end{equation*}
    where
    \begin{equation*}
      D_{B}\text{grad}_{g}=\lim_{\epsilon\to 0}\epsilon^{-1}(\text{grad}_{g+\epsilon B}-\text{grad}_{g})
    \end{equation*}
    Pick $p\in \gamma(\R)$, and pick coordinates $z_{1},\cdots,z_{n}$
    near $p$. Write
    \begin{equation*}
      \d f=\pd{f}{z_{i}}\d z_{i}\text{ and }g=g_{ij}\d x_{i}\d
      x_{j}\text{ and }\grad_{g}=a_{i}(g)\bd_{i}
    \end{equation*}
    Since
    \begin{equation*}
      g_{ij}a_{i}(g)=\pd{f}{z_{j}}\implies a_{i}(g)=(g^{-1})_{ij}\pd{f}{z_{j}},
    \end{equation*}
    We compute
    \begin{equation*}
      (g+B)^{-1}=g_{ij}^{-1}(1+g_{ij}^{-1}B_{ij})^{-1}=(g^{-1})-(g^{-2})B+O(\abs{B}^{2})\implies
      D_{B}\text{grad}_{g}=((g^{-2})B)_{ij}\pd{f}{z_{j}}\bd_{i}.
    \end{equation*}
    Since $\bd f/\bd z_{j}$ is a non-zero vector, and $B$ is
    arbitrary, we can choose $B$ so that
    $((g^{-2})B)_{ij}\pd{f}{z_{j}}\bd_{i}$ so that the pairing
    $\ip{D_{1}\Phi_{(g,\gamma)},c}\ne 0$, unless $c$ vanishes at
    $z$. This proves that $c$ must vanish at $z$, and hence $c$ is
    identically zero, contradicting our assumption that $c\ne
    0$. Therefore $D\Phi_{(g,\gamma)}$ is surjective, as desired.

    It follows that the family of sections $\Phi:\mathscr{G}\cross
    \mathscr{P}\to \mathscr{E}$ satisfies the requirements (i) and    (ii) needed to apply the parametric transversality theorem.
  \end{example}
\end{clear}

% \begin{clear}{Bonus}
% Here are a sequence of exercises culminating in a proof of Theorem \ref{thm:a251}.

% \begin{xca}[just for fun]
%   Prove that $L:W^{1,p}(\R,\C^{n})\to L^{p}(\R,\C^{n})$ given by

%   $L(u)=\bd_{t}u$ is not Fredholm, even though $\bd_{t}$ and
%   $\bd_{t}^{*}=-\bd_{t}$ have finite dimensional kernels. 
% \end{xca}
% \begin{xca}\label{xca:a252}
%   Prove that $L:W^{1,p}(\R,\C)\to L^{p}(\R,\C)$ given by
%   \begin{equation*}
%     L(u)=\bd_{t}u+\lambda u\text{ for $\lambda\not\in i\R$}.
%   \end{equation*}
%   is Fredholm as follows: for test functions $u$, let $v=L(u)$, we can write
%   \begin{equation*}
%     e^{\lambda t}u(t)-e^{\lambda s}u(s)=\int_{s}^{t}e^{\lambda
%       \tau}v(\tau)\,\d \tau.
%   \end{equation*}
%   If $\text{Re}(\lambda)>0$, take $s\to -\infty$, and conclude
%   \begin{equation*}\tag{$1$}
%     u(t)=\int_{-\infty}^{0}e^{\lambda r}v(t+r)\,\d r.
%   \end{equation*}
%   If $\text{Re}(\lambda)<0$, take $t\to+\infty$, (and then relabel $s\to t$) and conclude
%   \begin{equation*}\tag{$2$}
%     u(t)=-\int_{0}^{\infty}e^{\lambda r}v(t+r)\,\d r.
%   \end{equation*}
%   By pairing $u$ with a test function in $L^{q}$ and integrating,
%   conclude from either (1) or (2) (and recalling $v=Lu$) that
%   \begin{equation*}
%     \norm{u}_{L^{p}}\le \abs{\text{Re}(\lambda)}^{-1}\norm{Lu}_{L^{p}}.
%   \end{equation*}
%   Now estimate
%   \begin{equation*}
%     \begin{aligned}
%       \norm{u}_{W^{1,p}}&=\norm{u}_{L^{p}}+\norm{\bd_{t}u}_{L^{p}}\le
%       (\abs{\text{Re}(\lambda)}^{-1}+1)\norm{Lu}_{L^{p}}+\norm{\lambda
%         u}_{L^{p}}\\
%       &\le (1+(1+\abs{\lambda})
%       \abs{\text{Re}(\lambda)}^{-1})\norm{Lu}_{L^{p}}\\
%       &=C_{\lambda}\norm{Lu}_{L^{p}},
%     \end{aligned}    
%   \end{equation*}
%   where the constant $C_{\lambda}$ blows up as
%   $\text{Re}(\lambda)\to 0$. By density of test functions in
%   $W^{1,p}$, this estimate holds for all elements in
%   $W^{1,p}$. Conclude that $u\mapsto Lu$ is a closed embedding (i.e.\
%   has no kernel and closed image).

%   We can prove that $L:W^{1,p}\to L^{p}$ is surjective fairly easily now that we know
%   that $L$ has closed image. Because $\text{im}(L)$ is closed, it
%   suffices to prove that $\text{im}(L)$ is dense. If $\text{im}(L)$ is
%   not dense, then we can find an $L^{q}$ function $\varphi\ne 0$ so that
%   \begin{equation*}
%     \ip{Lu,\varphi}=0\text{ for all $u$}\implies
%     -\bd_{t}\varphi+\lambda \varphi=0\text{ in the sense of distributions}. 
%   \end{equation*}
%   It follows that $\varphi\in W^{1,q}$ and
%   $\bd_{t}\varphi-\lambda\varphi=-L^{*}\varphi=0$. By repeating the
%   argument above, we conclude
%   \begin{equation*}
%     \norm{\varphi}_{1,q}\le C_{\lambda}\norm{L^{*}\varphi}_{q},
%   \end{equation*}
%   which implies $L^{*}\varphi=0\implies \varphi=0$, contradicting the
%   fact that we assumed $\varphi\ne 0$. Therefore $L$ is an
%   isomorphism (and so is certainly Fredholm). 
% \end{xca}
% \begin{xca}
%   Let $A$ be an arbitrary $n\cross n$ matrix with no imaginary
%   eigenvalues. Prove that the differential operator $L=\bd_{t}+A$
%   induces an isomorphism
%   \begin{equation*}
%     L:W^{1,p}\to L^{p},
%   \end{equation*}
%   by arguing as follows: First suppose that $A$ has distinct
%   eigenvalues. It follows that we can choose a frame
%   $e_{1},\cdots,e_{n}$ of eigenvectors so that in this frame $u=\sum u_{i}e_{i}$
%   \begin{equation*}
%     Lu=\sum_{i}(\bd_{t}u_{i}+\lambda_{i}u_{i})e_{i}.
%   \end{equation*}
%   Since we know that maps $u_{i}\mapsto \bd_{t}u_{i}+\lambda_{i}u_{i}$ are
%   isomorphisms, it follows that $L$ is bijective, and hence an
%   isomorphism. Moreover, use the previous results to conclude an estimate
%   \begin{equation*}
%     \norm{u}_{W^{1,p}}\le
%     C(\text{Re}(\lambda_{1}),\cdots,\text{Re}(\lambda_{n}))\norm{Lu}_{L^{p}}=C_{A}\norm{Lu}
%   \end{equation*}
%   where the constant $C_{A}$ is a continuous function of the eigenvalues
%   $\lambda_{1},\cdots,\lambda_{n}\in \C\setminus i\R$.

%   Now suppose that $A$ has repeat eigenvalues. For all $A'$ with
%   simple eigenvalues, compute
%   \begin{equation*}
%     \norm{\bd_{t}+A-(\bd_{t}+A')}\le \abs{A-A'},
%   \end{equation*}
%   and so 
%   \begin{equation*}
%     \norm{u}_{W^{1,p}}\le C_{A'}\norm{L'u}_{L^{p}}\implies
%     \norm{u}_{W^{1,p}}\le
%     C_{A'}\norm{Lu}_{L^{p}}+C_{A'}\abs{A-A'}\norm{u}_{W^{1,p}}. 
%   \end{equation*}
%   We claim that we can pick $A'$ so that $C_{A'}\abs{A-A'}<1/2$. This
%   can be achieved since
%   \begin{equation*}
%     C_{A'}\abs{A-A'}\le C_{A}\abs{A-A'}+\underbrace{\abs{C_{A}-C_{A'}}\abs{A'-A}}_{\text{ctns}},
%   \end{equation*}
%   so it suffices to pick $A'$ sufficiently close to $A$. Then we
%   conclude that
%   \begin{equation*}
%     \norm{u}_{W^{1,p}}\le
%     C\norm{Lu}_{L^{p}}.
%   \end{equation*}
%   Arguing as we did above (or differently if you wish), we conclude that $L$ is an isomorphism.  
% \end{xca}
% \begin{xca}
%   Consider a time dependent family of matrices $A(t)$, so that
%   $A(-\infty)$ and $A(+\infty)$ converge to matrices with no
%   imaginary eigenvalues. Prove that $L=\bd_{t}+A(t)$ is Fredholm as
%   follows: let $L_{-}=\bd_{t}+A(-\infty)$,
%   $L_{+}=\bd_{t}+A(+\infty)$. We have shown that there are constants
%   $C_{\pm}$ so that
%   \begin{equation*}
%     \norm{u}_{L^{p}}\le C_{\pm}\norm{L_{\pm}u}_{L^{p}}.
%   \end{equation*}
%   Suppose that $\abs{A(t)-A(-\infty)}<\epsilon$, $t<-R$,
%   $\abs{A(t)-A(\infty)}<\epsilon$, $t>R$, for some constant $\epsilon>0$
%   to be determined as function of $C_{\pm }$ ($R$ depends on $\epsilon$). Consider a partition
%   of unity $\rho_{-},\rho,\rho_{+}$ so that $\rho_{-}$ is supported in
%   $(-\infty,-R)$, $\rho$ in $(-2R,2R)$, and $\rho_{+}$ in
%   $(R,\infty)$. For any $u$ compute
%   \begin{equation*}
%     \begin{aligned}
%       \norm{u}_{L^{p}}&\le \norm{\rho_{-}u}_{L^{p}}+\norm{\rho
%         u}_{L^{p}}+\norm{\rho_{+}u}_{L^{p}}\\
%       &\le C_{-}\norm{L_{-}\rho_{-}u}_{L^{p}}+\norm{\rho
%         u}_{L^{p}}+C_{+}\norm{L_{+}\rho_{+}u}_{L^{p}}\\
%       &\le C_{-}\norm{L\rho_{-}u}_{L^{p}}+\norm{\rho
%         u}_{L^{p}}+C_{+}\norm{L\rho_{+}u}_{L^{p}}+\epsilon(C_{-}+C_{+})\norm{u}_{L^{p}}\\
%       &\le C\left(\,\norm{Lu}_{L^{p}}+\norm{\rho u}_{L^{p}}+\norm{[L,\rho_{-}]u}_{L^{p}}+\norm{[L,\rho_{+}]u}_{L^{p}}\,\right)+\epsilon(C_{-}+C_{+})\norm{u}_{L^{p}},
%     \end{aligned}
%   \end{equation*}
%   where $[L,\rho_{\pm}]=L\rho_{\pm}-\rho_{\pm} L$ is the commutator, which is a zeroth
%   order operator supported in $[-2R,2R]$ (since the derivatives of
%   $\rho_{\pm}$ are supported $[-2R,2R]$). Now let's agree to have
%   picked $\epsilon$ smaller than $2^{-1}(C_{-}+C_{+})^{-1}$, so that
%   we conclude
%   \begin{equation*}
%     \norm{u}_{L^{p}}\le 2C\left(\,\norm{Lu}_{L^{p}}+\norm{\rho u}_{L^{p}}+\norm{[L,\rho_{-}]u}_{L^{p}}+\norm{[L,\rho_{+}]u}_{L^{p}}\,\right).
%   \end{equation*}
%   Consider the continuous map
%   \begin{equation*}
%     K:W^{1,p}(\R)\to W^{1,p}([-2R,2R])^{3}\ \ Ku=(\rho u, [L,\rho_{-}]u,[L,\rho_{+}]u).
%   \end{equation*}
%   Arguing as we did in Exercise \ref{xca:a252}, we conclude a new
%   constant $C$ (independent of $u$) so that
%   \begin{equation*}
%     \norm{u}_{W^{1,p}}\le C\left(\,\norm{Lu}_{L^{p}}+\norm{Ku}_{L^{p}([-2R,2R])^{3}}\,\right).
%   \end{equation*}
%   Since $K:W^{1,p}\to W^{1,p}([-2R,2R])^{3}\to L^{p}([-2R,2R])^{3}$ is
%   a compact operator (since $$W^{1,p}([-2R,2R])\to L^{p}([-2R,2R])$$ is
%   compact by Rellich compactness) conclude that $L$ has closed image and finite dimensional
%   kernel. Since $L^{*}$ is of the same form as $L$, we can repeat the
%   argument to show that $\ker L^{*}$ is finite dimensional.

%   Finally, conclude (via a standard argument) that $L$ is Fredholm by
%   showing that $\coker L$ is isomorphic to
%   $(\ker L^{*})^{\text{dual}}$.
% \end{xca}
% \begin{xca}
%   In this final exercise, we will show that the index of
%   \begin{equation*}
%     L=\bd_{t}+A(t),
%   \end{equation*}
%   is equal to the spectral flow of $A(t)$.

%   We argue as follows. First, consider a perturbation $A^{s}$ so that
%   $A^{0}=A$ and

%   (i) $A^{s}(\pm \infty)$ has no imaginary eigenvalues, for all $s$,

%   (ii) $A^{1}(t)$ has distinct eigenvalues for each $t$ (including $\pm\infty$).

%   We leave it to the reader to convince themselves that this is
%   possible. The family of operators $L^{s}=\bd_{t}+A^{s}(t)$ is a
%   family of Fredholm operators (by ($i$)) and hence the index does not
%   change during this perturbation. Without loss, suppose that $A(t)$
%   has distinct eigenvalues for all time.

%   The space of unit-length eigen-frames for $A(t)$ is discrete (since
%   $A(t)$ has distinct eigenvalues) and hence we can find a traveling
%   family of unit length eigenvectors $e_{1}(t),\cdots,e_{n}(t)$ smooth on $\R$,
%   and continuous on $\closure{\R}$. By writing
%   \begin{equation*}
%     u=\sum_{i}u_{i}e_{i}(t),
%   \end{equation*}
%   we conclude that
%   \begin{equation*}
%     Lu=\sum_{i}(\bd_{t}u_{i}+\lambda_{i}(t)u_{i})e_{i}+\sum e_{i}'(t)u_{i}.
%   \end{equation*}
%   Since $A(\pm\infty)$ converge, $e_{i}'(t)\to 0$ as
%   $t\to\pm\infty$. Hence the family of operators
%   \begin{equation*}
%     L^{s}u=\sum_{i}(\bd_{t}u_{i}+\lambda_{i}(t)u_{i})e_{i}+s\sum e_{i}'(t)u_{i}
%   \end{equation*}
%   is a family of Fredholm operators. The index is unchanged, and so it
%   suffices to prove the case when
%   \begin{equation*}
%     Lu=\sum_{i}(\bd_{t}u_{i}+\lambda_{i}(t)u_{i})e_{i}(t).
%   \end{equation*}
%   Conclude (directly) that in this case
%   \begin{equation*}
%     \text{index}(L)=\sum_{i}\text{index}(\bd_{t}+\lambda_{i}(t)).
%   \end{equation*}
%   This reduces the problem to computing the index of a fairly simple
%   differential operator. We compute
%   \begin{equation*}\tag{$\ast$}
%     \bd_{t}u+\lambda_{i}(t)u=0\iff
%     u(t)=u(0)\exp(-\int_{0}^{t}\lambda_{i}(s)\,\d s).
%   \end{equation*}
%   The only way this can possible converge to an element of $W^{1,p}$
%   is if
%   \begin{equation*}
%     \text{Re}\lambda_{i}(-\infty)<0\text{ and }\text{Re}\lambda_{i}(+\infty)>0,
%   \end{equation*}
%   conversely, if the above is satisfied, conclude that ($\ast$)
%   defines an element of $W^{1,p}$ in the kernel of
%   $\bd_{t}+\lambda_{i}(t)$. Therefore we conclude
%   \begin{equation*}
%     \dim\ker(\bd_{t}+\lambda_{i}(t))=
%     \begin{cases}
%       1&\text{ if }\text{Re}\lambda_{i}(-\infty)<0<\text{Re}\lambda_{i}(+\infty)\\
%       0&\text{ otherwise }.
%     \end{cases}
%   \end{equation*}
%   The adjoint of the differential operator is
%   $(\bd_{t}+\lambda_{i}(t))^{*}=-(\bd_{t}-\lambda_{i}(t))$, and so the same argument shows
%   that
%   \begin{equation*}
%     \dim\ker(\bd_{t}-\lambda_{i}(t))=\begin{cases}
%       1&\text{ if }\text{Re}\lambda_{i}(-\infty)>0>\text{Re}\lambda_{i}(+\infty)\\
%       0&\text{ otherwise }.
%     \end{cases}
%   \end{equation*}
%   Conclude that
%   \begin{equation*}
%     \text{index}(\bd_{t}+\lambda_{i}(t))=\#\text{ eigenvalues which
%       cross the imaginary axis, with signs},
%   \end{equation*}
%   and conclude that
%   \begin{equation*}
%     \text{index}(L)=\#\text{ eigenvalues which
%       cross the imaginary axis, with signs},
%   \end{equation*}
%   as desired.
% \end{xca}  
% \end{clear}
% \begin{clear}{Proof that $W^{k,p}_{\R}$ is a manifold model}
  
% \begin{lemma}[Banach algebra]
%   The obvious bilinear pairing
%   \begin{equation*}
%     W^{k,p}_{\R}(E_{1})\ocross W^{k,p}_{\R}(\Hom(E_{1},E_{2}))\to W^{k,p}_{\R}(E_{2})
%   \end{equation*}
%   is continuous.   
% \end{lemma}
% \begin{proof}
%   It suffices to prove this for $E_{1}=\R^{n}$, $E_{2}=\R^{m}$, whereby it reduces to the claim that
%   \begin{equation*}
%     W^{k,p}(\R^{n})\ocross W^{k,p}(\Hom(\R^{n},\R^{m}))\to W^{k,p}(\R^{m})
%   \end{equation*}
%   is continuous. This is true because 
%   \begin{equation*}
%     \d^{k}(Au)=\sum_{a+b=k}\d^{a}A\ocross \d^{b}u,
%   \end{equation*}
%   so
%   \begin{equation*}
%     \norm{Au}_{W^{k,p}}\le \sum_{a+b\le k}\norm{A}_{W^{k,p}}\norm{u}_{W^{k,p}},
%   \end{equation*}
%   where we use that $\norm{\d^{b}u}_{C^{0}}\le \norm{u}_{W^{k,p}}$ if $b<k$ and $\norm{\d^{b}u}_{L^{p}}\le \norm{u}_{W^{k,p}}$ for all $b$ (and similarly for $A$). 
% \end{proof}

% \begin{defn}
%   Let $U$ be an precompact open subset of $E$ (the total space of $E\to \conj{\R}$), and define $W^{k,p}_{\R}(U)$ to be the set of sections of $W^{k,p}_{\R}(E)$ which lie in $U$. This is open in the $C^{0}$ topology (which is just the compact open topology) and hence is open in $W^{k,p}_{\R}(U)$ since $W^{k,p}_{\R}(U)$ includes into $C^{0}(\conj{\R},E)$.

%   Suppose we have two bundles $E_{1},E_{2}$ and an open subset
%   $U\subset E_{1}$. Let $f:U\to E_{2}$ be a fiber preserving smooth
%   function which extends to $\closure{U}$, and which vanishes on
%   $0\cap \bd\conj{\R}$. It is clear that there is a continuous composition map
%   \begin{equation*}
%     f\circ : C^{0}_{0}(U)\to C^{0}_{0}(E_{2})
%   \end{equation*}
%   This morphism is the empty morphism if there are no sections of $E_{1}$ lying in $U$ and vanishing at the endpoints. We can thereby suppose that $U$ contains $\pm\infty$. It will be important to note that $C^{0}_{0}(U)$ is locally convex: if $u\in C^{0}_{0}$, then we can cover the graph of $u$ by finitely many open sets of the form $I\cross B$, where $B$ is a ball in each fiber. Then this open set is convex around $u$. 
% \end{defn}
% \begin{lemma}
%   Let $f:\closure{U}\subset E_{1}\to E_{2}$ be a fiber preserving smooth map. There is a unique continuous composition map making the diagram commute
%   \begin{equation*}
%     \begin{tikzcd}
%       C^{0}_{0}(U)\arrow[r,"f\circ "]&C^{0}_{0}(E_{2})\\
%       W^{k,p}_{\R}(U)\arrow[u,"\subset"]\arrow[r,"f\circ"]&W^{k,p}_{\R}(E_{2})\arrow[u,"\subset"]
%     \end{tikzcd}
%   \end{equation*}  
% \end{lemma}
% \begin{proof}
%   The uniqueness is clear, since the inclusions are injective. Note that $f\circ$ is already well-defined on the dense subspace $C^{\infty}_{0}(U)\subset W^{k,p}_{\R}(U)$. It suffices to show that this function $f\circ $ is continuous on $C^{\infty}_{0}(U)$ in the $W^{k,p}_{\R}(U)$ topology. 

%   We can now assume that $E_{1}=\R^{n}$, $E_{2}=\R^{m}$. We introduce some notation. Let $F^{1}f$ denote the fiber derivative of $f$, so that $F^{1}f$ is a function $\closure{U}\to \Hom(\R^{n},\R^{m})$, and let $f_{t}:\closure{U}\to \R^{n}$ be the time derivative.
%   \begin{claim}
%     Let $u\in C^{\infty}_{0}(U)$. There is a constant $C$ depending on $u$ and $f$, so that for each $v$ sufficiently close to $u$, we have
%     \begin{equation*}
%       \norm{f\circ u-f\circ v}_{W^{k,p}}\le C\norm{u-v}_{W^{k,p}}.
%     \end{equation*}
%   \end{claim}
%     We begin by requiring $v$ to lie in a convex open set around $u$ and $\norm{u-v}_{W^{k,p}}\le 1$ and so that we may easily estimate $\norm{v}_{W^{k,p}}\le 1+\norm{u}_{W^{k,p}}$.

%     We argue by induction. The base case follows easily, since we can estimate
%     \begin{equation*}
%       \abs{f(t,u(t))-f(t,v(t))}\le \norm{F^{1}f}_{C^{0}}\abs{u(t)-v(t)}.
%     \end{equation*}
%     For the inductive step $k-1\to k$, observe that it suffices to estimate
%     \begin{equation*}
%       \norm{\bd_{t}(f(t,u(t))-f(t,v(t)))}_{W^{k-1,p}}\le p^{k}(\norm{u}_{W^{k,p}})\norm{u-v}_{W^{k,p}}.
%     \end{equation*}
%     We compute
%     \begin{equation*}
%       \bd_{t}(f(t,u(t))=(F^{1}f\circ u)\cdot u+f_{t}\circ u,
%     \end{equation*}
%     and hence
%     \begin{equation*}
%       \begin{aligned}
%         \norm{\bd_{t}(f(t,u(t))-f(t,v(t)))}_{W^{k-1,p}}\le &\norm{F^{1}f\circ u-F^{1}f\circ v}_{W^{k-1,p}}\norm{u}_{W^{k-1,p}}\\
%         &+\norm{F^{1}f\circ v}\norm{u-v}_{W^{k-1,p}}+\norm{f_{t}\circ u-f_{t}\circ v}_{W^{k-1,p}}.
%       \end{aligned}
%     \end{equation*}
%     By the inductive assumption, applied twice, we estimate
%     \begin{equation*}
%       \norm{F^{1}f\circ u-F^{1}f\circ v}_{W^{k-1,p}}\norm{u}_{W^{k-1,p}}+\norm{f_{t}\circ u-f_{t}\circ v}_{W^{k-1,p}}\le C\norm{u-v}_{W^{k,p}},
%     \end{equation*}
%     where we recall that $C$ is allowed to depend on $\norm{u}_{W^{k-1,p}}$.
%     Next, we estimate the term
%     \begin{equation*}
%       \norm{F^{1}f\circ v}\le \norm{F^{1}f\circ v-F^{1}f\circ u}+\norm{F^{1}f\circ u}\le C(1+\norm{F^{1}f\circ u})=C,
%     \end{equation*}
%     and thereby conclude
%     \begin{equation*}
%       \norm{\bd_{t}(f(t,u(t))-f(t,v(t)))}_{W^{k-1,p}}\le C\norm{u-v}_{W^{k,p}},
%     \end{equation*}
%     as desired. This completes the proof of the Claim, and therefore the Lemma.
% \end{proof}
% \end{clear}
% \begin{clear}{Gradient flow lines are elements in Schwarz's path space
%     $\mathscr{P}$}
%   Let $\conj{X}=\conj{\R}$, $X=\R$, and let $W$ be any manifold model. Let $\varphi$ be a Morse function with critical points at $x$ and $y$, and let $\gamma:\R\to M$ be a trajectory of $\text{grad}_{\varphi}$ joining $x$ to $y$. We claim that the continuous extension $\gamma:\conj{\R}\to M$ is smooth, and hence is an element in $\mathscr{P}^{W}_{x,y}$.

%   This claim essentially says that $\gamma$ decays ``enough'' at $x$ and $y$ so that it can be smoothly extended to $\conj{\R}$. To prove the claim, we remark that in local coordinates near $x$, and the $s$ coordinate near $-1$ in $[-1,1]$, we know that on $(-1,-1+\epsilon)$ we have
%   \begin{equation*}
%     u=\gamma\circ \varphi^{-1}\text{ satisfies }\bd_{s}u=\frac{\gamma'\circ \varphi^{-1}}{\cos^{2}(\pi/2 s)}
%   \end{equation*}
%   It therefore suffices to prove that
%   \begin{equation*}
%     \abs{\gamma^{(k)}(t)}\le\ \text{const}_{k,n}(1+t^{2})^{-2n}
%   \end{equation*}
%   for all $n$ and $k$, as $t\to \pm \infty$. We first observe that 
%   \begin{equation*}
%     \gamma^{(k)}=\bd_{t}^{k-1}(\text{grad} \circ \gamma)=\bd_{t}^{k-2}(\d \text{grad} \circ \gamma)\gamma'=\cdots=\sum_{a_{1}+\cdots+a_{j}=k-1} (\d^{j}\text{grad}\circ \gamma)(\gamma^{(a_{1})},\cdots,\gamma^{(a_{j})}).
%   \end{equation*}
%   We can estimate this as follows
%   \begin{equation*}
%     \sum_{a_{1}+\cdots+a_{j}=k-1} (\d^{j}\text{grad}\circ \gamma)(\gamma^{(a_{1})},\cdots,\gamma^{(a_{j})})\le \sum\norm{\text{grad}}_{C^{k}}\abs{\gamma^{(a_{1})}}\cdots \abs{\gamma^{(a_{j})}},
%   \end{equation*}
%   so if we prove it when $k=0,1$, it will follow for all higher $k$ by induction. Since $\text{grad}$ vanishes at $x=0$, we can estimate
%   \begin{equation*}
%     \abs{\grad(\gamma(t))}\le \norm{\text{grad}}_{C^{1}}\abs{\gamma(t)},
%   \end{equation*}
%   and so it suffices to prove for $k=0$. To prove it for $k=0$ will require a more careful analysis of the ODE, and will require us to use invertibility of $\d\grad_{x}$.

%   \begin{xca}
%     Let $\varphi,\grad,\gamma$ be as above. Suppose that $\gamma(t)$ lies in a coordinate ball around $x$ for $t<T<<0$.

%     Prove that for all compactly supported $f$
%     \begin{equation*}
%       \ud{}{t}f(t)\varphi(\gamma(t))=f'(t)\varphi(\gamma(t))+f(t)\abs{\grad(\gamma(t))}^{2},
%     \end{equation*}
%     and hence
%     \begin{equation*}
%       \int_{-\infty}^{T} f(t) \abs{\grad(\gamma(t))}^{2}\d t\le f(T)\abs{\varphi(\gamma(T))}+\int_{-\infty}^{T}\abs{f'(t)}\abs{\varphi(\gamma(t))}\,\d t
%     \end{equation*}
%     By picking $f$ positive and compactly supported so $f(t)=\abs{t}^{k}$ on $[S,T]$, and so $\abs{f'(t)}\le k\abs{t}^{k-1}$ everywhere, conclude that
%     \begin{equation*}
%       \abs{t}^{k-1}\abs{\gamma(t)}^{2}\text{ integrable }\implies \abs{t}^{k}\abs{\grad(\gamma(t))}^{2}\text{ integrable }\implies \abs{t}^{k}\abs{\gamma(t)}^{2}\text{ integrable},
%     \end{equation*}
%     where we have used the fact that $\abs{\gamma(t)}\le \text{const}\abs{\grad(\gamma(t))}$ for $t$ in $(-\infty,T]$, this follows from non-degeneracy of $\d\grad_{x}$. Since $\abs{\grad(\gamma(t))}^{2}$ is integrable, we conclude that $\abs{t}^{k}\abs{\gamma(t)}^{2}$ is integrable for all $k$.

%     To prove that $\abs{t}^{k}\abs{\gamma(t)}$ is bounded, compute
%     \begin{equation*}
%       \ud{}{t}(\abs{t}^{k}\gamma(t))=\abs{t}^{k}\text{grad}\circ \gamma+k\abs{t}^{k-1}\gamma(t),
%     \end{equation*}
%     which implies that the first derivatives of $\abs{t}^{k}\gamma(t)$ are integrable for all $k$. It follows that $\abs{t}^{k}\gamma(t)$ is bounded for all $k$, and hence
%     \begin{equation*}
%        \abs{\gamma(t)}\le\ \text{const}_{n}(1+t^{2})^{-2n},\text{ as $t\to\pm \infty$}
%      \end{equation*}
%      for all $n$, as desired. 
%   \end{xca}  

% \end{clear}

\end{document}