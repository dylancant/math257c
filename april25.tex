\documentclass{amsart}
\usepackage{preamble}

\begin{document}
\section*{\textbf{April 25 (Dylan/Lie)}}

Back to Morse theory to complete the analytic program. Recall the
outline we intend to follow. The reader is encouraged to read the
first section of Schwarz's book for more details on this outline.

\begin{enumerate}
\item Functional setup.
\item Fredholm/index theory.
\item Transversality.
\item Compactness.
\item Gluing
\item Orientation (not today).
\end{enumerate}

Let's recall the setting: suppose that $M$ is a closed manifold, $f$
is a Morse function, and $g$ is a Riemannian metric on $M$. The last
time we were considering this setup, we proved the ``local stable
manifold theorem.'' In this lecture, we intend to prove a more
``global'' theorem -- we will explore the moduli space of flow lines
joining two critical points of $f$ in $M$. 

Let $\grad$ denote the gradient with respect to $g$ of the Morse
function $f$. A \textbf{flow line} $\gamma:\R\to M$ is a $C^{\infty}$
function satisfying
\[
  \gamma'+\grad_f\circ \gamma=0.
\]
If $x,y$ are critical points of $f$, and $\gamma$ is a flow line
satisfying the asymptotic conditions $\gamma(-\infty)=x$,
$\gamma(\infty)=y$, then we call $\gamma$ a \textbf{connecting trajectory}.

Define the moduli space of (parametrized) flow lines joining $x$ to $y$ $$\M(x,y)=\{\text{connecting trajectories between x and y}\}.$$ This space $\M$
is what we will study using analysis techniques. Unfortunately,
because translation reparametrization preserves connecting
trajectories, $\M(x,y)$ is infinite (if it is non-empty), so
ultimately we will want to consider the quotient
\begin{equation*}
  \hat{\mathscr{M}}=\mathscr{M}/\R.
\end{equation*}
These moduli spaces feature in the definition of the ``Morese
complex;'' one can define a chain complex $(CM_\ast,\bd)$ (over
$\Z/2\Z$), which is freely generated in each degree
$$CM_{i}:=\Z/2\Z\,\langle \text{\text{critical points of Morse index
  i}}\rangle,$$ and whose boundary map $\bd$ is defined by ``counting flow lines''
$$\bd:CM_{i+1}\to CM_i:\langle \bd x,y\rangle=\#\hat{\M}(x,y).$$

\begin{clear}{Remark}
Without continuation maps, homotopies between continuation maps etc,
the theory is incomplete. We won't consider these ideas yet -- for now
we are preoccupied with establishing foundations for the construction
of the prerequisite moduli spaces.
\end{clear}

1) Need to construct a Banach bundle $\E\to\P$ where $\P$ is a Banach manifold and a section $s$ and such that $s^{-1}(0)=M(x,y)$.

$\P$ is a Banach manifold of pathes $x\to y$. The charts of $\P$ looks llike the following:

Assume $\gamma:\R\to  M$ smooth for simplicity with some kind of asymptotics condition at $\pm\infty$ which makes them converge to $x$ and $y$.

$U\subset X(\gamma^*TM)$ open where $X$ denotes a Banach space containing $C_c(\R;\gamma^*TM)$ (I assume they are dense).

$\implies U\xrightarrow{arrow}\P$. Moreover the aymptotic condition that $\gamma$ satisfied should also be satified for smooth paths inside the image of $U$.

Two solutions:

\begin{enumerate}

\item Schwarz's book: roughly, replace the domain $\R$ with $[-1,1]$, noting that $(-1,1)$ is diffeomorphic to $\R$ and use the smooth structure on $[-1,1]$, asyptotic condition is now $-1\mapsto x, +1\mapsto y$.

\[
X=H_\R^{1,2}(\gamma^* small enough disk in TM)
\]


\item Floer's approach: Use slightly modified Sobolev norm.
$W^{k,p,\sigma,\sigma_+}(\R_t,\R):=\{f\in W^{k,p}_{loc}|\norm{f}_{k,p,\sigma,\sigma_+}<\infty\}$

$\norm{f}_{k,p,\sigma,\sigma_+}=\norm{exp((\beta(t)\sigma_++\beta(-t)\sigma_-)t)\cdot f}_{k,p}$

This is a Banach space.

Aymptotic condition near $\pm\infty$ is $\gamma$ should be given by exponentiating a family of vector of at $TM_x$ on $(-\infty, \rho)$. and satisfy true exponential decay condition for some $\sigma_-<0$.

The above characterization of chart of Banach space gives tangent spaces of pathes that satisfy the aymptotic condition. (Need Proof).

\end{enumerate}

The vector bundle is locally just $W^{k-1,p,\sigma,\sigma_+}(\R_t,\R)$ and the section goes as $\bd/\bd t+\grad\circ$

2) With a lot of pain, most previous work generalize here. But the linearization is a little more involved because because we are exponentiallly sections not just adding them: the right operator is $\bd/\bd t+\Gamma$ the Christoffel symbols are involved.

3) Consequences of parametric transversality theorem for Banach manifolds.

\begin{prop}
$G,M$ Banach manifolds, $E\to M$ is a Banach space bundle. $\varphi:G\times M\to E$ smooth $G$-parametric sections (Family of sections, try to find a generic one such that intersection with $0$ is a true manifold), there is a countable trivialization, i.e.
\[
\{(U,\psi|U\subset M open, \psi:E|_{U}\to E_u\times U\}
\]
such that for every member:
\begin{itemize}
\item $0$ is a regualr value of $pr\circ\psi\circ\varphi:G\times U\to E_u$
\item $pr\circ\psi\circ\varphi_g:U\to E_u$ is a Fredholm map of Banach manifold (All tangent map are) of index $r$ for every $g\in G$.

\end{itemize}

Then there is a $G$-set (Baire set) $\Sigma\subset G$ such that $Z_g=\{\varphi_g(m)=0\}$ is a closed submanifold of $M$ for $g\in\Sigma$ 

For us, $G=$ Banach manifold of Riemannian metric, If we choose $G$ large enough, 1 is indeed satified but it is not a general theorem, need to make computations.





\end{prop} 

\end{document}